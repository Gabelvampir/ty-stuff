% !TEX TS-program = pdflatex
% !TEX encoding = UTF-8 Unicode

% This file is a template using the "beamer" package to create slides for a talk or presentation
% - Giving a talk on some subject.
% - The talk is between 15min and 45min long.
% - Style is ornate.

% MODIFIED by Jonathan Kew, 2008-07-06
% The header comments and encoding in this file were modified for inclusion with TeXworks.
% The content is otherwise unchanged from the original distributed with the beamer package.

\documentclass[handout]{beamer}
%\documentclass{beamer}

% Copyright 2004 by Till Tantau <tantau@users.sourceforge.net>.
%
% In principle, this file can be redistributed and/or modified under
% the terms of the GNU Public License, version 2.
%
% However, this file is supposed to be a template to be modified
% for your own needs. For this reason, if you use this file as a
% template and not specifically distribute it as part of a another
% package/program, I grant the extra permission to freely copy and
% modify this file as you see fit and even to delete this copyright
% notice. 


\mode<presentation>
{
  %\usetheme{Rochester}
  %\usetheme{Marburg}
  %\usetheme{Madrid}
  %\usetheme{Luebeck}
  \usetheme{Frankfurt}
  %\usetheme{Dresden}
  %\usetheme{Copenhagen}
  %\usetheme{Berlin}
  %\usetheme{Warsaw}
\useinnertheme{circles}

  %\setbeamercovered{transparent}
  % or whatever (possibly just delete it)

	\setbeamertemplate{navigation symbols}{}%remove navigation symbols
}

\usepackage[ngerman]{babel}
\usepackage[utf8x]{inputenc}

\usepackage{graphicx}
\usepackage{wrapfig}

%\usepackage{times}
\usepackage{lmodern}
\usepackage[T1]{fontenc}
% Or whatever. Note that the encoding and the font should match. If T1
% does not look nice, try deleting the line with the fontenc.

%code listing stuff
\usepackage{listings}
\usepackage{color}

\definecolor{mygreen}{rgb}{0,0.6,0}
\definecolor{mygray}{rgb}{0.5,0.5,0.5}
\definecolor{mymauve}{rgb}{0.58,0,0.82}

\lstset{ %
  backgroundcolor=\color{white},   % choose the background color; you must add \usepackage{color} or \usepackage{xcolor}
  basicstyle=\footnotesize,        % the size of the fonts that are used for the code
  breakatwhitespace=false,         % sets if automatic breaks should only happen at whitespace
  breaklines=true,                 % sets automatic line breaking
%  captionpos=b,                    % sets the caption-position to bottom
  commentstyle=\color{mygreen},    % comment style
  deletekeywords={...},            % if you want to delete keywords from the given language
  escapeinside={\%*}{*)},          % if you want to add LaTeX within your code
  extendedchars=true,              % lets you use non-ASCII characters; for 8-bits encodings only, does not work with UTF-8
  frame=single,                    % adds a frame around the code
  keepspaces=true,                 % keeps spaces in text, useful for keeping indentation of code (possibly needs columns=flexible)
  keywordstyle=\color{mygreen},       % keyword style
  language=Cobol,                 % the language of the code
  morekeywords={*,...},            % if you want to add more keywords to the set
  numbers=none,                    % where to put the line-numbers; possible values are (none, left, right)
  numbersep=5pt,                   % how far the line-numbers are from the code
  numberstyle=\tiny\color{mygray}, % the style that is used for the line-numbers
  rulecolor=\color{black},         % if not set, the frame-color may be changed on line-breaks within not-black text (e.g. comments (green here))
  showspaces=false,                % show spaces everywhere adding particular underscores; it overrides 'showstringspaces'
  showstringspaces=false,          % underline spaces within strings only
  showtabs=false,                  % show tabs within strings adding particular underscores
  stepnumber=2,                    % the step between two line-numbers. If it's 1, each line will be numbered
  stringstyle=\ttfamily\color{red},     % string literal style
  tabsize=2,                       % sets default tabsize to 2 spaces
  title=\lstname                   % show the filename of files included with \lstinputlisting; also try caption instead of title
}


\title[COBOL-\"Uberblick]   % (optional, use only with long paper titles)
{COBOL}

\subtitle
{Die erste Programmiersprache für Wirtschaftsanwendungen} % (optional)

\author % (optional, use only with lots of authors)
{Sebastian Deußer}

%\subject{COBOL \"Ubersicht}
% This is only inserted into the PDF information catalog. Can be left
% out. 

% If you have a file called "university-logo-filename.xxx", where xxx
% is a graphic format that can be processed by latex or pdflatex,
% resp., then you can add a logo as follows:

% \pgfdeclareimage[height=0.5cm]{university-logo}{university-logo-filename}
% \logo{\pgfuseimage{university-logo}}

% If you wish to uncover everything in a step-wise fashion, uncomment
% the following command: 
%\beamerdefaultoverlayspecification{<+->}


\begin{document}

\begin{frame}
  \titlepage
\end{frame}


\section{Übersicht}
\subsection{Namensbedeutung}
\begin{frame}{Namensbedeutung}
	\begin{columns}[c]
		\begin{column}{4cm}
			\includegraphics[width=4cm]{CobolCogLogo}\\ 
			\includegraphics[width=4cm]{GnuCOBOLLogoTransparent}
		\end{column}
		\begin{column}{6cm}
			\includegraphics[width=2.5cm]{DOSMSCOBOL45} \hspace{1.5pt}
			\includegraphics[width=2.5cm]{StandardCobol}
		\end{column}
	\end{columns}
	\pause
	\begin{itemize}[<+->]
		\item
			COBOL steht für COmmon Business-Oriented Language.
		\item
			entwickelt für betriebswirtschaftliche Programme (im Gegensatz zum technisch-wissenschaftlichen Fokus anderer Sprachen)
	\end{itemize}
\end{frame}


\subsection{Historische Entwicklung}
\begin{frame}{Anfänge}
	\begin{itemize}[<+->]
		\item
			entwickelt durch Arbeitsgruppe in der 2. H\"alfte von 1959
		\item
  \begin{minipage}[t]{0.7\textwidth}
%	\vspace{-5pt}
	federführend war Grace Hopper, Erfinderin des ersten Compilers (A-0) und Mitentwicklerin fr\"uher Computer (z.B. Harvard Mark I+II, UNIVAC I)
	\end{minipage}\hspace{2pt}
	\begin{minipage}[t]{0.15\textwidth}
	\vspace{-12pt}%\raggedright
    \includegraphics[width=1.5cm]{Grace_Hopper_small}
	\end{minipage}
		\item
			basierte auf Hoppers FLOW-MATIC, IBMs COMTRAN (``Business-Version'' von FORTRAN) und Honeywells FACT
		\item
		erster Standard (COBOL 60) festgelegt am 03. Januar 1960, danach individuelle (inkompatible) Erweiterungen von verschiedenen Firmen
	\end{itemize}
\end{frame}

\begin{frame}{Versionen des Standards}
	\begin{itemize}[<+->]
		\item
			ANS COBOL 1968: ANSI Standard um die verschiedenen \"Anderungen nach 1959 zu einer neuen Basis zusammenzufassen
		\item
			COBOL 1974: 2. ANSI Standard, neue Features wie Datei-Organisation, Report Modul; Einführung der Möglichkeit ein Programm in Funktionen zu unterteilen; Abschaffung von Features, deshalb inkompatibel zu vorherigen Standards
		\item
			COBOL 1985: \"uberarbeiteter ANSI Standard, Einf\"uhrung G\"ultigkeitsbereich-Terminatoren (z.B. \texttt{END-IF}), neue Features geschachtelte Unterprogramme, neue Statements, die Operatoren >= und <=; ersetzen von selbstmodifizierenden Code (\texttt{ALTER}) durch Referenz Modifikation
	\end{itemize}
\end{frame}

\begin{frame}{Entwicklung um und nach 2000}
	\begin{itemize}[<+->]
		\item
			viele Programm aus den 80ern und davor verwendeten in Records Datumsfelder mit fester 2-stelliger Jahreszahl
		\item
			=> um 1999/2000 Massensterben unter COBOL Anwendungen wegen Y2K-Bug; Programme mussten entweder grundlegend überarbeitet oder ganz ersetzt werden
		\item
			COBOL 2002: neue zeitgemäße Features wie Locale/Unicode-Unterstützung, Objektorientierung zur Einbindung in Java und .NET-Frameworks, Fließkommaunterstützung
		\item
			Ausblick: COBOL 20XX Standard in Entwicklung (erwartet 2014-6), baut primär Objektorientierung aus und passt Datentypen an neue Standards an
		\end{itemize}
\end{frame}


\section{Einsatz}
\subsection{Einsatzgebiete}
\begin{frame}{Einsatzgebiete}
	\pause
	\begin{columns}[t]
		\begin{column}{4cm}
			\includegraphics[width=4cm]{Deutsche-boerse}
		\end{column}
	\pause
		\begin{column}{3cm}
			\includegraphics[width=3cm]{sparda_logo}
		\end{column}
	\pause
		\begin{column}{3.5cm}
			\includegraphics[width=3.5cm]{california-state-flag1} 
		\end{column}
	\end{columns}
	\pause
	\begin{itemize}[<+->]
		\item
			Haupteinsatzgebiet ist betriebswirtschaftliche Datenverarbeitung
		\item
			bei klassischer Aufteilung nach Benutzerschnittstelle, Verabreitungsteil und Datenhaltungsteil stellt COBOL den Verarbeitungsteil eines EDV-Programms
	\end{itemize}
\end{frame}


\subsection{heutiger Einsatz}
\begin{frame}{heutiger Einsatz}
	\begin{itemize}[<+->]
		\item
			COBOL hatte wenig Einfluss auf sp\"atere Programmiersprachen da Fokus auf relativ simple Algorithmen und hohes I/O-Volumen akademisch uninteressant (Ausnahme: SAPs Programmiersprache ABAP)
		\item
			geschätzte 70-80\% aller Geschäftstransaktion involvieren COBOL Programme, oft als Teil von jahrzehntelang gewachsenen Systemen
		\item
			komplette Neuentwicklungen nur noch selten in COBOL
		\item
			2010 geschätzt \"uber 40 Milliarden Zeilen COBOL Code in Industrieprogrammen verwendet, Wachstumsrate 4 Milliarden Zeilen Code pro Jahr
	\end{itemize}
\end{frame}


\section{Vor- und Nachteile}
\subsection{Vorteile}
\begin{frame}{Vorteile}
	\begin{itemize}[<+->]
		\item
			entworfen um Programme auf verschiedener Hardware laufen zu lassen ohne große Ver\"anderung des Codes
		\item
			durch mehrstufiges Design der einzelnen Programmsprachen-Module lassen sich COBOL-Programme auf sehr eingeschr\"ankter Hardware mit geringen Anpassungen ausf\"uhren
		\item
			der ANSI-Standard wird durchschnittlich alle 10-15 Jahre an neue Gegebenheiten angepasst, z.B. kam 2002 Unterstüzung für Objektorientierung dazu
		\item
			es gibt einen kostenlosen quelloffenen Compiler namens GNU COBOL (ehemals OpenCOBOL) für POSIX-kompatible Betriebssysteme
	\end{itemize}
\end{frame}


\subsection{Nachteile}
\begin{frame}{Nachteile}
	\begin{itemize}[<+->]
		\item
		    bis COBOL 74 gab es keine M\"oglichkeit Programme zu strukturieren, z.B. waren alle Variablen global
		\item
			aufgrund vieler Eigenentwicklungen von Firmen und anderen Gruppen gibt es viele verschiedene Standards von COBOL die teilweise inkompatibel zueinander sind
		\item
			da die Syntax sich an geschriebenem Englisch orientiert werden COBOL Programme schnell recht langwierig zu schreiben
		\item
			COBOL Programmierer sterben langsam aus
	\end{itemize}
\end{frame}


\section{Praxis}
\subsection{Entwicklungsumgebung}
\begin{frame}{Entwicklungsumgebung}
	\begin{itemize}[<+->]
		\item
			viele Computer/Betriebsystemhersteller (z.B. IBM, Siemens, Unisys, HP) entwickelten (und vertreiben teilweise immer noch) eigene Compiler, oft mit eigenen Erweiterungen des Standards
		\item
			es gibt verschiedene Codegeneratoren die COBOL-Programme oder Teile davon generieren um die Entwicklungsarbeit zu erleichtern
		\item
			Plugins für viele gängige IDEs, z.B. cobolclipse für Eclipse
	\end{itemize}
\end{frame}

\begin{frame}{Entwicklungsumgebung (2)}
	\begin{itemize}
		\item
			verbreiteste (kommerzielle) COBOL IDE ist Visual COBOL von Micro Focus; besitzt moderne Features wie Generierung von Java Byte-Code, .NET Unterstützung und Webservices
		\includegraphics[width=9cm]{VisualCOBOL2}
	\end{itemize}
\end{frame}


\subsection{Sprachspezifikation}
\begin{frame}{Sprachspezifikation}
	\begin{itemize}[<+->]
		\item
			COBOL wurde so entworfen das der Code relativ lesbarem Englisch entspricht z.B. ``\texttt{ADD b TO c GIVING a}'' (a = b + c)
		\item
			importieren von Programmteilen als sgn. Copybooks mit \texttt{COPY} Befehl, vergleichbar mit \texttt{\#include} in C u.\"A.
		\item
			Kontrollstrukturen:
			\begin{itemize}[<+->]
				\item \texttt{IF...ELSE}: funktioniert wie erwartet
				\item \texttt{EVALUATE}: vergleichbar mit CASE/Switch (eingeführt mit COBOL 85)
				\item \texttt{PERFORM}: vergleichbar mit for-Schleifen
				\item \texttt{GO TO}: Standard GOTO, optional mit Bedingung
			\end{itemize}
	\end{itemize}
\end{frame}

\begin{frame}{Datenstrukturen}
	\begin{itemize}[<+->]
		\item
			Grunddatenstruktur ist ein sogenannter Record, ein multidimensionales heterogenes Array
		\item
			Records k\"onnen mit homogenen Arrays gemischt verwendet werden, ein Array kann also Inhalt eines Records sein und umgekehrt
		\item
			über die \texttt{PICTURE} Klausel kann man den Inhalt (inklusive Feldgrößen) von Records exakt definieren
		\item
			Variablen müssen bei der Deklaration typisiert werden, allerdings hat man Dank Records und der \texttt{PICTURE} Klausel vergleichsweise viel Freiheit dabei
	\end{itemize}
\end{frame}

\begin{frame}{Aufteilung eines Programms}
	\begin{itemize}[<+->]
		\item
			jedes Programm besteht aus 4 Teilen (\texttt{DIVISION}) zur Trennung von:
			\begin{itemize}[<+->]
			 	\item hardwareabh\"angigem und unabh\"angigem Code
				\item Algorithmen Beschreibungen und Daten Beschreibungen
			 \end{itemize}
		\item
			\texttt{IDENTIFICATION}: beginnt jedes Programm, benennt Programm und den Autor (Autor optional), sonstige Kommentare als Dokumentation
		\item
			\texttt{ENVIRONMENT}: beinhaltet maschinenabhängige Programmspezifikationen wie z.B. Verbindungen zwischen dem Programm und externen Daten
		\item
			\texttt{PROCEDURE}: beinhaltet die Algorithmen
		\item
			\texttt{DATA}: beinhaltet die Daten Beschreibungen
	\end{itemize}
\end{frame}

\setbeamercovered{transparent}
\def\beamertemplatetransparentcoveredmedium{\setbeamercovered{transparent=20}}
\beamertemplatetransparentcoveredmedium
\begin{frame}[fragile]{Code-Beispiele}
	\onslide<1-1>
	\noindent\begin{minipage}{.44\textwidth}
		\begin{lstlisting}{Hello World}
			IDENTIFICATION DIVISION.
			PROGRAM-ID. HELLO-WORLD.
			PROCEDURE DIVISION.
			 DISPLAY "Hello, world."
			 STOP RUN.
		\end{lstlisting}
	\end{minipage}\hfill\pause
	\onslide<2-2>
	\noindent\begin{minipage}{.53\textwidth}
		\begin{lstlisting}{Schleifen}
			PERFORM VARYING i FROM 0 BY 1
			 UNTIL i >= 10
			 DISPLAY i
			END-PERFORM
		\end{lstlisting}
	\end{minipage}\pause
	\onslide<3-3>
	\lstset{numbers=left,firstnumber=1}
	\begin{lstlisting}{EVALUATE}
		EVALUATE True
		 WHEN Nenner > 0
		  COMPUTE Zahl = Zaehler / Nenner
		 WHEN Nenner < 0
		  COMPUTE Zahl = Zaehler / Nenner * -1
		 WHEN OTHER
		  DISPLAY "Fehler"
		  MOVE 0 TO Zaehler
		END-EVALUATE
	\end{lstlisting}
\end{frame}


\section*{Quellen}
\begin{frame}{Quellen}
	\begin{enumerate}
		\item COBOL Artikel englische Wikipedia
			\url{http://en.wikipedia.org/wiki/COBOL}
		\item COBOL Artikel deutsche Wikipedia
			\url{http://de.wikipedia.org/wiki/COBOL}
		\item
			Terrence W. Pratt - Programming Languages: Design and Implementation (Prentice Hill 1975)
		\item OpenCOBOL Programmers Guide \url{http://opencobol.add1tocobol.com/OpenCOBOL\%20Programmers\%20Guide.pdf}
	\end{enumerate}
\end{frame}


\end{document}