% !TEX TS-program = pdflatex
% !TEX encoding = UTF-8 Unicode

% This file is a template using the "beamer" package to create slides for a talk or presentation
% - Giving a talk on some subject.
% - The talk is between 15min and 45min long.
% - Style is ornate.

% MODIFIED by Jonathan Kew, 2008-07-06
% The header comments and encoding in this file were modified for inclusion with TeXworks.
% The content is otherwise unchanged from the original distributed with the beamer package.

\documentclass{beamer}


% Copyright 2004 by Till Tantau <tantau@users.sourceforge.net>.
%
% In principle, this file can be redistributed and/or modified under
% the terms of the GNU Public License, version 2.
%
% However, this file is supposed to be a template to be modified
% for your own needs. For this reason, if you use this file as a
% template and not specifically distribute it as part of a another
% package/program, I grant the extra permission to freely copy and
% modify this file as you see fit and even to delete this copyright
% notice. 


\mode<presentation>
{
  %\usetheme{Rochester}
  %\usetheme{Marburg}
  %\usetheme{Madrid}
  %\usetheme{Luebeck}
  \usetheme{Frankfurt}
  %\usetheme{Dresden}
  %\usetheme{Copenhagen}
  %\usetheme{Berlin}
  %\usetheme{Warsaw}

  %\setbeamercovered{transparent}
  % or whatever (possibly just delete it)

	\setbeamertemplate{navigation symbols}{}%remove navigation symbols
}


\usepackage[ngerman]{babel}
% or whatever

\usepackage[utf8x]{inputenc}
% or whatever

%\usepackage{times}
\usepackage{lmodern}
\usepackage[T1]{fontenc}
% Or whatever. Note that the encoding and the font should match. If T1
% does not look nice, try deleting the line with the fontenc.


\title[COBOL-\"Uberblick]   % (optional, use only with long paper titles)
{COBOL}

\subtitle
{Die erste Programmiersprache für Wirtschaftsanwendungen} % (optional)

\author % (optional, use only with lots of authors)
{Sebastian Deußer}
% - Use the \inst{?} command only if the authors have different
%   affiliation.

%\institute[Universities of Somewhere and Elsewhere] % (optional, but mostly needed)
%{
%  \inst{1}%
%  Department of Computer Science\\
%  University of Somewhere
%  \and
%  \inst{2}%
%  Department of Theoretical Philosophy\\
%  University of Elsewhere}
% - Use the \inst command only if there are several affiliations.
% - Keep it simple, no one is interested in your street address.

%\date[Short Occasion] % (optional)
%{Date / Occasion}

%\subject{COBOL \"Ubersicht}
% This is only inserted into the PDF information catalog. Can be left
% out. 



% If you have a file called "university-logo-filename.xxx", where xxx
% is a graphic format that can be processed by latex or pdflatex,
% resp., then you can add a logo as follows:

% \pgfdeclareimage[height=0.5cm]{university-logo}{university-logo-filename}
% \logo{\pgfuseimage{university-logo}}



% Delete this, if you do not want the table of contents to pop up at
% the beginning of each subsection:



% If you wish to uncover everything in a step-wise fashion, uncomment
% the following command: 

%\beamerdefaultoverlayspecification{<+->}


\begin{document}

\begin{frame}
  \titlepage
\end{frame}

%\begin{frame}{Namensbedeutung}
  % You might wish to add the option [pausesections]
%\end{frame}


% Since this a solution template for a generic talk, very little can
% be said about how it should be structured. However, the talk length
% of between 15min and 45min and the theme suggest that you stick to
% the following rules:  

% - Exactly two or three sections (other than the summary).
% - At *most* three subsections per section.
% - Talk about 30s to 2min per frame. So there should be between about
%   15 and 30 frames, all told.

\section{Übersicht}

\subsection{Namensbedeutung}

\begin{frame}{Namensbedeutung}
	\begin{itemize}
		\item
			COBOL steht für COmmon Business-Oriented Language.
		\item
			Entwickelt für betriebswirtschaftliche Programme (im Gegensatz zum Fokus auf technische-wissenschaftliche Anwendungen vorheriger Programmiersprachen).
	\end{itemize}
\end{frame}

\subsection{Historische Entwicklung}

\begin{frame}{Anfänge}
	\begin{itemize}
		\item
			Entwickelt durch eine Arbeitsgruppe in der 2. H\"alfte von 1959.
		\item
			Federführend war Grace Hopper, die Erfinderin des ersten Compilers (A-0) und Mitentwicklerin einiger fr\"uher Computer (zB Harvard Mark I+II, UNIVAC I)
		\item
			Entwicklung basierte auf Hoppers FLOW-MATIC und IBMs COMTRAN (``Business-Version'' von FORTRAN)
		\item
		Offizielle Festlegung des Standards (bekannt als COBOL 60) am 03. Januar 1960, danach individuelle (inkompatibele) Erweiterungen durch verschiedene Firmen und Universit\"aten
	\end{itemize}
\end{frame}

\begin{frame}{Versionen des Standards}
	\begin{itemize}
		\item
			ANS COBOL 1968: ANSI Standard um die verschiedenen \"Anderungen nach 1959 wieder zu einer kompatiblen Basis zusammenzufassen.
		\item
			COBOL 1974: dieser zweite ANSI Standard f\"uhrte neue Features ein wie Datei-Organisation, das Report Modul und das Segmentierungsmodul, schaffte allerdings auch Features ab wie das \texttt{NOTE} Statement. Inkompatibel zu vorherigen Standards
		\item
			COBOL 1985: dieser \"uberarbeitete ANSI Standard f\"uhrte G\"ultigkeitsbereich Terminatoren wie \texttt{END-IF} ein. Ebenso Features wie geschachtelte Unterprogramme, die Statements \texttt{CONTINUE}, \texttt{EVALUATE} und \texttt{INITIALIZE}, die Operatoren >= und <= sowie Referenz Modifikation
	\end{itemize}
\end{frame}

\begin{frame}{Versionen des Standards (2)}
		\begin{itemize}
			\item
				COBOL 2002 und objekorientiertes COBOL: Anfang der 90er wurde entschieden das dem n\"achste Standard von COBOL Objektorientierung hinzugef\"ugt werden sollte.
		\end{itemize}
\end{frame}

\section{Einsatz}

\subsection{Einsatzgebiete}

\begin{frame}{Einsatzgebiete}
	\begin{itemize}
		\item
			Haupteinsatzgebiet ist betriebswirtschaftliche Datenverarbeitung
		\item
			Bei klassischer Aufteilung eines EDV-Programms nach Benutzerschnittstelle, Verabreitungsteil und Datenhaltungsteil stell COBOL \"ublicherweise den Verarbeitungsteil
	\end{itemize}
\end{frame}

\subsection{heutiger Einsatz}

\begin{frame}{heutiger Einsatz}
	\begin{itemize}
		\item
			COBOL hatte auf die meisten sp\"ateren Programmiersprachen nicht viel Einfluss da der Fokus auf relativ simple Algorithmen und hohes I/O-Volumen akademisch uninteressant waren
		\item
			die meisten heute noch verwendeten COBOL Anwendungen sind Teil \"uber Jahrzehnte gewachsene Systeme, komplette Neuentwicklungen finden nur noch selten in COBOL statt
		\item
			aber immer noch verwendet in vielen großen Rechnungssystemen, da die Neuentwicklung dieser Systeme sehr teuer wäre, u.a. wegen oft fehlender Dokumentation
		\item
			es wird gesch\"atzt das 2009 \"uber 40 Milliarden Zeilen COBOL Code in Industrieprogrammen verwendet wird, mit einer Wachstumsrate von 4 Milliarden Zeilen Code pro Jahr
		\item
			SAPs Programmiersprache ABAP wurde sehr stark von COBOL beeinflusst
	\end{itemize}
\end{frame}

\section{Vor- und Nachteile}

\subsection{Vorteile}

\begin{frame}{Vorteile}
	\begin{itemize}
		\item
			entworfen um Programme auf verschiedener Hardware laufen zu lassen ohne große Ver\"anderung des Codes
		\item
			durch mehrstufiges Design der einzelnen Programmsprachen-Module lassen sich COBOL-Programme auf sehr eingeschr\"ankter Hardware mit geringen Anpassungen ausf\"uhren
		\item
			es gibt einen kostenlosen quelloffenen Compiler namens GNU COBOL (ehemals OpenCOBOL) für POSIX-kompatible Betriebssysteme
	\end{itemize}
\end{frame}

\subsection{Nachteile}

\begin{frame}{Nachteile}
	\begin{itemize}
		\item
		    bis COBOL 74 gab es keine M\"oglichkeit Programme zu strukturieren, zB waren alle Variablen global
		\item
			aufgrund vieler Eigenentwicklungen von Firmen und anderen Gruppen gibt es viele verschiedene Standards von COBOL die teilweise inkompatibel zueinander sind
	\end{itemize}
\end{frame}

\section{Praxis}

\subsection{Entwicklungsumgebung}

\begin{frame}{Entwicklungsumgebung}
	\begin{itemize}
		\item
			da COBOL auf Firmen ausgerichtet ist und aus der Zeit stammt in der jeder Computerhersteller sein eigenes (meistens inkompatibles) Betriebssystem hatte gibt/gab es von jedem dieser Hersteller (zB IBM, Siemens, Unisys, HP) eigene Compiler, teilweise f\"ur verschiedene Standards
		\item
			es gibt verschiedene Codegeneratoren die COBOL-Programme oder Teile davon generieren um die Entwicklungsarbeit zu erleichtern
	\end{itemize}
\end{frame}

\subsection{Sprachspezifikation}

\begin{frame}{Sprachspezifikation}
	\begin{itemize}
		\item
			COBOL wurde so entworfen das der Code relativ lesbarem Englisch entspricht (z.B. \texttt{ADD b TO c GIVING a} f\"ur a = b + c)
		\item
			jedes Programm besteht aus 4 Teilen um hardwareabh\"angigen Code von unabh\"angigem und Algorithmen Beschreibungen von Daten Beschreibungen zu trennen
	\end{itemize}
\end{frame}

\begin{frame}{Aufteilung eines Programms}
	\begin{itemize}
		\item
			\texttt{IDENTIFICATION}: beginnt jedes Programm und dient dazu das Programm und den Autor zu benennen und um andere Kommentare als Dokumentation anzugeben
		\item
			\texttt{ENVIRONMENT}: beinhaltet die maschinenabhängigen Programmspezifikationen wie zB die Verbindungen zwischen dem Programm und externen Daten
		\item
			\texttt{PROCEDURE}: beinhaltet die Algorithmen
		\item
			\texttt{DATA}: beinhaltet die Daten Beschreibungen
	\end{itemize}
\end{frame}

\begin{frame}{Quellen}
	\begin{itemize}
		\item
			\url{http://en.wikipedia.org/wiki/COBOL}
		\item
			\url{http://de.wikipedia.org/wiki/COBOL}
		\item
			Terrence W. Pratt - Programming Languages: Design and Implementation (Prentice Hill 1975)
		\item OpenCOBOL Programmers Guide
	\end{itemize}
\end{frame}

%\begin{frame}{Make Titles Informative.}
%
%  You can create overlays\dots
%  \begin{itemize}
%  \item using the \texttt{pause} command:
%    \begin{itemize}
%    \item
%      First item.
%      \pause
%    \item    
%      Second item.
%    \end{itemize}
%  \item
%    using overlay specifications:
%    \begin{itemize}
%    \item<3->
%      First item.
%    \item<4->
%      Second item.
%    \end{itemize}
%  \item
%    using the general \texttt{uncover} command:
%    \begin{itemize}
%      \uncover<5->{\item
%        First item.}
%      \uncover<6->{\item
%        Second item.}
%    \end{itemize}
%  \end{itemize}
%\end{frame}
%
%
%
%
%\section*{Summary}
%
%\begin{frame}{Summary}
%
%  % Keep the summary *very short*.
%  \begin{itemize}
%  \item
%    The \alert{first main message} of your talk in one or two lines.
%  \item
%    The \alert{second main message} of your talk in one or two lines.
%  \item
%    Perhaps a \alert{third message}, but not more than that.
%  \end{itemize}
%  
%  % The following outlook is optional.
%  \vskip0pt plus.5fill
%  \begin{itemize}
%  \item
%    Outlook
%    \begin{itemize}
%    \item
%      Something you haven't solved.
%    \item
%      Something else you haven't solved.
%    \end{itemize}
%  \end{itemize}
%\end{frame}


\end{document}


