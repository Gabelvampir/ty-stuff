%% LyX 2.0.6 created this file.  For more info, see http://www.lyx.org/.
%% Do not edit unless you really know what you are doing.
\documentclass[english,nohyper]{tufte-handout}
\usepackage[T1]{fontenc}
\usepackage[latin9]{inputenc}

\makeatletter

%%%%%%%%%%%%%%%%%%%%%%%%%%%%%% LyX specific LaTeX commands.

\title{Linux Text-Mode Survival Guide}

\makeatother

\usepackage{babel}
\begin{document}
\maketitle

\section{Basisbefehle:}
\begin{description}
\item [{ls}] - zeigt den Inhalt von Ordnern an. Optionen: -a zeigt auch
versteckte Dateien an, -l gibt auch Attribute wie Gr��e und Leserechte
aus
\item [{cd}] - wechselt Verzeichnisse, cd .. wechselt eine Ebene nach oben
\item [{cat}] - gibt den Inhalt einer Datei aus
\item [{less}] - zeigt den Inhalt einer Datei scrollbar an
\end{description}

\section{n�tzliche Programme:}

Text-Editoren:
\begin{description}
\item [{nano}] - gut geeignet f�r Anf�nger, aber nicht so m�chtig wie andere
Text-Editoren
\item [{mcedit}] - Editor des Midnight Commanders, orientiert sich am Norton
Commander Editor
\item [{vim}] - f�r Anf�nger ungewohnt zu bedienen, besitzt m�chtige Features
wie z.B. Block Ersetzungen und einige Erweiterungen
\end{description}
Datei-Browser:
\begin{description}
\item [{mc}] - Midnight Commander: orientiert sich an Norton Commander
f�r DOS, hat �ber die Jahre aber einige Zusatzfunktionen erhalten\end{description}

\end{document}
