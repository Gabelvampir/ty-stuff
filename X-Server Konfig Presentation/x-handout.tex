%% LyX 2.0.6 created this file.  For more info, see http://www.lyx.org/.
%% Do not edit unless you really know what you are doing.
\documentclass[english]{tufte-handout}
\usepackage[T1]{fontenc}
\usepackage[latin9]{inputenc}

\makeatletter

%%%%%%%%%%%%%%%%%%%%%%%%%%%%%% LyX specific LaTeX commands.

\title{Linux Text-Mode Survival Guide}

\makeatother

\usepackage{babel}
\begin{document}
\maketitle

\section*{Basisbefehle:}
\begin{itemize}
\item \emph{ls} - zeigt den Inhalt von Ordnern an. Optionen: -a zeigt auch
versteckte Dateien an, -l gibt auch Attribute wie Gr��e und Leserechte
aus
\item \emph{cd} - wechselt Verzeichnisse, cd .. wechselt eine Ebene nach
oben
\item \emph{cat} - gibt den Inhalt einer Datei aus
\item \emph{less} - zeigt den Inhalt einer Datei scrollbar an
\end{itemize}

\section*{n�tzliche Programme:}


\subsection*{Text-Editoren:}
\begin{itemize}
\item \emph{nano} - gut geeignet f�r Anf�nger, aber nicht so m�chtig wie
andere Text-Editoren
\item \emph{mcedit} - Editor des Midnight Commanders, orientiert sich am
Norton Commander Editor
\item \emph{vim} - f�r Anf�nger ungewohnt zu bedienen, besitzt m�chtige
Features wie z.B. Block Ersetzungen und einige Erweiterungen
\end{itemize}

\subsection*{Datei-Browser:}
\begin{itemize}
\item \emph{mc} - Midnight Commander: orientiert sich an Norton Commander
f�r DOS, hat �ber die Jahre aber einige Zusatzfunktionen erhalten
wie z.B. browsen von FTP und SSH Zug�ngen und �ffnen von Archiven
\end{itemize}

\section*{Spielereien:}


\subsection*{Videos in Textmode und Framebuffer:}
\begin{itemize}
\item \emph{mplayer -vo aa VIDEO }- spielt das Video �ber die Ascii-Art
Bibliothek ab (schwarz-weiss)
\item \emph{mplayer -vo caca VIDEO} - spielt das Video �ber die Colour Ascii-Art
Bibliothek ab
\item \emph{mplayer -vo fbdev2 VIDEO} - spielt das Video �ber das Framebuffer-Device
ab (ben�tigt aktivierten Framebuffer, nicht im reinen Text-Modus)
\end{itemize}

\subsection*{Spiele:}
\begin{itemize}
\item frotz - bekanntester und verbreitester Infocom/Z-Machine Interpreter
f�r Interactive Fiction
\item nethack - eines der �ltesten rogue-likes
\end{itemize}

\part*{Was tun wenn der X-Server ohne Konfiguration nicht startet?}
\begin{enumerate}
\item Herrausfinden was f�r eine Grafikkarte der Rechner der Rechner hat:
lspci | grep VGA
\item Anhand dessen Treiber bestimmen und installieren:\end{enumerate}

\end{document}
