%% LyX 2.0.6 created this file.  For more info, see http://www.lyx.org/.
%% Do not edit unless you really know what you are doing.
\documentclass[english]{scrartcl}
\usepackage[T1]{fontenc}
\usepackage[latin9]{inputenc}
\usepackage{listings}

\makeatletter

%%%%%%%%%%%%%%%%%%%%%%%%%%%%%% LyX specific LaTeX commands.

\title{Linux Text-Mode Survival Guide}
\date{13. Dezember 2013}

\makeatother

\usepackage{babel}
\lstset{basicstyle=\ttfamily}
%%\lstset{frame=simple}
\begin{document}
\maketitle

\section{Basisbefehle:}
\begin{itemize}
\item \emph{ls} - zeigt den Inhalt von Ordnern an. Optionen: \emph{-a} zeigt
auch versteckte Dateien an, \emph{-l} gibt auch Attribute wie Gr��e
und Leserechte aus
\item \emph{cd} - wechselt Verzeichnisse, \emph{cd ..} wechselt eine Ebene
nach oben
\item \emph{rm} - l�scht Dateien
\item \emph{cp} - kopiert Dateien und Verzeichnisse (daf�r ben�tigt man
meistens die Option -r)
\item \emph{mv} - verschiebt Dateien und Verzeichnisse
\item \emph{less} - zeigt den Inhalt einer Datei scrollbar an
\item \emph{man} \emph{BEFEHL }- zeigt die manual-Page des Befehls an die
alle wichtigen Optionen auflisten sollte
\end{itemize}

\section{n�tzliche Programme:}


\subsection{Text-Editoren:}
\begin{itemize}
\item \emph{nano} - gut geeignet f�r Anf�nger, aber nicht so m�chtig wie
andere Text-Editoren
\item \emph{mcedit} - Editor des Midnight Commanders, orientiert sich am
Norton Commander Editor, einfach zu bedienen aber geringer Funktionsumfang
\item \emph{vim} - f�r Anf�nger ungewohnt zu bedienen, besitzt m�chtige
Features wie z.B. Block Ersetzungen und einige Erweiterungen
\end{itemize}

\subsection{Datei-Browser:}
\begin{itemize}
\item \emph{mc} - Midnight Commander, orientiert sich an Norton Commander
f�r DOS, hat �ber die Jahre aber einige Zusatzfunktionen erhalten
wie z.B. browsen von FTP und SSH Zug�ngen und �ffnen von Archiven
\end{itemize}

\subsection{Web-Browser:}
\begin{itemize}
\item \emph{lynx, w3m, elinks} - Text-Mode Web-Browser. Grafiken werden
nicht angezeigt, aber die meisten Webseiten (wie zB Google und die
meisten Foren die bei Fragen zu Linuxproblemen gefunden werden) die
kein Flash oder �hnliches ben�tigen sind benutzbar
\end{itemize}

\section{Spielereien:}


\subsection{Videos in Textmode und Framebuffer:}
\begin{itemize}
\item \emph{mplayer -vo aa VIDEO} - spielt das Video �ber die Ascii-Art
Bibliothek ab (schwarz-weiss)
\item \emph{mplayer -vo caca VIDEO} - spielt das Video �ber die Colour Ascii-Art
Bibliothek ab
\item \emph{mplayer -vo fbdev2 VIDEO} - spielt das Video �ber das Framebuffer-Device
ab (ben�tigt aktivierten Framebuffer, nicht im reinen Text-Modus)
\end{itemize}

\subsection{Spiele:}
\begin{itemize}
\item \emph{frotz} - bekanntester und verbreitester Infocom/Z-Machine Interpreter
f�r Interactive Fiction, ben�tigt Z-Files (Interactive Fiction "ROMs")
\item \emph{adventure} - Adaption des ersten Adventure Colossal Cave Adventure (ADVENT) aus dem bsdgames Paket
\item \emph{tetris-bsd} - ein Text-Mode Tetris aus dem bsdgames Paket
\item \emph{greed} - Ziel des Spiels ist es m�glichst viel der Zahlen zu
``essen'' ohne sich in eine Sackgasse zu man�vrieren. Man bewegt
sich immer so viele Felder weit wie die gerade gegessene Zahl angibt
\item \emph{ninvaders} - ein Space Invaders Klon f�r die Textkonsole
\end{itemize}

\newpage
\section*{Was tun wenn der X-Server ohne Konfiguration nicht startet?}
\begin{enumerate}
\item Falls es bereits eine Konfig gibt diese l�schen (\emph{/etc/X11/xorg.conf}
und Inhalt des Verzeichnisses \emph{/etc/X11/xorg.conf.d}). Mit \emph{startx}
testen ob der X-Server jetzt l�uft.
\item Falls nicht: herausfinden was f�r eine Grafikkarte der Rechner der
Rechner hat: \emph{lspci} | \emph{grep VGA}
\item Anhand dessen Treiber bestimmen und installieren:

\begin{itemize}
\item NVidia Treiber: \emph{nouvaeu} (Open Source Treiber, Paket \emph{xserver-xorg-video-nouveau})
oder \emph{nvidia} (Bin�rtreiber vom Hersteller, Paketname normal
nvidia-driver)
\item Ati/AMD Treiber: \emph{radeon} (Open Source Treiber, Paket \emph{xserver-xorg-video-radeon})
oder \emph{fglrx} (Bin�rtreiber vom Hersteller)
\item Intel: \emph{intel} (Open Source Treiber vom Hersteller im Kernel
integriert, X-Server Teil des Treibers in Paket \emph{xserver-xorg-video-intel})
\end{itemize}
Anmerkung: Wenn gute 3D Beschleunigung nicht dringend ben�tigt wird sollten bei NVidia und Ati Karten erst der Open Source Treiber ausprobiert werden da die Unterst�tzung f�r "�ltere" Grafikchips regelm��ig auf den Bin�rtribern entfernt wird.
\item Mit Text-Editor die Datei \emph{/etc/X11/xorg.conf} editieren. Falls
Datei nicht vorhanden folgendes eintragen (minimale Konfig):



\begin{lstlisting}
Section "ServerLayout"
	Identifier     "Layout0"
	Screen      0  "Screen0"
	InputDevice    "Keyboard0" "CoreKeyboard"
	InputDevice    "Mouse0" "CorePointer"
EndSection

Section "Files"
EndSection

Section "InputDevice"
	Identifier     "Mouse0"
	Driver         "mouse"
	Option         "Protocol" "auto"
	Option         "Device" "/dev/psaux"
	Option         "Emulate3Buttons" "no"
	Option         "ZAxisMapping" "4 5"
EndSection

Section "InputDevice"
	Identifier     "Keyboard0"
	Driver         "kbd"
EndSection

Section "Monitor"
	Identifier     "Monitor0"
	Option         "DPMS"
EndSection

Section "Device"     
	Identifier     "Device0"
	Driver         "nvidia" #hier passenden Treibernamen eintragen
EndSection

Section "Screen"
	Identifier     "Screen0"
	Device         "Device0"
	Monitor        "Monitor0"
	DefaultDepth    24
	SubSection     "Display"
		Depth       24
	EndSubSection
EndSection
\end{lstlisting}

Anmerkung: Bei neueren X-Servern kann man auch erst versuchen nur die \emph{Section "Device"} in die xorg.conf einzutragen.

\item Mit \emph{startx} den X-Server starten. Falls dies nicht geht zur
Fehlersuche die Datei \emph{/var/log/Xorg.0.log} betrachten. Dabei
prim�r nach Errors (gekennzeichnet mit (EE)) und Warnings (gekennzeichnet
mit (WW)) schauen. Falls die Meldungen nichtssagend sind Google (siehe Text-Mode Web-Browser) oder einen Linux-Guru um Hilfe fragen.\end{enumerate}

\end{document}
