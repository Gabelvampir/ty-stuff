\documentclass{beamer}  % For use with beamer v 2.20  
%  options  [handout] 

%  There is a VERY rich set of possible 
%  styles of presentations and color themes.  See the 
%  beamer documentation for a full list of possibilities. 
\usetheme{Madrid}  % JuanLesPins  Rochester  
		   %  Berkeley  Palo Alto    with sidebar and top 
		   % Goettingen  Marburg     with sidebar 
		   %  Copenhagen  Luebeck  Warsaw 
\useinnertheme{rounded}
\usecolortheme{crane}
\usefonttheme{structuresmallcapsserif}


\usepackage{pgfarrows}
%  Don't need to load the pgf package, but it has 
%  has itself some packages you might want, such as 
%  pgfarrows,,pgfnodes,pgfautomata,pgfheaps  
%  See the pgf documentation. 


\usepackage{amsmath,amssymb}

% \beamertemplatetransparentcovereddynamic
 		% overlays that are upcoming are transparent, in 
		% a manner that depends upon how far ahead  they are. 


\title{  Beamer Features}  

\author{Michael Lacey} 
\institute[Georgia Tech]{Georgia Institute of Technology}

\date{\today} 



\begin{document}

\frame{\titlepage} 


\section[Outline]{}  %  
\frame{\tableofcontents}%[pausesections]}

\section{What is Beamer?} 

\frame{\frametitle{What is Beamer?} 
\begin{itemize} 

\item<+-| alert@+>  A Latex package to create structured powerpoint style presentations. 
\item<+-| alert@+>  A `Beamer' is the thing that takes video input from say a computer. 
\item<+-| alert@+> The package has  article, and handout modes as well. 
\item<+-| alert@+> \alert{Works entirely inside of Latex ! }
\item<+-| alert@+>  And is freely available. Google {\tt latex beamer}
\item<+-| alert@+> \alert{Get the TeX file for this presentation to see how things work.} 

\end{itemize} 

}

\frame{
  \frametitle{Features of the Beamer Class}

  \begin{itemize}%<+->%| alert@+>
  \item<+-| alert@+> Easy overlays, creates slides and handouts as well. Read the {\tt beamer} manual. 
  \item<+-| alert@+> Predefined styles and looks of a wide type. 
  \item<+-| alert@+> Theorems, Alerts, Itemizations all handled in a structured way, with personalization possible. 
  \item<+-| alert@+> No external programs needed, and compatiable with AMS math packages. 
  \item<+-| alert@+> Fully supports hypertext features. 
  \item<+-| alert@+>  And is actively being developed, by Till Tantau.
  \end{itemize}
}


\frame{
\frametitle{Where do I get it?} 

\begin{itemize} 
\item<+->  {\tt beamer} is not part of a standard LaTex installation.  Most will need to get the packages. 
\item<+->  From any \href{http://www.ctan.org}{Comprehensive TeX Archive Network} mirror. 
\item<+->  Or easier yet, from the \href{http://latex-beamer.sourceforge.net/}{Sourceforge} website.  
\item<+->  To experiment with the files, put them in one of your local directories, and run pdflatex on 
this or another example file that comes with the distribution. 
\end{itemize} 


}

\section{Some Math} 


\frame{\frametitle{Math Displays} 
Single Line Displays operate in the Usual Way 
$$ \text{} \sum _{i,j=1}^\infty \otimes _{k=1}^{a_{i,j}} M_{k,j}  $$ 

\uncover<2->{
Multiline displays {\em with no tags} can be revealed line by line. }
\begin{align*} 
\uncover<2->{A&=B }
\\
\uncover<3->{&=C}
\\
\uncover<4->{&\le{}D}
\end{align*}
}








\frame{\frametitle{Theorem Like Envioronments} 



\begin{theorem}<2->[Name of the Theorem Comes as Optional Argument] 
$H_b $ is bounded iff there is a bounded function $\beta $ such that $ P_+ b=P_+ \beta $. 
\uncover<3->{ \textcolor{red}{ And a second part of the Theorem }}
\end{theorem}

\uncover<3->{
Notice how the second part of the Theorem was revealed on the next slide.}

}



\frame{\frametitle{ All  Theorem Like Enviornoments } 

The "theorem like examples" include as predefined formats  {\tt theorem}, {\tt corollary}, {\tt proof}, {\tt example},
{\tt examples}, {\tt definition}.  Usage is 


%\verb+ \begin{example}<2->[ A Name for the example]   ...... \end{example}+
}

\frame{
\begin{corollary}<+-> 
 Theorems, Corollaries, and Definitions have the Presentations. 
\end{corollary} 

\begin{proof}<+-> 
 Proofs put the little QED boxes at the end. 
\end{proof} 


\begin{examples}<+-> 
$H_b $ is bounded iff there is a bounded function $\beta $ such that $\ldots$
\end{examples} 

\begin{block}<+->{Your Name Goes Here} 
 Use the {\tt begin{block} } command.
\end{block} 

\begin{alertblock}<+-> { A Key Point }
Use {\tt alertblock} for those key points and examples.
\end{alertblock}

}

\section{Overlay Specifications} 


\frame{\frametitle{Overlay Specification}

Overlay specifications are given in side of {\tt <\ \ >}. Examples are: 


\only<5> { And this only appears on the 5th slide. }

\begin{itemize} 



\item<+>     {\tt <+> } Means that this material should appear on the next slide.   {\tt <+->} means that 
this appears on the next slide, and all subsequent slides. 

\item<3,5>  This will appear on the 3rd and 5th slide, with the command {\tt <3,5>}

\item<2-4>  
\textbf<3>{
gives us shown on 2, 3, 4 slides, and alerted on the 3rd slide. }


%\item<2,4|  alert @3> % {\tt  <2,4|bf\at 3> }
%gives us shown on 2, 3, 4 slides, and alerted on the 3rd slide. 

\item<5>  And use {\tt <+-> } for incremental uncoverings.  Very handy, especially when you 
move things around as you write the file. 

\end{itemize} 
}





\frame{\frametitle{Other commands with action specifications}

Some first words  for the slide 
\only<+>{  {\tt only}: Only appearing on this slide . }
\uncover<+->{ {\tt uncover}:  
Some words  uncovered, and occupying the previous places.  } 
\textbf<+>{ {\tt textbf, textcolor }: \textcolor{blue}{Some words randomly repeated. }}
\uncover<+->{\alert{ {\tt alert}:  Heads up!} }

% \only   \uncover   \textbf    \alert

}




\frame{\frametitle{{\tt enumerate} } 

\begin{enumerate}[<+->] 

\item  {\tt beamer} is not part of a standard LaTex installation.  Most will need to get the packages. 
\item  From any \href{http://www.ctan.org}{Comprehensive TeX Archive Network} mirror. 
\item  Or easier yet, from the \href{http://latex-beamer.sourceforge.net/}{Sourceforge} website.  
\item  To experiment with the files, put them in one of your local directories, and run pdflatex on 
this or another example file that comes with the distribution. 
\end{enumerate}

}


\frame{\frametitle{{\tt description} } 

\begin{description}[<+->] 

\item[ant]  {\tt beamer} is not part of a standard LaTex installation.  Most will need to get the packages. 
\item[bettle]  From any \href{http://www.ctan.org}{Comprehensive TeX Archive Network} mirror. 
\item[circada]  Or easier yet, from the \href{http://latex-beamer.sourceforge.net/}{Sourceforge} website.  
\item[termite]  To experiment with the files, put them in one of your local directories, and run pdflatex on 
this or another example file that comes with the distribution. 
\end{description}

}

\frame{\frametitle{{\tt itemize} } 

\begin{itemize}[<+->] 

\item[$\bullet$]  {\tt beamer} is not part of a standard LaTex installation.  Most will need to get the packages. 
\item[$\bullet$]  From any \href{http://www.ctan.org}{Comprehensive TeX Archive Network} mirror. 
\item[$\bullet$]  Or easier yet, from the \href{http://latex-beamer.sourceforge.net/}{Sourceforge} website.  
\item[$\bullet$]  To experiment with the files, put them in one of your local directories, and run pdflatex on 
this or another example file that comes with the distribution. 
\end{itemize}

}

\frame{\frametitle{Boxed Text}
There are three options for Boxed Text: (1) 
You can use LaTeX's {\tt fbox} command,  (2) the commands created by {\tt fancybox}. 
See the LaTeX Companion for more details.  (3) beamerboxes, see the beamer user guide.  These two examples 
use  beamerboxes. 



\begin{beamerboxesrounded}[shadow=true]{\tt beamerboxesrounded, with option shadow=true} 
$$
\int f(x-y)g(x+y)\,\frac{dy}y 
$$
\end{beamerboxesrounded}


\setbeamercolor{postit}{fg=black,bg=yellow}
\begin{center}
\begin{beamercolorbox}[sep=2em,wd=6cm] {postit} 
Some important point on a postit.
\end{beamercolorbox}
\end{center}
} 



\frame{\frametitle{Columns} 

\begin{columns}

		\begin{column}{7cm} 
		
		An important illustration goes here.
		\end{column}

		\begin{column}{8cm} 
		Typically some text 
		\\
		should go on the right 
		\end{column}

\end{columns} 
}

% \begin{columns}[options]  
%	\begin{column}[options]{width}
%		,,,,
%	\end{column}
%     %  Can have two or more columns 
%  \end{columns} 

%  options   for columns 
%		b= bottom of columns vertically aligned 
%		c= columns will be centered vertically, default in most circumstances 
%	 	t=  top line of columns vertically aligned
%		totalwidth={width dimension}  

%  options for column  govern vertical placement 
%`		b=bottom
%		t=top
%		c=center


\section{Graphics} 
%  
%   To use an image, you can use the command
%  \pgfimage[height=]{image_dir/ImageNameWithOutAnExtension}
%   Dont put the extension, as jpg, ps, and png are automatically searched for. 
%



\frame{\frametitle{Graphics} 


\begin{itemize}%<+->%| alert@+>
\item You'll probably want to include some graphics.  
\item  If you are familiar with the {\tt graphics} package, it works in {\tt beamer}.  
The basic command is {\tt includegraphics} 
\item  The graphics/drawing package {\tt pgf} is loaded automatically, and it's basic command is 
{\tt pgfuseimage} .
\item  Both of these commands are overlay aware!  
\end{itemize}
}







\def\zc#1 {\textcolor{#1}{#1}} 


\def\softness{0.4}
\definecolor{softred}{rgb}{1,\softness,\softness}
\definecolor{softgreen}{rgb}{\softness,1,\softness}
\definecolor{softblue}{rgb}{\softness,\softness,1}

\definecolor{softrg}{rgb}{1,1,\softness}
\definecolor{softrb}{rgb}{1,\softness,1}
\definecolor{softgb}{rgb}{\softness,1,1}



\frame{\frametitle{Colors} 

The LaTeX package {\tt color} and {\tt xcolor} are automatically loaded.  
Some colors are automatically defined: 
\zc red , 
\zc green ,
\zc blue , 
\zc cyan ,
\zc magenta ,
\zc yellow ,
\zc gray , 
\zc lightgray . 

\uncover<2-> { To go beyond this, you'll need to define some additional colors, and get a little more 
comfortable with the {\tt color} and {\tt xcolor } packages. 
Some examples: 
\zc softred ,
\zc softblue ,
\zc softgreen , 
\zc softrg , \zc softrb , \zc softgb .
}
}


\def\diags{ 
	\begin{pgfscope} \pgfsetlinewidth{1.2pt}
	\definecolorseries{diag}{hsb}{step}[hsb]{.575,1,1}{.11,-.05,0}
	\resetcolorseries{diag}
	  \color{diag!!+}
	 \pgfmoveto{\pgfxy(-3,-2)} \pgflineto{\pgfxy(-2,-3)} \pgfstroke
	 \color{diag!!+} 
	\pgfmoveto{\pgfxy(-3,-1)} \pgflineto{\pgfxy(-1,-3)} \pgfstroke
	 \color{diag!!+} 
	\pgfmoveto{\pgfxy(-3,0)} \pgflineto{\pgfxy(0,-3)} \pgfstroke
	  \color{diag!!+} 
	\pgfmoveto{\pgfxy(-3,1)} \pgflineto{\pgfxy(1,-3)} \pgfstroke
	\color{diag!!+} 
	\pgfmoveto{\pgfxy(-3,2)} \pgflineto{\pgfxy(2,-3)} \pgfstroke
	\color{diag!!+} 
	\pgfmoveto{\pgfxy(-3,3)} \pgflineto{\pgfxy(3,-3)} \pgfstroke
	\color{diag!!+}
	 \pgfmoveto{\pgfxy(-2,3)} \pgflineto{\pgfxy(3,-2)} \pgfstroke
	 \color{diag!!+} 
	\pgfmoveto{\pgfxy(-1,3)} \pgflineto{\pgfxy(3,-1)} \pgfstroke
	 \color{diag!!+} 
	\pgfmoveto{\pgfxy(0,3)} \pgflineto{\pgfxy(3,0)} \pgfstroke
	  \color{diag!!+} 
	\pgfmoveto{\pgfxy(1,3)} \pgflineto{\pgfxy(3,1)} \pgfstroke
	\color{diag!!+} 
	\pgfmoveto{\pgfxy(2,3)} \pgflineto{\pgfxy(3,2)} \pgfstroke
	\end{pgfscope}
	}






\frame{
\frametitle{ The {\tt pgf} package} 

\begin{columns}
  \begin{column}{2in} 
 The {\tt pgf } package has some nice features for drawing illustrations, all compatible with the {\tt uncover} and {\tt only} 
features of {\tt beamer}. 

  
  \end{column}
  
  \begin{column}{3in} 
  
  \begin{center}
  	\begin{pgfpicture}{0cm}{0cm}{6cm}{6cm}

	\begin{pgftranslate}{\pgfxy(3,3)} 	 
		 \pgfsetendarrow{\pgfarrowsingle} \pgfsetstartarrow{\pgfarrowsingle} 
		 %  coordinate axes
	

	\only<2>{ \color{lightgray}
	 \pgfmoveto{\pgfxy(-3,0)}\pgflineto{\pgfxy(3,0)}\pgfstroke
	 \pgfmoveto{\pgfxy(0,-3)}\pgflineto{\pgfxy(0,3)}\pgfstroke
	 
	 \diags   %  all the pretty diagonal 
	 
	 \color{black} 
	 \pgfputat{\pgfxy(-3.25,3.25)}{\pgfbox[center,center]{$\widehat \varphi(0)$}}
	 \pgfputat{\pgfxy(-2.0,3.25)}{\pgfbox[center,center]{$\widehat \varphi(1)$}}
	 \pgfputat{\pgfxy(-1.0,3.25)}{\pgfbox[center,center]{$\widehat \varphi(2)$}}
	 \pgfputat{\pgfxy(-3.35,2.0)}{\pgfbox[center,center]{$\widehat \varphi(-1)$}}
	 \pgfputat{\pgfxy(-3.35,1.0)}{\pgfbox[center,center]{$\widehat \varphi(-2)$}}	 
	 }
	 
	 \only<3>{	
				\color{black} 
				\pgfputat{\pgfxy(1.5,.3)}{\pgfbox[center,center]{$\hbox{Toeplitz}$}}
				\pgfmoveto{\pgfxy(0,-3)}\pgflineto{\pgfxy(0,0)}
					\pgflineto{\pgfxy(3,0)}\pgflineto{\pgfxy(3,-3)}
					\pgflineto{\pgfxy(0,-3)} \pgfclip
					\color{lightgray}
					\pgfmoveto{\pgfxy(-3,0)}\pgflineto{\pgfxy(3,0)}\pgfstroke
					\pgfmoveto{\pgfxy(0,-3)}\pgflineto{\pgfxy(0,3)}\pgfstroke
					\diags   %  all the pretty diagonal 
				}
				
		 \only<4>{
				\color{black} 
	\pgfputat{\pgfxy(-1.5,.3)}{\pgfbox[center,center]{$\hbox{Hankel}$}}
	
	\pgfsetendarrow{\pgfarrowsingle} \pgfsetstartarrow{\pgfarrowsingle} 
	\pgfmoveto{\pgfxy(-3,0)}\pgflineto{\pgfxy(0,0)}
					\pgflineto{\pgfxy(0,-3)}\pgflineto{\pgfxy(-3,-3)}
					\pgflineto{\pgfxy(-3,0)} 
		\pgfclip
	 \color{lightgray}
					\pgfmoveto{\pgfxy(-3,0)}\pgflineto{\pgfxy(3,0)}\pgfstroke
					\pgfmoveto{\pgfxy(0,-3)}\pgflineto{\pgfxy(0,3)}\pgfstroke
					\diags 
				}
	 
	 
	 
	 \end{pgftranslate}
	 \end{pgfpicture}
	 \end{center}
	 \end{column}
	 \end{columns}

	 }




\colorlet{redshaded}{red!25!averagebackgroundcolor}
\colorlet{shaded}{black!25!averagebackgroundcolor}
\colorlet{shadedshaded}{black!10!averagebackgroundcolor}
\colorlet{blackshaded}{black!40!averagebackgroundcolor}


\def\radius{0.96cm}
\def\innerradius{0.85cm}


\frame{\frametitle{Another {\tt pgf} example} 


See the code to see how this was done. 


 \begin{pgfpicture}{-2cm}{-1.75cm}{2cm}{2.25cm}
        \color{shaded}
        \pgfrect[fill]{\pgfxy(-2,-1.75)}{\pgfxy(4,4)}
                                %\pgfcircle[fill]{\pgforigin}{2cm}

        \only<1>{%
          \color{white}%
          \pgfcircle[fill]{\pgfpolar{90}{1cm}}{\innerradius}
          \pgfcircle[fill]{\pgfpolar{210}{1cm}}{\innerradius}
          \pgfcircle[fill]{\pgfpolar{330}{1cm}}{\innerradius}}%
        \only<2->{%      
        \color{softred}
        \pgfcircle[fill]{\pgfpolar{90}{1cm}}{\radius}
        \color{softgreen}
        \pgfcircle[fill]{\pgfpolar{210}{1cm}}{\radius}
        \color{softblue}
        \pgfcircle[fill]{\pgfpolar{330}{1cm}}{\radius}}%
        %
      \only<2->{%
        \begin{pgftranslate}{\pgfpolar{90}{1cm}}
          \pgfzerocircle{\radius}
          \pgfclip
          
          \begin{pgftranslate}{\pgfpolar{-90}{1cm}}
            \color{softrb}
            \pgfcircle[fill]{\pgfpolar{330}{1cm}}{\radius}
            \color{softrg}
            \pgfcircle[fill]{\pgfpolar{210}{1cm}}{\radius}
          \end{pgftranslate}
        \end{pgftranslate}

        \begin{pgftranslate}{\pgfpolar{210}{1cm}}
          \pgfzerocircle{\radius}
          \pgfclip
          
          \begin{pgftranslate}{\pgfpolar{30}{1cm}}
            \color{softgb}
            \pgfcircle[fill]{\pgfpolar{330}{1cm}}{\radius}
          \end{pgftranslate}
        \end{pgftranslate}}%
        %
        \color{black}
        \pgfcircle[stroke]{\pgfpolar{90}{1cm}}{\innerradius}
        \pgfcircle[stroke]{\pgfpolar{210}{1cm}}{\innerradius}
        \pgfcircle[stroke]{\pgfpolar{330}{1cm}}{\innerradius}

        \pgfputat{\pgfrelative{\pgfpolar{90}{1cm}}%
          {\pgfpoint{0pt}{-.5ex}}}%
        {\pgfbox[center,base]{$A\times B$}}
        \pgfputat{\pgfrelative{\pgfpolar{210}{1cm}}%
          {\pgfpoint{0pt}{-.5ex}}}%
        {\pgfbox[center,base]{$A\times A$}}
        \pgfputat{\pgfrelative{\pgfpolar{330}{1cm}}%
          {\pgfpoint{0pt}{-.5ex}}}%
        {\pgfbox[center,base]{$B\times B$}}

      \end{pgfpicture}
}









\end{document}
