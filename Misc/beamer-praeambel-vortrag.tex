\documentclass[12pt]{beamer}
\usepackage[T1]{fontenc}
%\usepackage[latin1,utf8]{inputenc}
\usepackage[utf8]{inputenc}
\usepackage[ngerman]{babel}
\usepackage[babel,german=guillemets]{csquotes}
%\usepackage[bgadd]{background}
%\usepackage[strings]{underscore}
\usepackage{lmodern,
            url,
            multicol,
            moreverb,
            verbatim,
            lipsum,
            multimedia,
            textpos,
%            pdfpages,
            wrapfig,
%            helvet,
            eso-pic,
            marvosym,
            listings,
            pifont,
            calc,
            graphicx,
            tikz,
            booktabs,
	    xspace,
            textcomp}
            
\useinnertheme{rectangles}
\hypersetup{pdfpagemode=FullScreen} % Fullscreen beim Start mit Adobe Reader
\lstset{backgroundcolor=\color{lightgray},
        basicstyle=\ttfamily\color{uniblue},
        frame=single,
        framerule=0pt,
        xrightmargin=\fboxsep,
        xleftmargin=\fboxsep,
        basewidth=.5em,
        extendedchars=true} % Defaultwerte beim Einbinden von Quellcode

%------------------------------
%
% Schrift\"anderungen
%
%------------------------------
\def\UrlFont{\color{uniblue}\tt} % URL-Font auf Typewriter
\usefonttheme[onlymath]{serif}   % Mathetext in Serifen
\renewcommand{\sfdefault}{pr4}   % Rotis Sans Serif
\renewcommand{\rmdefault}{pro}   % Rotis Serif

%------------------------------
%
% L\"angen
%
%------------------------------
\setlength{\parindent}{0pt}   % Kein Absatzeinzug
\setlength{\parskip}%
  {1.2ex plus .5ex minus.5ex} % Individueller Abstand zw. Abs\"atzen
\setlength{\unitlength}{1mm}  % Default f\"ur Einheitsl\"angen: 1mm

%------------------------------
%
% Neue/Alte Kommandos
%
%------------------------------
% \tex             -> Abk\"urzung f\"ur \TeX
% \latex           -> Abk\"urzung f\"ur \LaTeX
% \ttalert[1]      -> #1 in Typewriter hervorheben
% \B               -> Abk\"urzung f\"ur \textbullet
% \T               -> Abk\"urzung f\"ur \textbackslash
% \neueseite[2]    -> N\"achste Foliem mit #1 als Titel
%                     und #2 als Inhalt
% \lightitem[1]    -> Hellgraues \item der itemize-Umgebung.
%                     #1 ist das Item.
% \Email[1]        -> #1 ist die E-Mail-Adresse
% \Zeitraum[1]     -> #1 ist der Zeitraum des Kurses
% \beamertitlepage -> Erstellung der Titelseite an dieser
%                     Stelle
% \todo            -> Hilfskommando bei der Erstellung von
%                     Vortr\"agen
\newcommand{\tex}{\textrm{\TeX}}

\newcommand{\latex}{\textrm{\LaTeX}}

\newcommand{\ttalert}[1]{\alert{\texttt{#1}}}

\newcommand{\B}{\textbullet}

\newcommand{\T}{\textbackslash}

\newcommand{\neueseite}[3][t]{%
\begin{frame}[#1]{#2}
#3
\end{frame}
}

\newcommand{\tbf}[1]{\textbf{#1}}
\newcommand{\ttt}[1]{\texttt{#1}}

\newcommand{\lightitem}[1]{\item%
[\color{gray}\Squaresteel]%
\textcolor{gray}{#1}%
}

\renewcommand*{\Email}[1]{\def\myEmail{#1}}

\newcommand*{\Zeitraum}[1]{\def\myZeitraum{#1}}

\newcommand{\beamertitlepage}{%
\begin{frame}
\AddToShipoutPicture*{%
  \unitlength 1mm
  \begin{picture}(128,96)
    \fboxsep 0pt
    \put(0,5){\pgfuseimage{CPU}}
  \end{picture}}
  \centering
  \bfseries
  \textcolor{uniblue}{\Huge\inserttitle}
  \color{gray}
  \vskip 5mm
  \insertauthor\par
  \myEmail
  \vskip 5mm
  \myZeitraum\par
\end{frame}}

\newcommand{\todo}{\par\color{red}\Huge\textbf{ToDo!}\par}

% Setze ein Zeichen echts unten in die Ecke:
\newcommand{\corner}[1]{%
\begin{textblock*}{3mm}(123mm,88mm)
\resizebox{3mm}{!}{#1}%
\end{textblock*}%
}

%------------------------------
%
% Farben
%
%------------------------------
\definecolor{unigreen}{cmyk}{0.30,0,0.94,0}
\definecolor{uniblue}{cmyk}{0.80,0.55,0,0}
\definecolor{unigray}{cmyk}{0.40,0.31,0.29,0}
\definecolor{mygray}{gray}{.9}
%\definecolor{unigrey}{rgb}{0.667,0.667,0.667}
%\definecolor{uniblue}{rgb}{0.227,0.427,0.686}
%\definecolor{unigreen}{rgb}{0.784,0.827,0.09}
%\definecolor{unigreen}{cmyk}{1,0,0,0} % Pantone 382
%\definecolor{uniblue}{cmyk}{0,1,0,0}  % Pantone 285
%\definecolor{unigray}{cmyk}{0,0,1,0}  % Pantone Cool Grey 6
\definecolor{lightgreen}{rgb}{.467,.621,.176}
\definecolor{yellow}{rgb}{1,1,0}
\definecolor{darkyellow}{rgb}{.6,.6,0}
\definecolor{lightgray}{rgb}{0.937,0.937,0.937}

%------------------------------
%
% Eigene Boxen
%
%------------------------------
% \cbox[2]           -> #1 Breite (default: Textbreite),
%                       #2 Inhalt, hellgrau hinterlegte
%                       Box mit blauer Schrift.
% \ttbox[2]          -> wie \cbox nur mit Typewriter-Schrift
% \CBOX[2]           -> wie \ttbox, allerdings mit Quellcode
%                       (#2), #1 sind Optionen f\"ur das
%                       listings-Paket
% \topalignbox[1]    -> Box auf Headline statt Baseline
% \centeralignbox[1] -> Box vertikal zentriert statt auf Baseline
\newcommand{\cbox}[2][\linewidth]{%
\colorbox{lightgray}{%
\parbox{#1-2\fboxsep}{%
\color{uniblue}%
#2}}}

\newcommand{\ttbox}[2][\linewidth]{%
\colorbox{lightgray}{%
\parbox[c]{#1-2\fboxsep}{%
\color{uniblue}%
\ttfamily{#2}}}}

\newcommand{\CBOX}[2][basicstyle=\ttfamily]{%
\mbox{\lstinputlisting[#1]{#2}}%
}

\newlength\topgraphics

\newcommand{\topalignbox}[1]{%
  \setbox0=\hbox{#1}%
  \setlength{\topgraphics}{-\ht0}%
  \setbox0=\hbox{Y}%
  \addtolength{\topgraphics}{\ht0}%
  \raisebox{\topgraphics}{#1}%
}

\newlength\centergraphics

\newcommand{\centeralignbox}[1]{%
  \setbox0=\hbox{#1}%
  \setlength{\centergraphics}{-.5\ht0}%
  \setbox0=\hbox{Y}%
  \addtolength{\centergraphics}{\ht0}%
  \raisebox{\centergraphics}{#1}%
}

%------------------------------
%
% Imagedefinitionen f\"ur PGF
%
%------------------------------
\pgfdeclareimage[height=5mm]{RRZN}{./Abbildungen/rrzn_logo_tmp}
\pgfdeclareimage[height=5mm]{UNI}{./Abbildungen/luh_logo_cmyk_neu} % 17.5mm breit
\pgfdeclareimage[width=64mm]{CPU}{./Abbildungen/cpu}

%------------------------------
%
% Beamertemplate-\"Anderungen
%
%------------------------------
% Kopfzeile:
\setbeamertemplate{headline}{%
\vspace{3mm}\hspace{3mm}%
\pgfuseimage{RRZN}\hfill\pgfuseimage{UNI}%
\hspace{3mm}\mbox{}%
}

% Fu{\ss}zeile:
\setbeamertemplate{footline}{%
\begin{beamercolorbox}[ht=1mm]{colorbox}
\color{unigreen}%
\hspace{107.5mm}\rule{17.5mm}{1mm}
\end{beamercolorbox}
\begin{beamercolorbox}[leftskip=3mm,rightskip=3mm,ht=4mm]{colorbox}
\textbf{\insertauthor}, \inserttitle{}, \insertdate\hfill{}Seite \insertframenumber/\inserttotalframenumber%
\vspace{1.5mm}\vbox{}%
\end{beamercolorbox}
}
% Folientitel:
\setbeamertemplate{frametitle}{%
  \begin{beamercolorbox}%
        [wd=\paperwidth,leftskip=5mm,%
         rightskip=5mm,vmode]{frametitle}%
     \usebeamerfont*{frametitle}%
     \textbf{\insertframetitle}%
   \end{beamercolorbox}}
% Kleinkram ...:
\setbeamertemplate{navigation symbols}{}
%\setbeamertemplate{blocks}[rounded][shadow=true]
\setbeamertemplate{enumerate items}[square]
\setbeamertemplate{itemize items}{%
  \raisebox{1.5pt}{\scriptsize\vrule width1ex height1ex}}
\setbeamertemplate{itemize subitem}{%
  \raisebox{1.5pt}{\scriptsize\fboxsep0pt\fboxrule1pt\fbox{%
  \color{white}\rule{1ex-2pt}{1ex-2pt}}}}

%------------------------------
%
% Beamerfarben
%
%------------------------------
\setbeamercolor{colorbox}{fg=uniblue,bg=lightgray}
\setbeamercolor{frametitle}{fg=uniblue}
\setbeamercolor{title}{fg=uniblue}
\setbeamercolor{item}{fg=unigreen}
\setbeamercolor{subitem}{fg=unigreen}
\setbeamercolor{description item}{fg=uniblue}
\setbeamercolor{background canvas}{bg=}
\setbeamercolor{alerted text}{fg=uniblue}
\setbeamercolor{block body}{bg=lightgray}
\setbeamercolor{block title}{bg=unigray}

%------------------------------
%
% Beamergr\"o{\ss}en
%
%------------------------------
\setbeamersize{description width=5mm}
\setbeamersize{text margin left=5mm}
\setbeamersize{text margin right=5mm}

\begin{document}
\Zeitraum{}
% Beispieldokument:
%\documentclass[12pt]{beamer}
\usepackage[T1]{fontenc}
%\usepackage[latin1,utf8]{inputenc}
\usepackage[utf8]{inputenc}
\usepackage[ngerman]{babel}
\usepackage[babel,german=guillemets]{csquotes}
%\usepackage[bgadd]{background}
%\usepackage[strings]{underscore}
\usepackage{lmodern,
            url,
            multicol,
            moreverb,
            verbatim,
            lipsum,
            multimedia,
            textpos,
%            pdfpages,
            wrapfig,
%            helvet,
            eso-pic,
            marvosym,
            listings,
            pifont,
            calc,
            graphicx,
            tikz,
            booktabs,
	    xspace,
            textcomp}
            
\useinnertheme{rectangles}
\hypersetup{pdfpagemode=FullScreen} % Fullscreen beim Start mit Adobe Reader
\lstset{backgroundcolor=\color{lightgray},
        basicstyle=\ttfamily\color{uniblue},
        frame=single,
        framerule=0pt,
        xrightmargin=\fboxsep,
        xleftmargin=\fboxsep,
        basewidth=.5em,
        extendedchars=true} % Defaultwerte beim Einbinden von Quellcode

%------------------------------
%
% Schrift\"anderungen
%
%------------------------------
\def\UrlFont{\color{uniblue}\tt} % URL-Font auf Typewriter
\usefonttheme[onlymath]{serif}   % Mathetext in Serifen
\renewcommand{\sfdefault}{pr4}   % Rotis Sans Serif
\renewcommand{\rmdefault}{pro}   % Rotis Serif

%------------------------------
%
% L\"angen
%
%------------------------------
\setlength{\parindent}{0pt}   % Kein Absatzeinzug
\setlength{\parskip}%
  {1.2ex plus .5ex minus.5ex} % Individueller Abstand zw. Abs\"atzen
\setlength{\unitlength}{1mm}  % Default f\"ur Einheitsl\"angen: 1mm

%------------------------------
%
% Neue/Alte Kommandos
%
%------------------------------
% \tex             -> Abk\"urzung f\"ur \TeX
% \latex           -> Abk\"urzung f\"ur \LaTeX
% \ttalert[1]      -> #1 in Typewriter hervorheben
% \B               -> Abk\"urzung f\"ur \textbullet
% \T               -> Abk\"urzung f\"ur \textbackslash
% \neueseite[2]    -> N\"achste Foliem mit #1 als Titel
%                     und #2 als Inhalt
% \lightitem[1]    -> Hellgraues \item der itemize-Umgebung.
%                     #1 ist das Item.
% \Email[1]        -> #1 ist die E-Mail-Adresse
% \Zeitraum[1]     -> #1 ist der Zeitraum des Kurses
% \beamertitlepage -> Erstellung der Titelseite an dieser
%                     Stelle
% \todo            -> Hilfskommando bei der Erstellung von
%                     Vortr\"agen
\newcommand{\tex}{\textrm{\TeX}}

\newcommand{\latex}{\textrm{\LaTeX}}

\newcommand{\ttalert}[1]{\alert{\texttt{#1}}}

\newcommand{\B}{\textbullet}

\newcommand{\T}{\textbackslash}

\newcommand{\neueseite}[3][t]{%
\begin{frame}[#1]{#2}
#3
\end{frame}
}

\newcommand{\tbf}[1]{\textbf{#1}}
\newcommand{\ttt}[1]{\texttt{#1}}

\newcommand{\lightitem}[1]{\item%
[\color{gray}\Squaresteel]%
\textcolor{gray}{#1}%
}

\renewcommand*{\Email}[1]{\def\myEmail{#1}}

\newcommand*{\Zeitraum}[1]{\def\myZeitraum{#1}}

\newcommand{\beamertitlepage}{%
\begin{frame}
\AddToShipoutPicture*{%
  \unitlength 1mm
  \begin{picture}(128,96)
    \fboxsep 0pt
    \put(0,5){\pgfuseimage{CPU}}
  \end{picture}}
  \centering
  \bfseries
  \textcolor{uniblue}{\Huge\inserttitle}
  \color{gray}
  \vskip 5mm
  \insertauthor\par
  \myEmail
  \vskip 5mm
  \myZeitraum\par
\end{frame}}

\newcommand{\todo}{\par\color{red}\Huge\textbf{ToDo!}\par}

% Setze ein Zeichen echts unten in die Ecke:
\newcommand{\corner}[1]{%
\begin{textblock*}{3mm}(123mm,88mm)
\resizebox{3mm}{!}{#1}%
\end{textblock*}%
}

%------------------------------
%
% Farben
%
%------------------------------
\definecolor{unigreen}{cmyk}{0.30,0,0.94,0}
\definecolor{uniblue}{cmyk}{0.80,0.55,0,0}
\definecolor{unigray}{cmyk}{0.40,0.31,0.29,0}
\definecolor{mygray}{gray}{.9}
%\definecolor{unigrey}{rgb}{0.667,0.667,0.667}
%\definecolor{uniblue}{rgb}{0.227,0.427,0.686}
%\definecolor{unigreen}{rgb}{0.784,0.827,0.09}
%\definecolor{unigreen}{cmyk}{1,0,0,0} % Pantone 382
%\definecolor{uniblue}{cmyk}{0,1,0,0}  % Pantone 285
%\definecolor{unigray}{cmyk}{0,0,1,0}  % Pantone Cool Grey 6
\definecolor{lightgreen}{rgb}{.467,.621,.176}
\definecolor{yellow}{rgb}{1,1,0}
\definecolor{darkyellow}{rgb}{.6,.6,0}
\definecolor{lightgray}{rgb}{0.937,0.937,0.937}

%------------------------------
%
% Eigene Boxen
%
%------------------------------
% \cbox[2]           -> #1 Breite (default: Textbreite),
%                       #2 Inhalt, hellgrau hinterlegte
%                       Box mit blauer Schrift.
% \ttbox[2]          -> wie \cbox nur mit Typewriter-Schrift
% \CBOX[2]           -> wie \ttbox, allerdings mit Quellcode
%                       (#2), #1 sind Optionen f\"ur das
%                       listings-Paket
% \topalignbox[1]    -> Box auf Headline statt Baseline
% \centeralignbox[1] -> Box vertikal zentriert statt auf Baseline
\newcommand{\cbox}[2][\linewidth]{%
\colorbox{lightgray}{%
\parbox{#1-2\fboxsep}{%
\color{uniblue}%
#2}}}

\newcommand{\ttbox}[2][\linewidth]{%
\colorbox{lightgray}{%
\parbox[c]{#1-2\fboxsep}{%
\color{uniblue}%
\ttfamily{#2}}}}

\newcommand{\CBOX}[2][basicstyle=\ttfamily]{%
\mbox{\lstinputlisting[#1]{#2}}%
}

\newlength\topgraphics

\newcommand{\topalignbox}[1]{%
  \setbox0=\hbox{#1}%
  \setlength{\topgraphics}{-\ht0}%
  \setbox0=\hbox{Y}%
  \addtolength{\topgraphics}{\ht0}%
  \raisebox{\topgraphics}{#1}%
}

\newlength\centergraphics

\newcommand{\centeralignbox}[1]{%
  \setbox0=\hbox{#1}%
  \setlength{\centergraphics}{-.5\ht0}%
  \setbox0=\hbox{Y}%
  \addtolength{\centergraphics}{\ht0}%
  \raisebox{\centergraphics}{#1}%
}

%------------------------------
%
% Imagedefinitionen f\"ur PGF
%
%------------------------------
\pgfdeclareimage[height=5mm]{RRZN}{./Abbildungen/rrzn_logo_tmp}
\pgfdeclareimage[height=5mm]{UNI}{./Abbildungen/luh_logo_cmyk_neu} % 17.5mm breit
\pgfdeclareimage[width=64mm]{CPU}{./Abbildungen/cpu}

%------------------------------
%
% Beamertemplate-\"Anderungen
%
%------------------------------
% Kopfzeile:
\setbeamertemplate{headline}{%
\vspace{3mm}\hspace{3mm}%
\pgfuseimage{RRZN}\hfill\pgfuseimage{UNI}%
\hspace{3mm}\mbox{}%
}

% Fu{\ss}zeile:
\setbeamertemplate{footline}{%
\begin{beamercolorbox}[ht=1mm]{colorbox}
\color{unigreen}%
\hspace{107.5mm}\rule{17.5mm}{1mm}
\end{beamercolorbox}
\begin{beamercolorbox}[leftskip=3mm,rightskip=3mm,ht=4mm]{colorbox}
\textbf{\insertauthor}, \inserttitle{}, \insertdate\hfill{}Seite \insertframenumber/\inserttotalframenumber%
\vspace{1.5mm}\vbox{}%
\end{beamercolorbox}
}
% Folientitel:
\setbeamertemplate{frametitle}{%
  \begin{beamercolorbox}%
        [wd=\paperwidth,leftskip=5mm,%
         rightskip=5mm,vmode]{frametitle}%
     \usebeamerfont*{frametitle}%
     \textbf{\insertframetitle}%
   \end{beamercolorbox}}
% Kleinkram ...:
\setbeamertemplate{navigation symbols}{}
%\setbeamertemplate{blocks}[rounded][shadow=true]
\setbeamertemplate{enumerate items}[square]
\setbeamertemplate{itemize items}{%
  \raisebox{1.5pt}{\scriptsize\vrule width1ex height1ex}}
\setbeamertemplate{itemize subitem}{%
  \raisebox{1.5pt}{\scriptsize\fboxsep0pt\fboxrule1pt\fbox{%
  \color{white}\rule{1ex-2pt}{1ex-2pt}}}}

%------------------------------
%
% Beamerfarben
%
%------------------------------
\setbeamercolor{colorbox}{fg=uniblue,bg=lightgray}
\setbeamercolor{frametitle}{fg=uniblue}
\setbeamercolor{title}{fg=uniblue}
\setbeamercolor{item}{fg=unigreen}
\setbeamercolor{subitem}{fg=unigreen}
\setbeamercolor{description item}{fg=uniblue}
\setbeamercolor{background canvas}{bg=}
\setbeamercolor{alerted text}{fg=uniblue}
\setbeamercolor{block body}{bg=lightgray}
\setbeamercolor{block title}{bg=unigray}

%------------------------------
%
% Beamergr\"o{\ss}en
%
%------------------------------
\setbeamersize{description width=5mm}
\setbeamersize{text margin left=5mm}
\setbeamersize{text margin right=5mm}

\begin{document}
\Zeitraum{}
% Beispieldokument:
%\documentclass[12pt]{beamer}
\usepackage[T1]{fontenc}
%\usepackage[latin1,utf8]{inputenc}
\usepackage[utf8]{inputenc}
\usepackage[ngerman]{babel}
\usepackage[babel,german=guillemets]{csquotes}
%\usepackage[bgadd]{background}
%\usepackage[strings]{underscore}
\usepackage{lmodern,
            url,
            multicol,
            moreverb,
            verbatim,
            lipsum,
            multimedia,
            textpos,
%            pdfpages,
            wrapfig,
%            helvet,
            eso-pic,
            marvosym,
            listings,
            pifont,
            calc,
            graphicx,
            tikz,
            booktabs,
	    xspace,
            textcomp}
            
\useinnertheme{rectangles}
\hypersetup{pdfpagemode=FullScreen} % Fullscreen beim Start mit Adobe Reader
\lstset{backgroundcolor=\color{lightgray},
        basicstyle=\ttfamily\color{uniblue},
        frame=single,
        framerule=0pt,
        xrightmargin=\fboxsep,
        xleftmargin=\fboxsep,
        basewidth=.5em,
        extendedchars=true} % Defaultwerte beim Einbinden von Quellcode

%------------------------------
%
% Schrift\"anderungen
%
%------------------------------
\def\UrlFont{\color{uniblue}\tt} % URL-Font auf Typewriter
\usefonttheme[onlymath]{serif}   % Mathetext in Serifen
\renewcommand{\sfdefault}{pr4}   % Rotis Sans Serif
\renewcommand{\rmdefault}{pro}   % Rotis Serif

%------------------------------
%
% L\"angen
%
%------------------------------
\setlength{\parindent}{0pt}   % Kein Absatzeinzug
\setlength{\parskip}%
  {1.2ex plus .5ex minus.5ex} % Individueller Abstand zw. Abs\"atzen
\setlength{\unitlength}{1mm}  % Default f\"ur Einheitsl\"angen: 1mm

%------------------------------
%
% Neue/Alte Kommandos
%
%------------------------------
% \tex             -> Abk\"urzung f\"ur \TeX
% \latex           -> Abk\"urzung f\"ur \LaTeX
% \ttalert[1]      -> #1 in Typewriter hervorheben
% \B               -> Abk\"urzung f\"ur \textbullet
% \T               -> Abk\"urzung f\"ur \textbackslash
% \neueseite[2]    -> N\"achste Foliem mit #1 als Titel
%                     und #2 als Inhalt
% \lightitem[1]    -> Hellgraues \item der itemize-Umgebung.
%                     #1 ist das Item.
% \Email[1]        -> #1 ist die E-Mail-Adresse
% \Zeitraum[1]     -> #1 ist der Zeitraum des Kurses
% \beamertitlepage -> Erstellung der Titelseite an dieser
%                     Stelle
% \todo            -> Hilfskommando bei der Erstellung von
%                     Vortr\"agen
\newcommand{\tex}{\textrm{\TeX}}

\newcommand{\latex}{\textrm{\LaTeX}}

\newcommand{\ttalert}[1]{\alert{\texttt{#1}}}

\newcommand{\B}{\textbullet}

\newcommand{\T}{\textbackslash}

\newcommand{\neueseite}[3][t]{%
\begin{frame}[#1]{#2}
#3
\end{frame}
}

\newcommand{\tbf}[1]{\textbf{#1}}
\newcommand{\ttt}[1]{\texttt{#1}}

\newcommand{\lightitem}[1]{\item%
[\color{gray}\Squaresteel]%
\textcolor{gray}{#1}%
}

\renewcommand*{\Email}[1]{\def\myEmail{#1}}

\newcommand*{\Zeitraum}[1]{\def\myZeitraum{#1}}

\newcommand{\beamertitlepage}{%
\begin{frame}
\AddToShipoutPicture*{%
  \unitlength 1mm
  \begin{picture}(128,96)
    \fboxsep 0pt
    \put(0,5){\pgfuseimage{CPU}}
  \end{picture}}
  \centering
  \bfseries
  \textcolor{uniblue}{\Huge\inserttitle}
  \color{gray}
  \vskip 5mm
  \insertauthor\par
  \myEmail
  \vskip 5mm
  \myZeitraum\par
\end{frame}}

\newcommand{\todo}{\par\color{red}\Huge\textbf{ToDo!}\par}

% Setze ein Zeichen echts unten in die Ecke:
\newcommand{\corner}[1]{%
\begin{textblock*}{3mm}(123mm,88mm)
\resizebox{3mm}{!}{#1}%
\end{textblock*}%
}

%------------------------------
%
% Farben
%
%------------------------------
\definecolor{unigreen}{cmyk}{0.30,0,0.94,0}
\definecolor{uniblue}{cmyk}{0.80,0.55,0,0}
\definecolor{unigray}{cmyk}{0.40,0.31,0.29,0}
\definecolor{mygray}{gray}{.9}
%\definecolor{unigrey}{rgb}{0.667,0.667,0.667}
%\definecolor{uniblue}{rgb}{0.227,0.427,0.686}
%\definecolor{unigreen}{rgb}{0.784,0.827,0.09}
%\definecolor{unigreen}{cmyk}{1,0,0,0} % Pantone 382
%\definecolor{uniblue}{cmyk}{0,1,0,0}  % Pantone 285
%\definecolor{unigray}{cmyk}{0,0,1,0}  % Pantone Cool Grey 6
\definecolor{lightgreen}{rgb}{.467,.621,.176}
\definecolor{yellow}{rgb}{1,1,0}
\definecolor{darkyellow}{rgb}{.6,.6,0}
\definecolor{lightgray}{rgb}{0.937,0.937,0.937}

%------------------------------
%
% Eigene Boxen
%
%------------------------------
% \cbox[2]           -> #1 Breite (default: Textbreite),
%                       #2 Inhalt, hellgrau hinterlegte
%                       Box mit blauer Schrift.
% \ttbox[2]          -> wie \cbox nur mit Typewriter-Schrift
% \CBOX[2]           -> wie \ttbox, allerdings mit Quellcode
%                       (#2), #1 sind Optionen f\"ur das
%                       listings-Paket
% \topalignbox[1]    -> Box auf Headline statt Baseline
% \centeralignbox[1] -> Box vertikal zentriert statt auf Baseline
\newcommand{\cbox}[2][\linewidth]{%
\colorbox{lightgray}{%
\parbox{#1-2\fboxsep}{%
\color{uniblue}%
#2}}}

\newcommand{\ttbox}[2][\linewidth]{%
\colorbox{lightgray}{%
\parbox[c]{#1-2\fboxsep}{%
\color{uniblue}%
\ttfamily{#2}}}}

\newcommand{\CBOX}[2][basicstyle=\ttfamily]{%
\mbox{\lstinputlisting[#1]{#2}}%
}

\newlength\topgraphics

\newcommand{\topalignbox}[1]{%
  \setbox0=\hbox{#1}%
  \setlength{\topgraphics}{-\ht0}%
  \setbox0=\hbox{Y}%
  \addtolength{\topgraphics}{\ht0}%
  \raisebox{\topgraphics}{#1}%
}

\newlength\centergraphics

\newcommand{\centeralignbox}[1]{%
  \setbox0=\hbox{#1}%
  \setlength{\centergraphics}{-.5\ht0}%
  \setbox0=\hbox{Y}%
  \addtolength{\centergraphics}{\ht0}%
  \raisebox{\centergraphics}{#1}%
}

%------------------------------
%
% Imagedefinitionen f\"ur PGF
%
%------------------------------
\pgfdeclareimage[height=5mm]{RRZN}{./Abbildungen/rrzn_logo_tmp}
\pgfdeclareimage[height=5mm]{UNI}{./Abbildungen/luh_logo_cmyk_neu} % 17.5mm breit
\pgfdeclareimage[width=64mm]{CPU}{./Abbildungen/cpu}

%------------------------------
%
% Beamertemplate-\"Anderungen
%
%------------------------------
% Kopfzeile:
\setbeamertemplate{headline}{%
\vspace{3mm}\hspace{3mm}%
\pgfuseimage{RRZN}\hfill\pgfuseimage{UNI}%
\hspace{3mm}\mbox{}%
}

% Fu{\ss}zeile:
\setbeamertemplate{footline}{%
\begin{beamercolorbox}[ht=1mm]{colorbox}
\color{unigreen}%
\hspace{107.5mm}\rule{17.5mm}{1mm}
\end{beamercolorbox}
\begin{beamercolorbox}[leftskip=3mm,rightskip=3mm,ht=4mm]{colorbox}
\textbf{\insertauthor}, \inserttitle{}, \insertdate\hfill{}Seite \insertframenumber/\inserttotalframenumber%
\vspace{1.5mm}\vbox{}%
\end{beamercolorbox}
}
% Folientitel:
\setbeamertemplate{frametitle}{%
  \begin{beamercolorbox}%
        [wd=\paperwidth,leftskip=5mm,%
         rightskip=5mm,vmode]{frametitle}%
     \usebeamerfont*{frametitle}%
     \textbf{\insertframetitle}%
   \end{beamercolorbox}}
% Kleinkram ...:
\setbeamertemplate{navigation symbols}{}
%\setbeamertemplate{blocks}[rounded][shadow=true]
\setbeamertemplate{enumerate items}[square]
\setbeamertemplate{itemize items}{%
  \raisebox{1.5pt}{\scriptsize\vrule width1ex height1ex}}
\setbeamertemplate{itemize subitem}{%
  \raisebox{1.5pt}{\scriptsize\fboxsep0pt\fboxrule1pt\fbox{%
  \color{white}\rule{1ex-2pt}{1ex-2pt}}}}

%------------------------------
%
% Beamerfarben
%
%------------------------------
\setbeamercolor{colorbox}{fg=uniblue,bg=lightgray}
\setbeamercolor{frametitle}{fg=uniblue}
\setbeamercolor{title}{fg=uniblue}
\setbeamercolor{item}{fg=unigreen}
\setbeamercolor{subitem}{fg=unigreen}
\setbeamercolor{description item}{fg=uniblue}
\setbeamercolor{background canvas}{bg=}
\setbeamercolor{alerted text}{fg=uniblue}
\setbeamercolor{block body}{bg=lightgray}
\setbeamercolor{block title}{bg=unigray}

%------------------------------
%
% Beamergr\"o{\ss}en
%
%------------------------------
\setbeamersize{description width=5mm}
\setbeamersize{text margin left=5mm}
\setbeamersize{text margin right=5mm}

\begin{document}
\Zeitraum{}
% Beispieldokument:
%\documentclass[12pt]{beamer}
\usepackage[T1]{fontenc}
%\usepackage[latin1,utf8]{inputenc}
\usepackage[utf8]{inputenc}
\usepackage[ngerman]{babel}
\usepackage[babel,german=guillemets]{csquotes}
%\usepackage[bgadd]{background}
%\usepackage[strings]{underscore}
\usepackage{lmodern,
            url,
            multicol,
            moreverb,
            verbatim,
            lipsum,
            multimedia,
            textpos,
%            pdfpages,
            wrapfig,
%            helvet,
            eso-pic,
            marvosym,
            listings,
            pifont,
            calc,
            graphicx,
            tikz,
            booktabs,
	    xspace,
            textcomp}
            
\useinnertheme{rectangles}
\hypersetup{pdfpagemode=FullScreen} % Fullscreen beim Start mit Adobe Reader
\lstset{backgroundcolor=\color{lightgray},
        basicstyle=\ttfamily\color{uniblue},
        frame=single,
        framerule=0pt,
        xrightmargin=\fboxsep,
        xleftmargin=\fboxsep,
        basewidth=.5em,
        extendedchars=true} % Defaultwerte beim Einbinden von Quellcode

%------------------------------
%
% Schrift\"anderungen
%
%------------------------------
\def\UrlFont{\color{uniblue}\tt} % URL-Font auf Typewriter
\usefonttheme[onlymath]{serif}   % Mathetext in Serifen
\renewcommand{\sfdefault}{pr4}   % Rotis Sans Serif
\renewcommand{\rmdefault}{pro}   % Rotis Serif

%------------------------------
%
% L\"angen
%
%------------------------------
\setlength{\parindent}{0pt}   % Kein Absatzeinzug
\setlength{\parskip}%
  {1.2ex plus .5ex minus.5ex} % Individueller Abstand zw. Abs\"atzen
\setlength{\unitlength}{1mm}  % Default f\"ur Einheitsl\"angen: 1mm

%------------------------------
%
% Neue/Alte Kommandos
%
%------------------------------
% \tex             -> Abk\"urzung f\"ur \TeX
% \latex           -> Abk\"urzung f\"ur \LaTeX
% \ttalert[1]      -> #1 in Typewriter hervorheben
% \B               -> Abk\"urzung f\"ur \textbullet
% \T               -> Abk\"urzung f\"ur \textbackslash
% \neueseite[2]    -> N\"achste Foliem mit #1 als Titel
%                     und #2 als Inhalt
% \lightitem[1]    -> Hellgraues \item der itemize-Umgebung.
%                     #1 ist das Item.
% \Email[1]        -> #1 ist die E-Mail-Adresse
% \Zeitraum[1]     -> #1 ist der Zeitraum des Kurses
% \beamertitlepage -> Erstellung der Titelseite an dieser
%                     Stelle
% \todo            -> Hilfskommando bei der Erstellung von
%                     Vortr\"agen
\newcommand{\tex}{\textrm{\TeX}}

\newcommand{\latex}{\textrm{\LaTeX}}

\newcommand{\ttalert}[1]{\alert{\texttt{#1}}}

\newcommand{\B}{\textbullet}

\newcommand{\T}{\textbackslash}

\newcommand{\neueseite}[3][t]{%
\begin{frame}[#1]{#2}
#3
\end{frame}
}

\newcommand{\tbf}[1]{\textbf{#1}}
\newcommand{\ttt}[1]{\texttt{#1}}

\newcommand{\lightitem}[1]{\item%
[\color{gray}\Squaresteel]%
\textcolor{gray}{#1}%
}

\renewcommand*{\Email}[1]{\def\myEmail{#1}}

\newcommand*{\Zeitraum}[1]{\def\myZeitraum{#1}}

\newcommand{\beamertitlepage}{%
\begin{frame}
\AddToShipoutPicture*{%
  \unitlength 1mm
  \begin{picture}(128,96)
    \fboxsep 0pt
    \put(0,5){\pgfuseimage{CPU}}
  \end{picture}}
  \centering
  \bfseries
  \textcolor{uniblue}{\Huge\inserttitle}
  \color{gray}
  \vskip 5mm
  \insertauthor\par
  \myEmail
  \vskip 5mm
  \myZeitraum\par
\end{frame}}

\newcommand{\todo}{\par\color{red}\Huge\textbf{ToDo!}\par}

% Setze ein Zeichen echts unten in die Ecke:
\newcommand{\corner}[1]{%
\begin{textblock*}{3mm}(123mm,88mm)
\resizebox{3mm}{!}{#1}%
\end{textblock*}%
}

%------------------------------
%
% Farben
%
%------------------------------
\definecolor{unigreen}{cmyk}{0.30,0,0.94,0}
\definecolor{uniblue}{cmyk}{0.80,0.55,0,0}
\definecolor{unigray}{cmyk}{0.40,0.31,0.29,0}
\definecolor{mygray}{gray}{.9}
%\definecolor{unigrey}{rgb}{0.667,0.667,0.667}
%\definecolor{uniblue}{rgb}{0.227,0.427,0.686}
%\definecolor{unigreen}{rgb}{0.784,0.827,0.09}
%\definecolor{unigreen}{cmyk}{1,0,0,0} % Pantone 382
%\definecolor{uniblue}{cmyk}{0,1,0,0}  % Pantone 285
%\definecolor{unigray}{cmyk}{0,0,1,0}  % Pantone Cool Grey 6
\definecolor{lightgreen}{rgb}{.467,.621,.176}
\definecolor{yellow}{rgb}{1,1,0}
\definecolor{darkyellow}{rgb}{.6,.6,0}
\definecolor{lightgray}{rgb}{0.937,0.937,0.937}

%------------------------------
%
% Eigene Boxen
%
%------------------------------
% \cbox[2]           -> #1 Breite (default: Textbreite),
%                       #2 Inhalt, hellgrau hinterlegte
%                       Box mit blauer Schrift.
% \ttbox[2]          -> wie \cbox nur mit Typewriter-Schrift
% \CBOX[2]           -> wie \ttbox, allerdings mit Quellcode
%                       (#2), #1 sind Optionen f\"ur das
%                       listings-Paket
% \topalignbox[1]    -> Box auf Headline statt Baseline
% \centeralignbox[1] -> Box vertikal zentriert statt auf Baseline
\newcommand{\cbox}[2][\linewidth]{%
\colorbox{lightgray}{%
\parbox{#1-2\fboxsep}{%
\color{uniblue}%
#2}}}

\newcommand{\ttbox}[2][\linewidth]{%
\colorbox{lightgray}{%
\parbox[c]{#1-2\fboxsep}{%
\color{uniblue}%
\ttfamily{#2}}}}

\newcommand{\CBOX}[2][basicstyle=\ttfamily]{%
\mbox{\lstinputlisting[#1]{#2}}%
}

\newlength\topgraphics

\newcommand{\topalignbox}[1]{%
  \setbox0=\hbox{#1}%
  \setlength{\topgraphics}{-\ht0}%
  \setbox0=\hbox{Y}%
  \addtolength{\topgraphics}{\ht0}%
  \raisebox{\topgraphics}{#1}%
}

\newlength\centergraphics

\newcommand{\centeralignbox}[1]{%
  \setbox0=\hbox{#1}%
  \setlength{\centergraphics}{-.5\ht0}%
  \setbox0=\hbox{Y}%
  \addtolength{\centergraphics}{\ht0}%
  \raisebox{\centergraphics}{#1}%
}

%------------------------------
%
% Imagedefinitionen f\"ur PGF
%
%------------------------------
\pgfdeclareimage[height=5mm]{RRZN}{./Abbildungen/rrzn_logo_tmp}
\pgfdeclareimage[height=5mm]{UNI}{./Abbildungen/luh_logo_cmyk_neu} % 17.5mm breit
\pgfdeclareimage[width=64mm]{CPU}{./Abbildungen/cpu}

%------------------------------
%
% Beamertemplate-\"Anderungen
%
%------------------------------
% Kopfzeile:
\setbeamertemplate{headline}{%
\vspace{3mm}\hspace{3mm}%
\pgfuseimage{RRZN}\hfill\pgfuseimage{UNI}%
\hspace{3mm}\mbox{}%
}

% Fu{\ss}zeile:
\setbeamertemplate{footline}{%
\begin{beamercolorbox}[ht=1mm]{colorbox}
\color{unigreen}%
\hspace{107.5mm}\rule{17.5mm}{1mm}
\end{beamercolorbox}
\begin{beamercolorbox}[leftskip=3mm,rightskip=3mm,ht=4mm]{colorbox}
\textbf{\insertauthor}, \inserttitle{}, \insertdate\hfill{}Seite \insertframenumber/\inserttotalframenumber%
\vspace{1.5mm}\vbox{}%
\end{beamercolorbox}
}
% Folientitel:
\setbeamertemplate{frametitle}{%
  \begin{beamercolorbox}%
        [wd=\paperwidth,leftskip=5mm,%
         rightskip=5mm,vmode]{frametitle}%
     \usebeamerfont*{frametitle}%
     \textbf{\insertframetitle}%
   \end{beamercolorbox}}
% Kleinkram ...:
\setbeamertemplate{navigation symbols}{}
%\setbeamertemplate{blocks}[rounded][shadow=true]
\setbeamertemplate{enumerate items}[square]
\setbeamertemplate{itemize items}{%
  \raisebox{1.5pt}{\scriptsize\vrule width1ex height1ex}}
\setbeamertemplate{itemize subitem}{%
  \raisebox{1.5pt}{\scriptsize\fboxsep0pt\fboxrule1pt\fbox{%
  \color{white}\rule{1ex-2pt}{1ex-2pt}}}}

%------------------------------
%
% Beamerfarben
%
%------------------------------
\setbeamercolor{colorbox}{fg=uniblue,bg=lightgray}
\setbeamercolor{frametitle}{fg=uniblue}
\setbeamercolor{title}{fg=uniblue}
\setbeamercolor{item}{fg=unigreen}
\setbeamercolor{subitem}{fg=unigreen}
\setbeamercolor{description item}{fg=uniblue}
\setbeamercolor{background canvas}{bg=}
\setbeamercolor{alerted text}{fg=uniblue}
\setbeamercolor{block body}{bg=lightgray}
\setbeamercolor{block title}{bg=unigray}

%------------------------------
%
% Beamergr\"o{\ss}en
%
%------------------------------
\setbeamersize{description width=5mm}
\setbeamersize{text margin left=5mm}
\setbeamersize{text margin right=5mm}

\begin{document}
\Zeitraum{}
% Beispieldokument:
%\input{beamer-praeambel-vortrag.tex}
%
%\author{Prof. Max Mustermann}
%\title{\latex vs. Microsoft Word\textsuperscript{\texttrademark}}
%\Email{mustermann@latex.org}
%\date{\today}
%\Zeitraum{1. -- 10. Januar 1970}
%
%\beamertitlepage
%
%\neueseite{Inhalt}{
%\latex ist \enquote{What you see is what you mean.}
%
%Word ist \enquote{What you see is what you might get.}
%}
%
%\end{document}

%
%\author{Prof. Max Mustermann}
%\title{\latex vs. Microsoft Word\textsuperscript{\texttrademark}}
%\Email{mustermann@latex.org}
%\date{\today}
%\Zeitraum{1. -- 10. Januar 1970}
%
%\beamertitlepage
%
%\neueseite{Inhalt}{
%\latex ist \enquote{What you see is what you mean.}
%
%Word ist \enquote{What you see is what you might get.}
%}
%
%\end{document}

%
%\author{Prof. Max Mustermann}
%\title{\latex vs. Microsoft Word\textsuperscript{\texttrademark}}
%\Email{mustermann@latex.org}
%\date{\today}
%\Zeitraum{1. -- 10. Januar 1970}
%
%\beamertitlepage
%
%\neueseite{Inhalt}{
%\latex ist \enquote{What you see is what you mean.}
%
%Word ist \enquote{What you see is what you might get.}
%}
%
%\end{document}

%
%\author{Prof. Max Mustermann}
%\title{\latex vs. Microsoft Word\textsuperscript{\texttrademark}}
%\Email{mustermann@latex.org}
%\date{\today}
%\Zeitraum{1. -- 10. Januar 1970}
%
%\beamertitlepage
%
%\neueseite{Inhalt}{
%\latex ist \enquote{What you see is what you mean.}
%
%Word ist \enquote{What you see is what you might get.}
%}
%
%\end{document}
