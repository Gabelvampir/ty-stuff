% !TEX TS-program = pdflatex
% !TEX encoding = UTF-8 Unicode

% This is a simple template for a LaTeX document using the "article" class.
% See "book", "report", "letter" for other types of document.

\documentclass[12pt,a4paper,titlepage]{scrartcl} % use larger type; default would be 10pt

\usepackage[utf8x]{inputenc} % set input encoding (not needed with XeLaTeX)

%%% Examples of Article customizations
% These packages are optional, depending whether you want the features they provide.
% See the LaTeX Companion or other references for full information.

%%% PAGE DIMENSIONS
\usepackage{geometry} % to change the page dimensions
%\geometry{a4paper} % or letterpaper (US) or a5paper or....
\geometry{bottom=3.5cm} % for example, change the margins to 2 inches all round
% \geometry{landscape} % set up the page for landscape
%   read geometry.pdf for detailed page layout information

\usepackage{graphicx} % support the \begin{center}\includegraphics command and options
\usepackage{wrapfig}

% \usepackage[parfill]{parskip} % Activate to begin paragraphs with an empty line rather than an indent

%%% PACKAGES
%\usepackage{booktabs} % for much better looking tables
%\usepackage{array} % for better arrays (eg matrices) in maths
\usepackage{paralist} % very flexible & customisable lists (eg. enumerate/itemize, etc.)
\usepackage{verbatim} % adds environment for commenting out blocks of text & for better verbatim
%\usepackage{subfig} % make it possible to include more than one captioned figure/table in a single float
% These packages are all incorporated in the memoir class to one degree or another...
\usepackage[ngerman]{babel}
\usepackage{pifont} %for symbols (i.e. arrows)
\usepackage{fancyhdr}
%\usepackage{showframe} %shows the margins

\usepackage[colorlinks]{hyperref} % package for hyperlinks with \url
\usepackage[svgnames]{xcolor}
\usepackage[anythingbreaks]{breakurl}
\usepackage{listings}

%%redifine of emph, see http://tex.stackexchange.com/questions/6754/what-is-the-canonical-way-to-redefine-the-emph-command
\makeatletter
\DeclareRobustCommand{\em}{%
  \@nomath\em \if b\expandafter\@car\f@series\@nil
  \normalfont \else \bfseries \fi}
\makeatother

%%% HEADERS & FOOTERS
%set with fancy layout package
\usepackage{fancyhdr} % This should be set AFTER setting up the page geometry
\pagestyle{fancy} % options: empty , plain , fancy
\renewcommand{\headrulewidth}{0pt} % customise the layout...
\lhead{}\chead{}\rhead{}
\lfoot{}\cfoot{\thepage}\rfoot{}

\setlength{\parindent}{0mm} %set paragraph begin indentation to 0

% hyperlink color definitions
\hypersetup{citecolor=DeepPink4}
\hypersetup{linkcolor=DarkRed}
\hypersetup{urlcolor=DarkBlue} 

%%% SECTION TITLE APPEARANCE
\usepackage{sectsty}
\allsectionsfont{\sffamily\mdseries\upshape} % (See the fntguide.pdf for font help)
% (This matches ConTeXt defaults)

%%% ToC (table of contents) APPEARANCE
%\usepackage[nottoc,notlof,notlot]{tocbibind} % Put the bibliography in the ToC
%\usepackage[titles,subfigure]{tocloft} % Alter the style of the Table of Contents
%\renewcommand{\cftsecfont}{\rmfamily\mdseries\upshape}
%\renewcommand{\cftsecpagefont}{\rmfamily\mdseries\upshape} % No bold!

%%% END Article customizations

%%% The "real" document content comes below...

\title{Videostreaming mit dem Raspberry Pi und der Raspberry Pi Camera}
\author{Sebastian Deußer}
%\date{2. April 2014} % Activate to display a given date or no date (if empty),
         % otherwise the current date is printed 
\setcounter{section}{-1} % sets the section counter to start with 0

\begin{document}
\maketitle %title (page)

%header and footer definitions for fancyhdr
\pagestyle{fancy}
\lhead{}
\chead{\leftmark}
\rhead{}
\lfoot{Sebastian Deußer}
\cfoot{}
\rfoot{Seite \thepage}

\thispagestyle{fancy}

\section{Vorbereitungen}
Wir haben auf unserem Raspberry Pi das Raspbian System laufen. In raspi-config haben wir die Kamera angeschaltet. Für Methode 2 muss man zusätzlich zu den Standard-Paketen noch vlc installieren.
	\begin{center}
		\includegraphics[height=9cm]{Streaming/1367_MED}
	\end{center}
Die Raspberry Pi Kamera muss wie auf dem Foto zu sehen ist angeschlossen werden, mit den Kontakten des Flachbandkabels Richtung HDMI-Port. Beim Einbau muss man etwas vorsichtig sein da die Kamera empfindlich ist gegen elektrostatische Entladungen.

\newpage
\section{Methode 1: Mit MJPG-Streamer}
\small{(Quelle: \href{http://blog.miguelgrinberg.com/post/how-to-build-and-run-mjpg-streamer-on-the-raspberry-pi}{http://blog.miguelgrinberg.com/post/how-to-build-and-run-mjpg-streamer-on-the-raspberry-pi})}
\normalsize
\begin{enumerate}
\bfseries %number and `title' of an item shall be bold

\item build Abhängigkeiten von mjpg-streamer installieren:\newline
\texttt{\$ sudo apt-get install libjpeg8-dev imagemagick libv4l-dev}

\item Symlink für fehlende videodev.h setzen:\newline
\textnormal{Die videodev.h Header Datei die MJPG-Streamer benötigt wurde durch \mbox{videodev2.h} ersetzt. Um MJPG-Streamer trotzdem glücklich zu machen muss man folgenden symbolischen Link erstellen:}\newline
\texttt{\$ sudo ln -s /usr/include/linux/videodev2.h /usr/include/linux/videodev.h}

\item Download MJPG-Streamer:\newline
\textnormal{Der Quellcode für MJPG-Streamer ist bei sourceforge.net erhältlich, aber ein direkter Download-Link ist etwas schwierig zu bekommen:}\newline
\texttt{\$ wget \url{http://sourceforge.net/code-snapshots/svn/m/mj/mjpg-streamer/code/mjpg-streamer-code-182.zip}}

\item MJPG-Streamer Quellcode entpacken:\newline
\textnormal{Der downgeloadete Quellcode ist ein gepacktes zip file. Entpacke die Datei im Home-Verzeichnis (oder in einem temporären Verzeichnis):}\newline
\texttt{\$ unzip mjpg-streamer-code-182.zip}

\item MJPG-Streamer bauen:\newline
\textnormal{MJPG-Streamer hat einige Plugins, aber nur ein paar davon werden hier benötigt. Folgender Befehl baut nur das Nötigste:}\newline
\texttt{\$ cd mjpg-streamer-code-182/mjpg-streamer\newline
\$ make mjpg\_streamer input\_file.so output\_http.so}

\item MJPG-Streamer installieren:\newline
\textnormal{Die folgenden Befehle installieren das gebaute Programm:}\newline
\texttt{\$ sudo cp mjpg\_streamer /usr/local/bin\newline
\$ sudo cp output\_http.so input\_file.so /usr/local/lib/\newline
\$ sudo cp -R www /usr/local/www}

\item Starte die camera:\newline
\textnormal{Nun ist es Zeit das Kamera-Modul anzuwerfen:}\newline
\texttt{\$ mkdir /tmp/stream\newline
\$ raspistill --nopreview -w 640 -h 480 -q 5 -o /tmp/stream/pic.jpg -tl 100 -t 9999999 -th 0:0:0 \&}\newline
\textnormal{Natürlich kann man raspistill auch mit anderen Optionen starten.}

\item Starte MJPG-Streamer:\newline
\textnormal{Die Kamera schreibt nun Bilder, man muss nur noch MJPG-Streamer starten:}\newline
\texttt{\$ LD\_LIBRARY\_PATH=/usr/local/lib mjpg\_streamer -i \char`\"input\_file.so -f /tmp/stream -n pic.jpg\char`\" -o \char`\"output\_http.so -w /usr/local/www\char`\"} % \char`\" is a crude hack for straight quotation marks

\item Schau den Stream an!\newline
\textnormal{Nun kann man sich über einen Web-Browser verbinden und den Live-Stream ansehen. Man kann ihn vom Raspberry Pi selbst per Eingabe von \url{http://localhost:8080} in der Browser-Adresszeile ansehen. Oder wenn man ihn von einem anderen Rechner im Netzwerk ansehen will muss man \url{http://<IP-Adresse>:8080} aufrufen.}\newline
	    \includegraphics[width=0.6\textwidth]{Streaming/MJPG-Streamer_cut}

\item Aufräumen:\newline
\textnormal{Nachdem man überprüft hat das alles geht kann man den Quellcode wieder löschen, er wird nicht mehr benötigt.}\newline
\texttt{\$ cd ../../\newline
\$ rm -rf mjpg-streamer-182}
\end{enumerate}

\newpage
\section{Methode 2: H.264 Streaming mit VLC}
\small{(Quelle: \href{http://www.mybigideas.co.uk/RPi/RPiCamera/}{http://www.mybigideas.co.uk/RPi/RPiCamera/})}
\normalsize

Nachdem die Raspberry Pi Kamera über raspi-config eingerichtet wurde kann man das Streaming über VLC über RTSP wie folgt aktivieren:

\ttfamily
\begin{lstlisting}[breaklines]
raspivid -o - -t 9999999 |cvlc -vvv stream:///dev/stdin --sout '#rtp{sdp=rtsp://:8554/}' :demux=h264
\end{lstlisting}

\normalfont
(bei unserem Raspberry hat die Netzwerkschnittstelle Streaming mit vollen 1080p nicht verkraftet, testweise haben wir mit -w 640 -h 480 das Bild auf 640x480 runtergesetzt)
Mit diesem Befehl streamt der Pi auf Port 8554. Man kann sich dies anschaun indem man auf einem PC VLC startet und einen “Network Stream” öffnet mit der Adresse \url{rtsp://<IPofRaspberryPi>:8554}. RTSP funktioniert gut im lokalen Netz, aber es ist schwierig dies über den Router ins Internet zu streamen. Dies kann man umgehen indem man VLC auf Streaming über HTTP konfiguriert. Allerdings braucht VLC dabei etwas mehr CPU und es kann vorkommen das Frames gedropt werden. Dafür kann man dies über einen einfachen Port-Forward routen.

\ttfamily
\begin{lstlisting}[breaklines]
raspivid -o - -t 9999999 |cvlc -vvv stream:///dev/stdin --sout '#standard{access=http,mux=ts,dst=:8554}' :demux=h264
\end{lstlisting}

\normalfont
Dies kann man sich wieder über VLC auf der Adresse \url{http://<IPofRaspberryPi>:8554/ ansehen.}

Eine weitere Möglichkeit ist Video-Streaming über HLS (HTTP Live Streaming). Dieses scheint für iOS Geräte besser zu sein. Damit dies funktioniert muss ein Webserver auf dem Pi installiert sein. Um einen HLS Stream zu starten ruft man folgendes auf:

\ttfamily
\begin{lstlisting}[breaklines]
raspivid -o -  -w 920 -h 540 -t 9999999 |  vlc -v -I "dummy" stream:///dev/stdin :sout="#std{access=livehttp{seglen=10,delsegs=true,numsegs=5, index=/var/www/streaming/stream.m3u8, index-url=http://<IPofRaspberryPi>/streaming/stream-########.ts}, mux=ts{use-key-frames}, dst=/var/www/streaming/stream-########.ts}" :demux=h264
\end{lstlisting}

\normalfont
Diesen kann man über \url{http://<IPofRaspberryPi>/streaming/stream.m3u8} ansehen, allerdings braucht dies noch mehr CPU auf dem Pi als das Streaming über http.
\end{document}
