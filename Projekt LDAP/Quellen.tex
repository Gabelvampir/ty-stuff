\addsec{Quellen}
%\addcontentsline{toc}{section}{Quellen}
\begin{itemize}
\item Deckblattvorlage von \url{http://f.macke.it/LaTeXVorlageFIAE} unter Creative Commons
\item Informationen zur Entwicklung von LDAP aus \url{http://de.wikipedia.org/wiki/Lightweight_Directory_Access_Protocol#Geschichte}
\item Homepage des OpenLDAP-Projekts: \url{http://www.openldap.org}
\item Homepage des FreeRADIUS-Projekts: \url{http://freeradius.org}
\item Oliver Liebel, John Martin Ungar - OpenLDAP 2.4: das Praxisbuch (2009, Galileo Computing)
\item Jonathan Hassel - RADIUS (2003, O'Reilly)
\item Bei der Installation von OpenLDAP wurde die folgende Anleitung aus dem Debian Wiki konsultiert: \url{https://wiki.debian.org/LDAP/OpenLDAPSetup}
\item Zur Anbindung von FreeRADIUS an OpenLDAP wurde folgende Anleitung im Debian Wiki konsultiert (und für stellenweise veraltet befunden) \url{https://wiki.debian.org/FreeRadiusToLdap}
\item OpenLDAP Log-Level: \url{http://www.openldap.org/doc/admin24/runningslapd.html}
\item Grafik des LDAP-Verzeichnisbaums erstellt mit Graphviz: \url{http://www.graphviz.org}
\item Diese Dokumentation wurde erstellt mit \LaTeX{} (\url{http://www.latex-project.org}) und verwaltet in Git (\url{http://git-scm.com})
\end{itemize}
%\hilight{TODO: Quellen auflisten}

%\nocite{*}
%\begin{thebibliography}{9}
%
%\bibitem{lamport94}
%Leslie Lamport,
%\emph{\LaTeX: a document preparation system},
%Addison Wesley, Massachusetts,
%2nd edition,
%1994.
%
%\end{thebibliography}
