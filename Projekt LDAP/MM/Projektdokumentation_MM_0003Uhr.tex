% !TEX TS-program = pdflatex
% !TEX encoding = UTF-8 Unicode

% This is a simple template for a LaTeX document using the "article" class.
% See "book", "report", "letter" for other types of document.

\documentclass[11pt,a4paper,titlepage=firstiscover]{scrartcl} % use larger type; default would be 10pt

\usepackage[utf8]{inputenc} % set input encoding (not needed with XeLaTeX)
\usepackage[T1]{fontenc} % maps glyphs to dictonary characters, needed for seperation

%%% Examples of Article customizations
% These packages are optional, depending whether you want the features they provide.
% See the LaTeX Companion or other references for full information.

%%% PAGE DIMENSIONS
\usepackage{geometry} % to change the page dimensions
%\geometry{a4paper} % or letterpaper (US) or a5paper or....
\geometry{top=3.5cm,bottom=3.5cm} % for example, change the margins to 2 inches all round
% \geometry{landscape} % set up the page for landscape
%   read geometry.pdf for detailed page layout information

\usepackage{graphicx} % support the \begin{center}\includegraphics command and options
\usepackage{float}
\usepackage{tocstyle}

% \usepackage[parfill]{parskip} % Activate to begin paragraphs with an empty line rather than an indent

%%% PACKAGES
%\usepackage{booktabs} % for much better looking tables
%\usepackage{array} % for better arrays (eg matrices) in maths
\usepackage{paralist} % very flexible & customisable lists (eg. enumerate/itemize, etc.)
\usepackage{verbatim} % adds environment for commenting out blocks of text & for better verbatim
%\usepackage{subfig} % make it possible to include more than one captioned figure/table in a single float
% These packages are all incorporated in the memoir class to one degree or another...
\usepackage[ngerman]{babel}
\usepackage{blindtext}
%\usepackage{pifont} %for symbols (i.e. arrows)
%\usepackage{showframe} %shows the margins

\usepackage[colorlinks,linkcolor=blue]{hyperref} % package for hyperlinks with \url
%\usepackage[svgnames]{xcolor}
%\usepackage[anythingbreaks]{breakurl}
%\usepackage{listings}

\newcommand{\hilight}[1]{\colorbox{yellow}{#1}} %command for magic marker highlighting
%   (from http://pleasemakeanote.blogspot.de/2009/08/how-to-highlight-text-in-latex.html)

%%redifine of emph, see http://tex.stackexchange.com/questions/6754/what-is-the-canonical-way-to-redefine-the-emph-command
\makeatletter
\DeclareRobustCommand{\em}{%
  \@nomath\em \if b\expandafter\@car\f@series\@nil
  \normalfont \else \bfseries \fi}
\makeatother

%%% HEADERS & FOOTERS
%set with fancy layout package
\usepackage{fancyhdr} % This should be set AFTER setting up the page geometry
\pagestyle{fancy} % options: empty , plain , fancy
\renewcommand{\headrulewidth}{0pt} % customise the layout...
\lhead{}\chead{}\rhead{}
\lfoot{}\cfoot{\thepage}\rfoot{}

%\setlength{\parindent}{0mm} %set paragraph begin indentation to 0

% hyperlink color definitions
%\hypersetup{citecolor=DeepPink4}
%\hypersetup{linkcolor=DarkRed}
%\hypersetup{urlcolor=DarkBlue} 

%%% SECTION TITLE APPEARANCE
%\usepackage{sectsty}
%\allsectionsfont{\sffamily\mdseries\upshape} % (See the fntguide.pdf for font help)
% (This matches ConTeXt defaults)

%%% ToC (table of contents) APPEARANCE
%\usepackage[nottoc,notlof,notlot]{tocbibind} % Put the bibliography in the ToC
%\usepackage[titles,subfigure]{tocloft} % Alter the style of the Table of Contents
%\renewcommand{\cftsecfont}{\rmfamily\mdseries\upshape}
%\renewcommand{\cftsecpagefont}{\rmfamily\mdseries\upshape} % No bold!

%\usepackage{uarial}
\usepackage{helvet}
\renewcommand{\familydefault}{\sfdefault}

%%% END Article customizations

%%% The "real" document content comes below...

\titlehead{taylorix institut für berufliche Bildung e.V.}
\title{Projektdokumentation "Aufsetzen eines Authentifizierungsservers als Ersatz eines veralteten proprietären CommuniGate Servers"}
\author{Sebastian Deußer}
%\date{4. April 2014} % Activate to display a given date or no date (if empty),
         % otherwise the current date is printed 
%\setcounter{section}{-1} % sets the section counter to start with 0

\begin{document}
\maketitle %title (page)

%header and footer definitions for fancyhdr
\pagestyle{fancy}
\lhead{}
\chead{\leftmark}
\rhead{}
\lfoot{Sebastian Deußer}
\cfoot{}
\rfoot{Seite \thepage}

\newcommand{\tcr}[1]{\textcolor{red}{#1}}
\newcommand{\tcg}[1]{\textcolor{green}{#1}}

\thispagestyle{empty}
\tableofcontents
\newpage
\thispagestyle{fancy}
\setcounter{page}{1}  %Seitenzahlen erst nach TOC zählen

\section{Einleitung}
In den meisten Firmennetzwerken wird eine zentrale Stelle \tcg{benötigt, um } 
verschiedene \tcr{Anwender-Metadaten?} zu verwalten. Für diese Aufgabe wird 
üblicherweise ein Verzeichnisdienst verwendet. Basis eines Verzeichnisdienstes 
ist eine hierarchische Datenbank, die im Netzwerk verteilt angelegt sein kann. 
In dieser Datenbank können Daten zu verschiedenen Objekten abgelegt \tcg{werden, 
wie} etwa Konfigurationsdaten für Rechner oder diverse Daten für \tcg{Anwender, 
wie} z.B. Name, Passwort, \tcr{E-Mail-Konto}, Gruppenzugehörigkeiten \tcr{usw.}. 
Die Anforderungen an einen Verzeichnisdienst wurden in den 80er Jahren von der 
International Telecommunication Union in der \tcr{X.500-Spezifikation} 
\tcg{festgeschrieben, nachdem} weltweit Telekommunikationsunternehmen 
jahrzehntelang Praxiserfahrungen mit dem Thema bei Erstellung und Verwaltung 
von \tcr{Telefon-Verzeichnissen} sammelten und \tcr{dann$\to$diese?} zur 
Spezifikation beisteuerten.

In der Praxis stellte es sich als unpraktikabel für so gut wie alle Firmen 
\tcr{- abgesehen von großen Telekommunikationsunternehmen -} \tcr{heraus, die} 
gesamte \tcr{X.500-Spezifikation} zu \tcr{implementieren$\to$umzusetzen, da} dies 
einen hohen Implementierungsaufwand darstellt und der Betrieb sehr \tcr{Hardware
intensiv$\to$hardwareintensiv} ist. Deswegen wurde an der Universität von 
Michigan 1993 \tcr{das} LDAP (Lightweight Directory Access Protocol) entwickelt,
\tcr{das$\to$welches} ursprünglich als \tcr{abgespeckte$\to$?Synonym suchen?} 
Alternative \tcr{zum} DAP (Directory Access Protocol) \tcr{dienen sollte}. 
\tcr{das$\to$Das} \tcr{DAP war} traditionell \tcr{zum$\to$für den} Zugriff auf 
\tcr{X.500-Verzeichnisse} verwendet \tcr{wurde$\to$worden}.\newline
\tcr{Der Ansatz von$\to$Die Innovation beim?} LDAP \tcg{war, den} Zugriff durch 
den \tcr{TCP/IP-Protokollstapel} \tcr{zur$\to$zu} ermöglichen, was weit weniger 
\tcr{Implementierungsaufwand$\to$Aufwand} darstellt, als die komplette 
Implementierung \tcr{alles$\to$aller} \tcg{OSI-Schichten, wie} es bei DAP der 
Fall ist. Heutzutage ist LDAP das verbreiteste Protokoll zur Abfrage von 
Verzeichnisdiensten, u.a. \tcg{auch, da} TCP/IP-basierte Netzwerke \tcr{der 
verbreitesten Netzwerktyp ist$\to$ am verbreitetsten sind}. 
\tcr{u.a. auch$\to$neben den bereits oben erwähnten Vorteilen gegenüber dem 
DAP?} LDAP wird in vielen Anwendungen eingesetzt, vor allem im Adressbuchteil 
der meisten \tcr{E-Mail-Clients} wie z.B. Apple Adressbuch, Microsoft Outlook, 
Mozilla Thunderbird und in Verzeichnisdiensten wie z.B. Microsoft Active 
Directory Services, Apple Open Directory und dem aussterbenden Pionier Novell 
eDirectory. Wie viele Protokolle ist LDAP \tcr{Client/Server-basiert}.

In diesem Projekt wurde OpenLDAP verwendet, die verbreiteste freie \tcr{Open 
Source-Implementierung} des Netzwerkstandards und steht unter einen eigenen 
freien Lizenz, der OpenLDAP Public License. OpenLDAP ist außerdem die 
Referenzimplementierung des Standards, weswegen es in vielen Belangen (z.B. bei 
Schemadateien) mehr auf Protokollkonformität \tcr{achtet? hin ausgerichtet ist?} 
als andere Implementierungen. \tcr{Der$\to$Im?} \tcr{OpenLDAP-Server} ist nicht 
nur das Abfrageprotokoll \tcg{enthalten, sondern} auch ein Verzeichnisdienst. 
Es ist somit für viele Anwendungsfälle eine kostengünstige Lösung zum Aufbau 
eines Verzeichnisdienstes \tcr{Hier Reihenfolge umgestellt}.

RADIUS (Remote Authentication Dial-In User Service) ist ebenfalls ein 
\tcr{Client/Server-basiertes} Protokoll zur Authentifizierung, Autorisierung und 
\tcr{zum} Accounting (AAA-System) von Benutzern bei Einwahlverbindungen zu 
Netzwerken. RADIUS stellt den \tcr{de-facto Standard$\to$de facto-Standard} zur 
zentralen Authentifizierung von Einwahlverbindungen, wie z.B. über ISDN, DSL 
und WLAN (über IEEE 802.1X) dar. Üblicherweise ist ein RADIUS-Server an einen 
Verzeichnisdienst \tcg{angebunden, um} von ihm die Benutzerdaten für die 
Authentifizierung und Autorisierung (und teilweise auch \tcr{für das} 
Accounting) abzufragen.

Hier wurde FreeRADIUS verwendet, der \tcr{-} laut Aussage des Projektes \tcr{-} 
weltweit verbreiteste \tcr{RADIUS-Server}. Das freie \tcr{Open Source-Projekt} 
unter GPLv2\tcr{-Lizenz?} umfasst neben dem \tcr{RADIUS-Server} außerdem eine 
\tcr{PAM-Bibliothek}, ein \tcr{Apache-Webserver-Modul} und eine 
\tcr{Client-Bibliothek} (im Gegensatz zum Rest unter \tcr{BSD-Lizenz}).
\tcr{Rest von was?}


\section{Projektbeschreibung}
\subsection{Projektumfeld}
Die fgn GmbH wurde im August 2000 als SpinOff der Technische Universität (TU) 
Kaiserslautern gegründet. Die Gründer waren zuvor mehrere Jahre (seit 1996 bzw. 
1989) als freischaffende Consultants und Trainer tätig. Die Firma pflegt enge 
Kontakte zum Regionalen Hochschulrechenzentrum Kaiserslautern (RHRK), da die 
meisten Mitarbeiter das Netz der TU Kaiserslautern mit ca. 10.000 Ports, 
Diensten wie Mail, DNS und DHCP in der Vergangenheit betreut haben oder es 
\tcr{heute noch$\to$noch heute} betreuen.

Die Kernkompetenz der fgn GmbH ist anspruchsvolles Netzwerk-Knowhow, welches 
als Dienstleistung in drei eng verknüpften Tätigkeitsfeldern angeboten wird: 
Schulungen, Workshops und Netzwerk-Consulting (Beratung und \tcr{vor-Ort-Support, 
auch wenns komisch aussieht} von Firmen bei Problemen, Umstrukturierungen, 
Erweiterungen und Neuaufbau von Produktivnetzwerken).

\subsection{Ist-Analyse}
Im Praktikumsbetrieb fgn GmbH läuft der \tcr{E-Mail-Verkehr} und die 
Authentifizierung an den internen Webservices \tcr{und am?} \tcr{OpenVPN-Server} 
über einen alten \tcr{CommuniGate-Server} (v5.0.13 von November 2006). 
Ursprünglich wurde dieser auf einem eigenen Rechner aufgesetzt, inzwischen aber 
\tcr{-} wie viele andere Rechner der Firma \tcr{-} virtualisiert. 

Die Webservices, die ihn zur Anwender-Authentifizierung verwenden, laufen auf 
drei anderen VMs, ebenso der \tcr{OpenVPN-Server}. Sie kommunizieren mit dem 
LDAP-Teil von CommuniGate mittels den \tcr{Apache-Modul} \texttt{mod\_ldap} und 
verwenden zur Autorisierung entsprechend \texttt{mod\_authnz\_ldap} (bzw. auf 
einem \tcr{der} Rechner wegen eines veralteten \tcr{Apache-Webservers} noch 
\texttt{mod\_auth\_ldap}). Apache prüft dabei momentan nur auf Existenz des 
\tcr{angegebene$\to$angegebenen} Benutzeraccounts und ob das richtige Passwort 
angegeben wurde, weitere Berechtigungen sind momentan nicht implementiert. Die 
Ausnahmen dabei sind Nagios und Egroupware, aber dort werden die 
Berechtigungsgruppen intern verwaltet und nicht in LDAP abgelegt.\tcr{Ausnahme 
wovon? Ich kapiers grad nich}

Lediglich die Anwenderkonten auf den Betriebssystemen der diversen Anwender-PCs 
sind nicht von CommuniGate abhängig. Da die fgn GmbH über das Netz der 
Technischen Universität Kaiserslautern angebunden \tcg{ist, müssen} alle E-Mails 
auch über die Mailserver der Universität laufen. Diese verwenden \tcg{RADIUS, um} 
zu \tcg{prüfen, ob} die \tcr{E-Mail-Konten} der 
\tcr{fg-networking.de Domain$\to$besser 'Domain fg-networking.de'} tatsächlich 
vorhanden sind. Der \tcr{RADIUS-Server} dazu wird ebenso von CommuniGate 
bereitgestellt. 

Wegen der recht alten Softwareversion gibt es schon seit längerem regelmäßig 
Probleme, z.B. mit der \tcr{SSL-Authentifizierung} neuerer \tcr{E-Mail-Clients} 
(die \tcr{unterstützen$\to$unterstützten?} Versionen von SSL/TLS benutzen 
aktuelle \tcr{E-Mail-Clients} aus Sicherheitsgründen nur noch ungern). Ein 
weiteres großes Sicherheitsproblem von CommuniGate \tcg{ist, }\tcr{dass} es die 
Passwörter aller Anwenderkonten im Klartext (ohne sie vorher zu hashen 
\tcr{oder ähnliches$\to$o.ä.}) in Textdateien auf dem Server abspeichert.

\subsection{Soll-Analyse}
Eine Aktualisierung von CommuniGate wäre mit ähnlichem Aufwand \tcg{verbunden, 
wie} \tcr{ihn} ein komplett neues Aufsetzen von Ersatzservern \tcr{erfordern 
würde}(und wäre außerdem mit dem Kauf einer neuen Lizenz verbunden). Deswegen 
wurde \tcg{entschieden, die} Serverkomponenten E-Mail und Identity Management 
durch neue Server abzulösen. \tcr{Da} \tcr{In$\to$in} der Firma momentan sehr 
viel freie Software verwendet \tcg{wird, sollen} die Ersatzserver auch \tcr{auf 
auf$\to$auf} Basis von freier Software aufgesetzt werden. Da die 
\tcr{E-Mail-Infrastruktur} sehr kritisch für die Arbeit der Firma ist und bei 
der Migration der \tcr{E-Mail-Konten} mit vielen vertraulichen firmeninternen 
Informationen hantiert werden \tcg{muss, soll} die Installation des neuen 
\tcr{E-Mail-Servers} von einem Mitarbeiter von fgn durchgeführt werden.

Für das Identity Management sollen OpenLDAP und FreeRADIUS zum Einsatz 
\tcg{kommen, da} dies die \tcr{verbreitesten$\to$verbreitetsten} freien 
Implementierungen von LDAP und RADIUS sind und somit in Büchern und dem 
Internet \tcr{darüber das meiste Know-How} verfügbar ist. Außerdem erleichert 
es die spätere Pflege des Systems, da sich für solche verbreiteten Systeme 
einfacher Personal mit Fachkenntnis finden \tcr{lassen$\to$lässt}\tcg{, als} 
für die meisten anderen proprietären Implementierungen.

Für den \tcr{E-Mail-Teil} soll ein \tcr{\texttt{postfix}-Server} verwendet 
werden, allerdings war dieser zur Fertigstellung dieses Projektes noch nicht 
einsatzfähig. Die Intergration des neuen \tcr{E-Mail-Servers} ist damit nicht 
Teil dieses Projektes und wird zu einem späteren Zeitpunkt durchgeführt. Es 
werden lediglich die Vorarbeiten auf der Seite \tcr{vom$\to$des?} Identity 
Management \tcg{getätigt, die} \tcr{dazu$\to$/dev/null?} unabhängig von der 
endgültigen Konfiguration des neuen \tcr{E-Mail-Servers} sind.

\medskip \noindent 		%überspringt Zeile; verhindert einmalig Einrückung
Somit müssen im Rahmen der Projektarbeit folgende Arbeiten durchgeführt werden: 
\begin{itemize}
	\item \tcr{Linux-Grundsystem} installieren
	\item \tcr{OpenLDAP-Server} auf dem \tcr{Linux-System} installieren
	\item \tcr{OpenLDAP-Server} konfigurieren
	\item Verzeichnisstruktur erstellen
	\item grundlegende Benutzerdaten (Name, Passwort) aus bestehendem System übernehmen
	\item \tcr{FreeRADIUS-Server} installieren
	\item \tcr{FreeRADIUS-Server} konfigurieren und an den \tcr{OpenLDAP-Server} anbinden
	\item Absicherung des Systems
	\item Umkonfiguration der entsprechenden Systeme auf den neuen LDAP-Server
	\item Funktions- und Sicherheitstests
\end{itemize}


\subsection{Vorgaben}
\subsubsection{Wirtschaftliche Vorgaben}
Wie die meisten anderen Server der fgn GmbH soll \tcr{der Server$\to$dieser?} 
in einer neuen VM auf einem bereits bestehenden \tcr{VMware-ESXi-Server} laufen, 
\tcr{somit$\to$womit} keine zusätzlichen Hardwarekosten anfallen. Da in dem 
Projekt \tcr{zudem} ausschließlich freie Software zum Einsatz kommen 
\tcg{soll, fallen} auch keine \tcr{zusätzlichen$\to$/dev/null?} 
Softwarelizenzkosten an.

\subsubsection{Organisatorische Vorgaben}
Das Projekt wird im Praktikumsbetrieb mit Unterstützung des Mitarbeiters Erik 
Auerswald durchgeführt. \tcr{Zur$\to$Neben der} Projektdokumentation \tcr{werden 
zusätzlich jeweils$\to$wird zusätzlich } eine Kundendokumentation für 
Administratoren im firmeninternen Wiki erstellt. Hinzu kommen Funktions- und 
Sicherheitstests zur Qualitätssicherung. Eine Anwenderdokumentation ist nicht 
\tcg{notwendig, da} sich aus Sicht des Anwenders nichts gegenüber \tcr{des 
Ausgangszustandes$\to$dem Ausgangszustand} ändern soll.

\subsubsection{Zeitliche Vorgaben}
Das Projekt wird im Zeitraum vom 04.05.2015 – 18.05.2015 durchgeführt, wobei 
die Bearbeitungszeit von 35 Stunden nicht überschritten werden darf.

\section{Projektplanung}
\subsection{Planung des Ersatzservers}
Als Ersatzserver wird eine neue Virtual Machine auf einen der firmeneigenen 
\tcr{ESXi-Servern} verwendet. Als Betriebssystem \tcr{wird$\to$wurde} Debian Stable (zur 
Durchführungszeit des Projektes Version 8.0 Jessie) mit der Standardpaketauswahl 
ohne zusätzliche Vorauswahlen \tcr{ausgewählt}.

\subsection{Planung des Kommunikationskonzepts}
Sämtliche \tcr{firmeninterne$\to$firmeninternen} Webservices, der VPN, der 
kommende neue \tcr{E-Mail-Server} und der \tcr{FreeRADIUS-Daemon} greifen über 
das LDAP-Protokoll auf den \tcr{OpenLDAP-Daemon} \tcg{zu, um} Benutzer zu 
authentifizieren und \tcr{Einstellungen zu diesen zu holen$\to$nutzerspezifische 
Einstellungen abzurufen?}. Der \tcr{E-Mail-Server} der Universität greift auf 
den \tcr{FreeRADIUS-Daemon} \tcg{zu, um} Benutzerkonten zu prüfen, \tcr{welcher 
wiederrum$\to$woraufhin dieser?} die \tcr{entsprechenden?} Daten beim 
OpenLDAP-Daemon erfragt.

\subsection{Planung des Sicherheitskonzepts}\label{sec:Sicherheitskonzept}
Das Firmennetzwerk der fgn GmbH ist über \tcr{die$\to$das?} Netzwerk der TU 
Kaiserslautern ans Internet angebunden. Entsprechend wird jeglicher 
Netzwerkverkehr von der \tcr{TU-Firewall} vorgefiltert. Das Firmennetz von fgn 
ist zusätzlich noch durch eine eigene Firewall gesichert. In diese muss 
\tcr{- analog zur alten Regel -} eine Ausnahme für den neuen \tcr{RADIUS-Server} 
eingetragen werden. Zusätzlich wird auf dem neuen Server eine Software-Firewall 
\tcg{installiert, die} nur die für LDAP, RADIUS und zur Wartung benötigten Ports 
zulassen soll.

\subsection{Projektablaufplan}
Analyse und Planung (insgesamt 6 h)
	\begin{itemize}
	\item Ist-Analyse (3 h)
		\begin{itemize}
		\item Analyse des bestehenden \tcr{CommuniGate-Servers} und der damit 
				verbundenen Webservices (2 h)
		\item Aufnahme der Anforderungen an einen Ersatzserver (1 h)
		\end{itemize}
	\item Planung (3 h)
		\begin{itemize}
		\item Ausarbeitung eines Konzepts für den Ersatzserver (1 h)
		\item Ausarbeitung des Sicherheitskonzepts (unter Berücksichtigung des 
				Firmenkonzepts) (1 h)
		\item Ausarbeitung des Kommunikationskonzepts (Serverdienste untereinander 
				und extern) (1 h)
		\end{itemize}
	\end{itemize}
Umsetzung (insgesamt 20 h)
	\begin{itemize}
	\item Vorbereitungen (6 h)
		\begin{itemize}
		\item Dokumentation der Konfiguration des zu ersetzenden Servers (2 h)
		\item Erstellen einer Liste aller Dienste, die das bestehende Identity 
				Management nutzen (2 h)
		\item Prüfung von Möglichkeiten zum Importieren der bestehenden Anwender-Konten 
				in die neue Lösung (2 h)
		\end{itemize}
	\item Installation und Einrichtung des neuen Servers (8 h)
		\begin{itemize}
		\item Grundinstallation des \tcr{Linux-Systems} des neuen Servers (1 h)
		\item Installation und Konfiguration des \tcr{LDAP-Servers} (3 h)
		\item Installation und Konfiguration des \tcr{RADIUS-Servers} (2 h)
		\item Absicherung des Rechners (Firewall etc.) (2 h)
		\end{itemize}
	\item Abschließende Arbeiten (6 h)
		\begin{itemize}
		\item Umkonfiguration der Webservices (2 h)
		\item Import/Anlegen der \tcr{Anwender-Konten} (2 h)
		\item \tcr{Funkions-$\to$Funktions-} und Sicherheitstests (2 h)
		\end{itemize}
	\end{itemize}
Dokumentation (insgesamt 9 h)
	\begin{itemize}
	\item Erstellen der Projektdokumentation (8 h)
	\item Erstellen der Dokumentation für das firmeninterne Wiki (1 h)
	\end{itemize}

\section{Umsetzung}
\subsection{Vorbereitungen}
\subsubsection{Dokumentation der Konfiguration des zu ersetzenden Servers}\label{sec:Konfig-Doku-Alt}
Da die Konfigurationsdateien des \tcr{CommuniGate-Servers} sämtliche 
Anwenderpasswörter im Klartext \tcr{enthält, konnte mir bei der Durchführung des 
Projektes aus Sicherheitsgründen kein direkter Zugriff auf den Server gewehrt 
werden$\to$enthalten, kann firmenexternen Personen aus Sicherheitsgründen kein 
direkter Zugriff auf den Server gewährt werden.} Allerdings war dies \tcr{bei 
der Durchführung des Projektes} auch nicht notwendig, da \tcr{zur$\to$zum} 
Erfassen der wichtigen Konfigurationsdetails (vornehmlich der Struktur des 
integrierten Verzeichnisdienstes) lediglich Zugriff auf das Webinterface des 
Servers (zugänglich unter Port \texttt{9010} auf dem bisherigen Server 
\texttt{mail.fg-networking.de}, Zugriff durch Firewall von außerhalb des 
Firmennetzes geblockt) notwendig war. Unter Aufsicht von Herrn Auerswald wurde 
ein Auszug der im Klartext gespeicherten Benutzerdaten begutachtet \tcr{, und 
zwar (vor allem?, was denn noch?) die zum für den Durchführenden angelegten 
Nutzeraccount gehören, da hierbei keine Datenschutzprobleme auftreten konnten.}. 
\tcr{Dabei fiel auf das$\to$Dabei konnte beobachtet werden, dass} alles in 
einer gut organisierten Ordnerstruktur abgelegt ist\tcr{,$\to$;} es existiert 
eine Datei pro Anwender und die Daten in \tcr{der Datei$\to$dieser} sind in 
bezeichneten Feldern abgelegt.

Für das Projekt am Wichtigsten zu beachten \tcg{war, }\tcr{dass} der 
\tcr{E-Mail-Server} für \tcr{3$\to$drei (Zahlen unter 10 ausschreiben!)} Domains 
Anwenderkonten verwaltet: \texttt{fg-networking.de}, \texttt{schabler.de} und 
\texttt{worden.de}. Dies musste natürlich beim Entwurf der neuen 
Verzeichnistruktur beachtet \tcg{werden, um} später die Anbindung des neuen 
\tcr{E-Mail-Servers} ohne \tcg{erneute, unnötige} Umbauten am LDAP zu ermöglichen.

\subsubsection{Erstellen einer Liste aller \tcg{Dienste, die} das bestehende 
		Identity Management nutzen}
Die \tcg{Webtools, die} den bisherigen \tcr{LDAP-Server} \tcg{verwenden, laufen} 
unter \tcr{Apache-Webservern} auf den Rechnern \texttt{aio}, \texttt{nms} und 
\texttt{lab-mm}. Sie alle benutzen zum Abfragen von Nutzerdaten die \tcr{Apache 
eigenen$\to$Apache-eigenen} \tcr{LDAP-Module} (\texttt{mod\_ldap} und 
\texttt{mod\_authnz\_ldap}, bzw. \texttt{mod\_auth\_ldap} auf \texttt{aio} da 
hier noch eine ältere \tcr{Apache-Version} verwendet wird). 

Somit ist bei der Konfiguration \tcr{von allen$\to$aller Rechner} prinzipiell 
dasselbe zu \tcg{ändern, damit} später der neue Server verwendet 
\tcr{wird$\to$werden kann?}. So verwenden z.B. alle den alten Server ohne SSL, 
weswegen zusätzlich zur neuen \tcr{LDAP-Server-URI} auch noch das Stammzertifkat 
einzutragen ist. \tcr{Kopiert werden muss es nicht extra da es auf allen Rechnern 
bereits fürs servieren von HTTPS Verbindungen installiert wurde.???} 

Oberflächlich gibt es in der Konfiguration natürlich auch Unterschiede. So 
haben auf \texttt{aio} mehrere Tools eigene Unterseiten \tcr{Einträge$\to$mit 
Einträgen?} in der Konfiguration in denen jeweils der LDAP Server eingestellt 
ist, \tcr{während es} auf \texttt{nms} \tcr{gibt es$\to$/dev/null} lediglich 
einen Eintrag für alle Webtools \tcr{gibt}. Aber diese Detailunterschiede 
machen keinen wirklichen Unterschied \tcr{in$\to$bei} den später vorzunehmenden 
Änderungen.

Die Konfiguration von OpenVPN muss auch lediglich auf die neue \tcr{Server-URL} 
und BaseDN umgestellt \tcr{werden$\to$/dev/null}\tcr{, und ebenso$\to$sowie} 
das Stammzertifikat eingetragen werden. Dies geschieht in der Datei 
\texttt{/etc/openvpn/auth-ldap.config}, näheres dazu siehe \autoref{sec:VPN-Konfig}.

Bei Begutachtung der Egroupware-Konfiguration im Administrationsmenü fiel 
\tcg{auf,} \tcr{dass} hier keine Einstellungen zu LDAP zu finden waren. Nach 
kurzer Recherche stellte sich \tcg{heraus ,}\tcr{ dass} in Egroupware \tcr{-} 
wie bei vielen PHP-Webanwendungen \tcr{-} manche Einstellungen nur im Installer 
\tcr{vornehmbar$\to$vorzunehmen} sind, wie \tcr{hier$\to$/dev/null} z.B. die 
\tcr{LDAP-Einstellungen}. \tcr{Das$\to$Da das} Passwort für diesen Installer 
\tcr{war leider nicht$\to$nicht} \tcr{-} wie sonst üblich \tcr{-} im Firmenwiki 
hinterlegt und der mit dem Installer vertraute Mitarbeiter \tcr{befand sich 
zu$\to$zur} Bearbeitungszeit des Projektes im Urlaub \tcr{. Man$\to$war, hätte} 
\tcr{man} zwar relativ einfach das Passwort des Installers durch 
\tcr{editieren$\to$Editieren} der entsprechenden \tcr{PHP-Datei} ändern können
\tcr{,$\to$.} \tcr{aber da kein Anwesender$\to$Da jedoch keiner der Anwesenden} 
mit dem \tcr{Egroupware-Installer} vertraut war und die Groupware sehr wichtig 
für die Arbeit der Firma \tcg{ist, wurde} \tcg{entschieden, sie} erst später 
vom zuständigen Mitarbeiter auf den neuen \tcr{LDAP-Server} umstellen zu lassen.

\subsubsection{Prüfung von Möglichkeiten zum Importieren der bestehenden Anwenderkonten in die neue Lösung} \label{sec:Importsuche}
Leider bietet CommuniGate selbst keine Funktion zum Export seiner Benutzerdaten 
an. Über die \tcr{Standard-LDAP-Tools} wäre ein Auslesen der Daten möglich, in einer Form die man in das neue LDAP wieder importieren könnte. Da allerdings CommuniGate für die E-Mail bezogenen Attribute ein proprietäres LDAP-Schema benutzt wäre der Nachbearbeitungsaufwand für diese Daten sehr hoch. Es sollen aber einmal nur die notwendigsten Attribute (siehe \autoref{sec:LDAP-Attribute}) übernommen werden und CommuniGate speichert die Anwenderdaten in einer mit den Standard Unix/Linux Textverarbeitungswerkzeugen (\texttt{sed}, \texttt{awk}, \texttt{cut}) vergleichsweise einfach verarbeitbaren Form. 
\tcr{Ich verstehe den Absatz nicht, ist es jetzt gut auszulesen und in einer Form, die man verarbeiten kann, oder nicht?}

Somit stellt das Erzeugen von durch LDAP importierbaren 
\tcr{Datensätze$\to$Datensätzen} aus den gespeicherten Dateien von CommuniGate 
mit Hilfe von Skripten in diesem Fall die sinnvollste Vorgehensweise dar. Da 
das Entwickeln der Skripte ohne Zugriff auf die Daten umständlich \tcr{gewesen} 
\tcg{wäre, hatte} Herr Auerswald sich bereit \tcg{erklärt, diese} zu 
\tcg{schreiben, da} ihm Zugriff auf die Anwenderpasswörter 
\tcr{erlaubt$\to$gestattet?} ist. \tcr{Als$\to$Zur} Unterstützung \tcr{dieser 
Arbeit$\to$bei dieser Aufgabe} \tcr{habe ich$\to$wurde} Herrn Auerswald ein 
Template der zu generierenden \tcr{LDAP-Importdaten} und \tcr{den$\to$der} 
Befehlsaufruf von \texttt{slappasswd} zur Verfügung gestellt. \tcr{zum passenden 
hashen der Passwörter damit die diese in den Importdaten als Hashes stehen und 
nicht mehr im Klartext stehen)$\to$ Letzteres Programm/Skript wird dabei dazu 
verwendet, die Passwörter zu hashen, damit sie in dieser umgewandelten Form in 
die Importdaten eingefügt werden können.}

\subsection{Installation und Einrichtung des neuen Servers}
\subsubsection{Grundinstallation des Linux Systems des neuen Servers}
Nach der Erstellung einer neuen VM durch \tcr{einem$\to$einen} Mitarbeiter von 
fgn auf dem entsprechendem ESXi-Server wurde der Installer der aktuellen 
Debian Stable (Version 8.0 \texttt{Jessie} zur Bearbeitungszeit des Projektes) 
ausgeführt und von ihm \tcr{vom Mitarbeiter?} das Grundsystem installiert. 
Dabei wurden keine zusätzlichen \tcr{Paketvorauswahlen hinzugenommen$\to$
Paketauswahlen vorgenommen?}. Der \tcr{Debian-Installer} hat 
\tcr{bei$\to$/dev/null} während des 
\tcr{Installationsprozeses$\to$Installationsprozesses} nach Rückfragen auch 
die Grundeinrichtung \tcr{vorgenommen$\to$/dev/null} für grundlegende 
\tcr{Einstellung$\to$Einstellungen} wie z.B. \tcr{für} \tcr{IP-Adressen} und 
\tcr{Hostname$\to$den Hostnamen} vorgenommen. Für den neuen Rechner wurde der 
Hostname \texttt{id.fg-networking.de} \tcr{gewäehlt$\to$gewählt}.

\subsubsection{Installation und Konfiguration des LDAP Servers}
Nach der Grundinstallation wurde im neuen System dann die Pakete \texttt{slapd} 
(benannt nach dem Namen des \tcr{OpenLDAP-Daemons}) , \texttt{ldap-utils} und 
\texttt{ldapscripts}\tcg{, sowie} deren noch nicht im System vorhandenen 
Paketabhängigkeiten installiert. Da LDAPS (LDAP over SSL) verwendet werden soll, 
damit die über LDAP abgefragten Daten verschlüsselt übertragen werden 
\tcr{können?}, mussten mit der Certificate Authority (CA) der Firma ein 
Zertifikat und ein Private Key für den Server erstellt und zusammen mit dem 
Stammzertifikat der CA auf den neuen Server kopiert werden.

\medskip Beim Entwurf der Verzeichnisstruktur ergaben sich verschiedene Probleme:

\noindent Um in einem \tcr{LDAP-Verzeichnis} neue Attribute zu \tcg{definieren, 
gibt} es sogenannte Schemas (eigener Ausdruck, nicht zu verwechseln mit 
Schema/Schemen\tcr{Schemata?}). CommuniGate verwendet zum Verwalten der 
\tcr{E-Mail-Anwenderdaten} ein eigenes proprietäres \tcr{LDAP-Schema}\tcg{, um} 
die Daten in \tcr{selbsdefinierten$\to$selbstdefinierten} Attributen zu 
speichern. Diese Attribute können somit nicht ins neue LDAP übernommen 
\tcg{werden, ohne} verschiedene Implementierungsentscheidungen für den neuen 
E-Mailserver zu kennen. Da dieser zur Bearbeitungszeit dieses Projektes noch 
nicht weit genug fortgeschritten \tcg{war, um} eine Absprache zu 
\tcg{ermöglichen, wurde} \tcg{entschieden, erst} einmal nur die wichtigsten 
Attribute zu übernehmen und die \tcr{E-Mail-Attribute} später anzupassen. Dies 
wird sehr wahrscheinlich auch wieder \tcr{mittels} per Skript 
\tcr{generierten$\to$generierter?} \tcr{Datensätzen$\to$Datensätze} möglich 
sein und stellt somit einen vertretbar geringen Mehraufwand für die 
\tcr{E-Mail-Server-Einrichtung} dar.

% Weiterhin musste die neue Verzeichnisstruktur \tcg{berücksichtigen, } \tcr{dass} 
% mehrere voneinander unabhängige Namensräume (die \tcr{3$\to$drei} in 
% \nameref{sec:Konfig-Doku-Alt} erwähnten E-Mail Domains) zu verwalten sind. Das 
% \tcg{bedeutet, } \tcr{dass} in den Domains Konten mit demselben 
% unqualifizierten \tcr{was meinst Du mit "`unqualifiziert"'?} Namen vorhanden 
% sein \tcg{können, die} aber voneinander unabhängig sind. Der erste Ansatz dazu 
% \tcg{war, } \tcr{jedem$\to$jeder} der \tcr{3$\to$drei} Domains eine eigene 
% Datenbank (im Sinne von getrennten \tcg{Dateien, die} von einem \tcr{LDAP-Server} 
% verwaltet werden) zu geben. Allerdings ließen sich in verschiedenen Versuchen 
% \tcr{die$\to$diese} Datenbänke zwar anlegen, \tcr{allerdings$\to$jedoch} war 
% ein Zugriff \tcr{auf sie$\to$darauf} nicht möglich. Getestet wurde dies mit 
% entsprechenden \tcr{\texttt{ldapsearch}-Aufrufen. Da sich 
% \tcr{unter Anderem$\to$u.a.} wegen den eher unpraktischen Logleveln 
% (siehe \autoref{sec:LDAP-Loglevel}) von OpenLDAP das genaue Problem mit dieser 
% Vorgehensweise nicht lokalisieren \tcr{lies$\to$ließ}\tcg{, musste} eine andere 
% Struktur ersonnen werden.

Der Ansatz für die alternative Verzeichnisstruktur basierte 
\tcr{unter Anderem$\to$u.a.} \tcr{darauf wie LDAP mit Domain Namen umgeht.
$\to$auf der Methode, mit der LDAP Domainnamen behandelt:} \tcr{Diese$\to$Sie} 
werden \tcr{von LDAP$\to$/dev/null} an den Trennzeichen (\tcr{Punkt$\to$hier: 
Punkte}) aufgespalten in einzelne Domain Components (LDAP Bezeichnung 
\tcr{'siehe'/'Abkürzung:'?}\texttt{dc}). Da alle \tcr{3$\to$drei} 
\tcr{Domain-Namen} mit \texttt{.de} \tcg{enden, konnte} \texttt{dc=de} als 
Wurzel \tcr{der des Baumes hierarchischen LDAP Verzeichnisstruktur$\to$des 
Baumes der hierarchischen LDAP-Verzeichnisstruktur?} verwendet werden. In der 
\tcr{1.$\to$ersten} Ebene unter der Wurzel verzweigt dieser Baum dann in die 
\tcr{3$\to$drei} Domain Components \texttt{fg-networking.de}, 
\texttt{schabler.de} und \texttt{worden.de}. Da Objektnamen von ihrem vollen 
Kontext (quasi ihrem Pfad im Baum) \tcg{abhängen, sind} auch in diesem Modell 
\tcr{3$\to$drei} unabhängige Namensräume gewährleistet. In den Domains 
\texttt{schabler.de} und \texttt{worden.de} sind nur relativ 
\tcr{wenig$\to$wenige} Benutzerkonten hinterlegt (weniger als 
\tcr{jeweils 10 s $\to$ 10 Stück pro Domain}) \tcr{und beide Domains haben auch 
keinen$\to$. Keine der beiden hat einen} eigenen Administrator (sie werden von 
fgn mit administriert). Somit gibt es auch keine Bedenken hinsichtlich 
Datenschutz und \tcg{Performance, wenn} sich alles in einer gemeinsamen 
Datenbank befindet. \tcr{Daher gibt es auch keine Bedenken bezüglich 
Datenschutz und Performance, wenn alle nutzerspezifischen Daten in einer 
zentralen Datenbank abgelegt werden.?} 
 
In der nächsten Ebene wurde dann jeweils eine Organizational Unit (\texttt{OU}) 
namens "people" \tcg{angelegt, in} die dann die Benutzerkonten eingeordnet 
\tcr{werden$\to$wurden?}. Die \tcr{"people"-\texttt{OU}} wurde \tcg{eingezogen, 
um} \tcr{neben Konten für Anwender} zukünftig auch andere \tcr{Konten als 
Anwenderkonten$\to$Typen von Konten, wie z.B. Rechnerkonten,}) in LDAP ablegen 
zu \tcg{können, ohne} die Verzeichnisstruktur grundlegend ändern zu müssen 
\tcr{um eine Unterscheidung der Konten zu erhalten$\to$/dev/null}.

Die oberste Ebene der Verzeichnisstruktur (die \tcr{\texttt{de}-Wurzel})
\tcg{, wurde} mit Hilfe des \tcr{Debian-Konfigurationsskripts} erstellt. Dabei 
wurden neben den \tcr{4$\to$vier} Schema\tcr{sicher, dass es keinen Plural 
gibt?} der \tcr{Debian-Standardkonfiguration} (\texttt{core}, \texttt{cosine}, 
\texttt{nis} und \texttt{inetorgperson}) zusätzlich das 
\tcr{\texttt{freeradius}-Schema} aus dem \tcr{\texttt{freeradius}-Paket} (siehe 
\nameref{sec:RADIUS-Konf}) und ein \tcr{\texttt{postfix}-Schema} (aus \tcr{den 
Galileo Press Praxisbuch$\to$dem Galileo-Press-Praxisbuch} zu OpenLDAP 2.4) 
verwendet (Erkläuterungen zu den Schemas \tcr{dieser Plural beispielsweise?} 
siehe \autoref{sec:LDAP-Schema}). Die \tcr{1.$\to$erste} Ebene wurde dann mit 
\texttt{slapadd} angelegt und die Grundstruktur darunter mit \texttt{ldapadd} 
erstellt (genaue Beschreibung der verwendeten LDAP Data Interchange Format 
\tcr{(LDIF)-Dateien} siehe \autoref{sec:Erstelle-DB}).

\subsubsection{Installation und Konfiguration des RADIUS Servers}\label{sec:RADIUS-Konf}
Auf dem Server wurden die Pakete \texttt{freeradius} und \texttt{freeradius-ldap} 
\tcr{und$\to$sowie - } zu Testzwecken \tcr{-} das Paket 
\texttt{freeradius-utils} installiert. In den Konfigurationsdateien wurde dann 
LDAP als mögliche Quelle für Benutzerdaten eingetragen, die \tcr{LDAP-Module} 
aktiviert und der Zugriff durch die \tcr{E-Mail-Server} der \tcr{Tu$\to$TU} 
Kaiserslautern erlaubt. Zusätzlich musste das zu dem \tcr{TU-E-Mail-Servern} 
gehörende Shared Secret eingetragen werden, welches aus den 
Konfigurationsdateien von CommuniGate übernommen \tcg{wurde, da} es nicht in 
der firmeninternen Dokumentation festgehalten war. Näheres zur Konfiguration 
siehe \autoref{sec:RADIUS-Konfig}

\subsubsection{Absicherung des Rechners (Firewall etc.)}
Wie in \nameref{sec:Sicherheitskonzept} \tcg{erläutert, sind} dem Server 
\tcr{schon 2 bereits existierende $\to$auf Grund seiner Position im Netzwerk 
bereits zwei} Firewalls vorgeschaltet. Die einzige \tcg{Änderung, die} an 
diesen vorgenommen werden \tcr{muss$\to$musste?}\tcg{, ist} das Eintragen einer 
Ausnahmeregel\tcr{um den Zugriff$\to$, mit der der Zugriff} auf den 
\tcr{RADIUS-Server} aus dem Universitätsnetz \tcr{zu erlauben$\to$erlaubt wird}. 
Dafür wird die bereits bestehende \tcr{Firewall-Regel} für den in CommuniGate 
\tcr{enthaltenem$\to$enthaltenen} RADIUS auf die IP des neuen Servers angepasst.

Zusätzlich zu den \tcg{Firewalls, die} bereits das Firmennetz der fgn GmbH 
\tcr{bereits$\to$/dev/null} \tcg{schützen, wurde} die simpel zu bedienende 
Uncomplicated Firewall (\texttt{ufw}) \tcg{verwendet, um} lediglich die für 
OpenLDAP, FreeRADIUS und SSH notwendigen Ports zuzulassen, und 
Netzwerkverbindungen auf den restlichen Ports nicht 
\tcr{zuzlassen$\to$zuzulassen}. Näheres zur \tcr{Firewall-Konfiguration} siehe 
\autoref{sec:Firewall-Konfig}.

Der \tcr{OpenLDAP-Server} wurde so \tcg{konfiguriert, } \tcr{dass} es lediglich 
\tcr{SSL-verschlüsselte} Verbindungen zulässt, somit ist ein Abhören der 
Kommunikation mit dem \tcr{LDAP-Server} erschwert. Modifikation 
\tcr{des$\to$der} Daten im Verzeichnisdienst ist nur dem 
\tcr{Administrator-LDAP-Benutzer} gestattet, der mit einem \tcr{-} den 
Passwortrichtlinien von fgn \tcr{entsprechendem$\to$entsprechenden -} Passwort 
gesichert ist. Die Firewall auf dem Server lässt außerdem nur Verbindungen aus 
dem Firmennetz \tcr{und nur} auf dem entsprechenden Port (636) zu. Als anonymer 
Benutzer ist lediglich der Zugriff auf \tcg{Attribute, die} nur anderweitig 
öffentlich verfügbare \tcg{Informationen, wie} \tcr{den Namen} des 
\tcg{Anwenders, enthalten} \tcr{und$\to$sowie} das Testen eines Passworts gegen 
den gespeicherten \tcr{Passwort-Hash}\tcr{möglich}. \tcr{Letzteres$\to$Der 
letztgenannte Vorgang} ist ein ernstzunehmender \tcg{Angriffspunkt, der} bisher 
nur durch die Verbindungseinschränkungen durch die Firewall abgefedert wird und 
auf jeden Fall noch \tcr{versucht werden muss anderweitig abzusichern$\to$auf 
zuverlässigere Art und Weise abgesichert werden muss}.\tcr{Dies wird jedoch 
nicht mehr im Rahmen des Projektes geschehen.?}

\tcr{Wo endete oben eigentlich Deine Aufzählung der Probleme mit der 
Verzeichnisstruktur? Fiel mir nur gerade wieder ein...}

Zum \tcr{FreeRADIUS-Server} können sich generell nur Clients \tcg{verbinden, 
die} vorher in die Konfiguration eingetragen \tcg{wurden, da} beide Seiten das 
Shared Secret einstellen müssen. Verbindungen von komplett unbekannten Clients 
können somit schon rein prinzipbedingt nicht aufgebaut werden. Der einzige 
Angriffsvektor in \tcr{diese$\to$dieser} Richtung ist also ein \tcg{Client, 
der} das Shared Secret zwischen dem FreeRADIUS und den \tcr{TU-E-Mail-Servern} 
kennt. Leider ist das Shared Secret die große Schwachstelle des 
\tcr{RADIUS-Protokolls} (was auch seit ungefähr 15 Jahren hinlänglich bekannt 
ist), somit ist das \tcr{einzige$\to$Einzige}\tcg{, was} 
\tcr{gegen$\to$/dev/null} dagegen \tcg{hilft, die} Verbindungsbeschränkungen 
durch die \tcr{fgn-Firewall}. Die einzige Möglichkeit eines Zugriffs wäre also 
\tcr{die das herausfinden$\to$durch das Herausfinden} des Shared Secrets und 
das Spoofing der IP eines \tcr{E-Mail-Servers} \tcr{der TU Kaiserslautern}
\tcr{gegeben}. Beides zusammen genommen ist \tcr{jedoch} recht unwahrscheinlich.
\tcr{und erschwerend hinzu kommt noch das$\to$Erschwerend kommt hinzu, dass} 
die TU selbst Maßnahmen gegen IP-Spoofing in ihrem Netz (131.246.0.0/16) betreibt.

Der \tcr{OpenSSH-Server}\tcg{, der} \tcr{außerdem (neben was? ggf. Wort 
löschen)} auf dem Rechner für Fernzugriffe \tcg{läuft, wird} mit der 
\tcr{Debian-Standardkonfiguration} \tcg{betrieben, unter} \tcr{denen$\to$der?} 
Zugriff auf den \tcr{root-Account} nur mit einem eingetragenen \tcr{SSH-Key} 
möglich ist. Es existieren auch keine normalen Benutzer auf \tcr{den$\to$dem} 
\tcg{Rechner, weswegen} das Angriffsszenario vom Kapern eines Benutzeraccounts 
und dem \tcr{darauffolgenden} Erlangen von \tcr{root-Rechten} durch eine 
Schwachstelle (Priviledge Escalation) ausfällt. Wie bei sämtlichen 
\tcr{fgn-Servern}\tcg{, wurde} der SSH Zugang aus dem gesamten Internet 
erlaubt. \tcr{Hier werden aber Brute Force Attacken$\to$Brute-Force-Attacken 
werden} durch die \tcr{fgn-Firewall} \tcg{erschwert, da} diese für 
\tcr{SSH-Verbindungsversuche} einen Rate Limiter eingestellt hat.

\subsection{Abschließende Arbeiten}
\subsubsection{Umkonfiguration der Webservices}
Da die umzustellenden Dienste (Webservices und OpenVPN) bereits den alten 
\tcr{LDAP-Server} \tcg{verwenden, musste} hier nicht viel eingerichtet werden. 
Die größte \tcg{Änderung, die} vorzunehmen \tcg{war, ist} daher das zusätzliche 
Eintragen des \tcr{CA-Stammzertifikates} in die Konfigurationsdatei 
\texttt{/etc/apache2/mods-available/ldap.conf} (bzw. bei OpenVPN in die 
entsprechende Konfigurationsdatei) \tcg{, da} sonst keine \tcr{LDAP-Verbindung} 
über SSL aufgebaut werden kann. Ansonsten \tcr{muss$\to$mussten?} nur noch in 
der Site-Konfiguration die URL des \tcr{LDAP-Servers} und die BaseDN angepasst 
werden (OpenVPN ist analog anzupassen). Zu Details der anzupassenden 
Konfigurationen siehe \autoref{sec:Apache-Konfig} und \autoref{sec:VPN-Konfig}.

Wie in \nameref{sec:Konfig-Doku-Alt} \tcg{geschildert, konnte} Egroupware zur 
Bearbeitungszeit dieses Projektes noch nicht umgestellt werden, dies muss dann 
noch vom zuständigen Mitarbeiter vorgenommen werden.

\subsubsection{Import/Anlegen der \tcr{Anwender-Konten}}
Wie \tcr{vorher geschildert (in \nameref{sec:Importsuche})$\to$in 
\nameref{sec:Importsuche} erwähnt} \tcg{, wurde} entschieden, \tcr{dass} die 
praktikabelste \tcg{Methode, um} die Benutzerkonten im neuen Verzeichnisdienst 
zu erstellen, eine Generierung einer von LDAP importierbaren \tcr{???}
Mit Hilfe der von Herrn Auerswald erstellten Skripte (Skripte siehe 
\autoref{sec:SkriptA} und \autoref{sec:SkriptB}) wurde dann aus den Daten von 
CommuniGate eine \tcr{LDIF-Datei} \tcg{erzeugt, die} dann mit \texttt{ldapadd} 
importiert wurde.

Anmerkung: CommuniGate speichert Mailinglisten und Ähnliches ebenfalls in 
\tcr{LDAP-Objekten}, es wurden hier nur die Konten von echten Anwendern und zur 
Verwaltung der Webservices etc. nötige Konten importiert.

\subsubsection{\tcr{Funkions-$\to$Funktions-} und Sicherheitstests}
Schon während der Arbeiten wurden immer wieder prüfbare Teilkomponenten getestet: 
Vor dem Erstellen der Verzeichnisstruktur wurde bereits eine Teststruktur 
\tcr{erstellt?} \tcg{, die} lediglich die \tcr{\texttt{fg-networking}-Domain} 
\tcr{enthielt$\to$/dev/null} und ein Testkonto \tcr{enthielt}. Dieses Testkonto 
wurde dann mit Hilfe von \texttt{ldapsearch} aus den \tcr{LDAP-Utils} abgefragt, 
\tcr{zuerst$\to$/dev/null} noch bevor LDAPS eingerichtet war. Nach der 
Einrichtung von LDAPS musste das Stammzertifikat in die Konfigurationsdatei 
\texttt{/etc/ldap/ldap.conf} eingetragen werden. \tcr{dies$\to$Dies} ist 
\tcr{aber$\to$jedoch} für den Normalbetrieb nicht \tcg{notwendig, da} dieser 
Eintrag nur von den \tcr{LDAP-Utils} ausgewertet wird. Apache und OpenVPN haben 
dazu eigene Einstellungen. \tcr{dann versteh ich nach Lesen des Abschnitts 
nicht, weshalb Du das Stammzertifikat überhaupt eintragen musstest?}

Die Webservices auf \texttt{nms} wurden dann kurzzeitig auf dieses LDAP 
\tcg{umgestellt, um} die generelle Funktion der Authentifizierung testen zu 
können. Der Rechner \texttt{nms} wurde \tcg{gewählt, da} die Webservices 
\tcr{auf ihm$\to$darauf} wenig benutzt werden, und das auch nur von 
\tcg{Mitarbeitern, die} zur Bearbeitungszeit vor Ort waren und so einfach und 
schnell über die Tests informiert werden konnten. Hierbei \tcr{viel$\to$fiel 
dann} \tcg{auf, } \tcr{dass} es anfänglich noch einen Fehler in den 
\tcr{SSL-Einstellungen} gab. \tcr{dieser stellte sich aber lediglich als 
Tippfehler in der Apache Konfiguration heraus.$\to$Dieser konnte nach kurzer 
Suche schnell als Tippfehler in der Apache-Konfiguration identifiziert und 
behoben werden.} \tcr{Nach Korrektur dessen$\to$Danach} konnte man sich mit 
dem Testkonto erfolgreich an den Webservices anmelden.

Die Anbindung des \tcr{FreeRADIUS-Servers} an den \tcr{OpenLDAP-Server} wurden 
mit Hilfe des Programms \texttt{radtest} aus den \tcr{FreeRADIUS-Utils} 
getestet. Hier wurde ebenfalls das oben erwähnte Testkonto abgefragt zusammen 
mit dem Shared Secret. \tcr{Hier wurde - unter Verwendung des Shared Secret - 
ebenfalls das oben erwähnte Testkonto abgefragt.?} Mit demselben Befehl 
\tcr{($\to$,} aber einem echten Anwenderkonto \tcr{)$\to$,} wurde auch der 
\tcr{CommuniGate-Server} \tcg{getestet, um} einen Vergleichswert zu haben, da 
ein echter Praxistest erst bei Fertigstellung des neuen \tcr{E-Mail-Servers} 
durchgeführt werden kann.

Abschließend wurde noch \tcr{getestet$\to$versucht, } einen Managed Switch an 
den \tcr{RADIUS-Server} \tcg{anzubinden, damit} darüber die Benutzerkonten aus 
dem LDAP zum \tcr{anmelden$\to$Anmelden} verwendet werden können. Nach 
Informationen aus den \tcr{LDAP-Logs} funktionierte das 
\tcr{authentifizieren$\to$Authentifizieren} am Server auch, allerdings konnte 
man sich trotzdem nicht erfolgreich am Switch anmelden. Dafür hätte man 
zusätzliche Berechtigungen im LDAP hinterlegen müssen. Da nicht sofort 
ersichtlich \tcg{wurde, was} genau dafür zu konfigurieren \tcg{wäre, und} die 
Anbindung von Switches erst irgendwann in Zukunft geschehen soll, wurde das 
Ganze dann nicht weiter verfolgt.

\medskip \noindent
\tcr{An Sicherheitstests$\to$Zum Testen der Sicherheit} wurde nach der 
Umstellung der Webservices erfolglos \tcg{versucht, sich} an diesen mit 
Anmeldeinformationen (falscher Benutzername, richtiger Benutzername mit 
falschem Passwort) anzumelden. Dasselbe wurde mit \texttt{radtest} am 
\tcr{RADIUS-Server} getestet. Abschließend wurde noch per \texttt{nmap} von 
innerhalb des Firmennetzes und von einem externen Rechner aus \tcg{geprüft, } 
\tcr{ob} ausschließlich die richtigen Ports geöffnet sind (Näheres dazu in 
\autoref{sec:NMAP-Test-int} und \autoref{sec:NMAP-Test-ext}).

\section{Projektkosten}
\hilight{TODO}

\section{Projektabschluss}
\subsection{Fazit}
Sämtliche Systeme von \tcg{fgn, die} vorher den \tcr{LDAP-Server} von 
CommuniGate \tcg{benutzten, wurden} erfolgreich auf den neuen 
\tcr{OpenLDAP-Server} umgestellt. Da der alte \tcr{E-Mail-Server} weiterhin in 
Betrieb \tcg{bleibt, bis} er \tcr{vom neuen$\to$von einem neuen} Server 
abgelöst \tcg{wird, findet} bis dahin quasi ein Parallelbetrieb der 
\tcr{LDAP-Server} statt. \tcr{Deswegen$\to$Daher} werden die Benutzerpasswörter 
bei Fertigstellung des neuen \tcr{E-Mail-Servers} noch einmal zur Sicherheit 
synchronisiert werden müssen\tcr{,$\to$;} \tcr{allerdings wird die$\to$dies 
wird jedoch} über Skripte wieder einfach zu bewerkstelligen sein. Die 
\tcr{bereits vorhandenen/im Rahmen des Projekts erstellten?} Skripte werden 
\tcr{dazu} allerdings leicht modifiziert werden \tcg{müssen, da} \tcr{diesmal 
LDAP Daten verändert werden sollen statt neu hinzugefügt$\to$dann Daten 
innerhalb des LDAP verändert, statt (wie im Rahmen des Projekts erfolgt) neu 
hinzugefügt werden sollen} und die \tcr{LDIF-Syntax} dafür leicht anders ist.

Der \tcr{FreeRADIUS-Server} wurde ebenso erfolgreich eingerichtet und \tcr{-} 
einfacher als erwartet \tcr{-} an den \tcr{OpenLDAP-Server} angebunden. 
Allerdings verrichtet er bisher noch keinen sinnvollen \tcg{Dienst, da} bei fgn 
RADIUS momentan nur beim \tcr{E-Mail-Server} benutzt wird und dieser 
\tcr{ja$\to$bislang} noch selbst einen \tcr{RADIUS-Server} betreibt. Dies wird 
sich voraussichtlich erst mit der Fertigstellung des neuen \tcr{E-Mail-Servers} 
ändern.

Bei der Zeitplanung des Projektes wäre im Nachhinein 
\tcr{folgendes$\to$Folgendes} zu verbessern gewesen:\tcr{Absatz!}
Für den Entwurf und die Implementierung der Verzeichnisstruktur hätte mehr Zeit 
eingeplant werden müssen. Als der Zeitplan für den Projektantrag ausgearbeitet 
\tcg{wurde, waren} die Vorrecherchen zu LDAP und vor \tcr{Allem$\to$allem} zur 
bestehenden Konfiguration noch nicht weit genug \tcg{fortgeschritten, } 
\tcr{als$\to$/dev/null)} \tcr{dass} Probleme aus dieser Richtung vorhersehbar 
gewesen wären. Insgesamt wäre der Entwurf der \tcr{LDAP-Verzeichnisstruktur} am 
Besten als zusätzlicher Punkt zur Planung hinzugefügt worden. In der 
Umsetzungsphase \tcr{hätte man dagegen?} weniger Zeit für die Vorbereitung 
veranschlagen können, da nur eine übersichtlich kleine Anzahl an Diensten den 
\tcr{LDAP-Server} verwendet und bei diesen nicht viel an der Konfiguration zu 
ändern war.

\subsection{Ausblick}
Die \tcr{erste zukünftige$\to$voraussichtlich erste} Änderung \tcr{für die in 
diesem Projekt$\to$an den im Rahmen dieses Projektes} erstellten Serverdienste 
wird nach aktueller Planung die Anbindung des neuen \tcr{E-Mail-Servers} sein. 
Dafür werden den in OpenLDAP bestehenden Benutzerkonten die \tcg{Attribute, die} 
der \tcr{E-Mail-Server} benötigt (wie z.B. die \tcr{E-Mail-Adresse}) 
\tcr{befüllt$\to$angepasst?} werden müssen. Dies wird voraussichtlich wieder 
über durch Skripte erstellte \tcr{LDIF-Dateien} geschehen.

Ebenso bietet sich für eine Erweiterung des Projektsystems die Einbindung der 
anderen \tcr{Linux-Rechner} in der Firma an. Dort könnte man dadurch die 
bisherigen lokalen Benutzerkonten durch \tcr{ein$\to$/dev/null} zentral in LDAP 
\tcr{abgelegtes und verwaltetes Konto$\to$abgelegte und verwaltete Konten} 
ersetzen. Dazu müsste man auf den \tcr{Linux-Rechnern} LDAP an das 
\tcr{PAM-System} anbinden und im \tcr{LDAP-Server} die Benutzer um die 
passenden Attribute (wie z.B. Home-Verzeichnis, UID, Gruppen) erweitern. 
Letzteres kann wieder geskriptet werden. Als Anwendersystem wird im Moment 
allerdings nur \tcr{1$\to$ein!} Rechner \tcr{verwendet (\texttt{fgnfs})$\to$
, und zwar \texttt{fgnfs}, verwendet} \tcg{, für} den der Aufwand im Rahmen 
des Projektes \tcr{zu groß schien für das zu erwartende Ergebnis$\to$im 
Vergleich zum erwarteten Nutzen jedoch zu groß erschien}. \tcr{Zur Anbindung} 
an das PAM-System müsste auf den Rechnern vermutlich ein passendes 
PAM-LDAP-Modul sowie eventuell zusätzlich das LDAP-Modul für den Name Switch 
Service (\texttt{NSS}) installiert werden.

Eine weitere für fgn attraktive Erweiterungsmöglichkeit ist das Anbinden der 
Managed Switches \tcr{des$\to$der} Schulungslabore an den 
\tcr{FreeRADIUS-Server} (und somit \tcr{an} LDAP). Dies würde Probleme durch 
Änderungen der Anmeldeinformationen der \tcr{Switche$\to$Switches} durch 
Schulungsteilnehmer vorbeugen die abundzu auftreten. \tcr{Dies würde Problemen
vorbeugen, die ab und zu bei Änderungen der Anmeldeinformationen der Switches 
durch die Schulungsteilnehmer auftreten.?} Da die Schulungslabore \tcr{in einem 
VPN sind$\to$sich in einem VPN befinden} müsste RADIUS dazu ebenfalls \tcr{ins 
VPN gehängt werden$\to$in das VPN aufgenommen werden}. \tcr{Desweiteren} 
müssten Konten für die Switches erstellt \tcg{werden, } \tcr{mit den passenden 
Attributen die zu Autorisierung benötigt werden$\to$die alle Attribute, die zur 
Autorisierung benötigt werden, enthalten}. In die Switche müsste außerdem der 
RADIUS Server eingetragen werden und im RADIUS Server die Clients eingetragen 
werden (jeweils dann mit passenden Shared Secret "Paaren").\tcr{Außerdem müsste
in die Konfigurationen der Switches der RADIUS-SERVER sowie in die Konfiguration
des Servers die Clients - jeweils mit passenden Shared Secret-Paaren - eingetragen 
werden.}

fgn hat neben den diversen \tcr{Linux-Rechnern} auch noch \tcr{2$\to$zwei} 
\tcr{Windows-PCs} in der \tcg{Verwaltung, sowie} einige virtuelle 
\tcr{Windows-Rechner} \tcr{an$\to$in} den Schulungslaboren angeschlossen. Für 
diese könnte man mit Samba 4 eine kleine \tcr{Active Directory-Domäne} 
\tcg{aufbauen, die} an den \tcr{OpenLDAP-Server} angebunden wird. Allerdings 
benötigt Samba 4 inzwischen ein eigenes LDAP (\tcr{dies wird für 
Replikationsfunktionen benötigt$\to$für Replikationsfunktionen}) und lässt sich 
somit nur umständlich an das nun bestehende LDAP anbinden. Dieser 
Erweiterungsmöglichkeit stellt insgesamt von den aufgezählten für fgn die 
unattraktivste dar, da es nur wenige Windowsrechner gibt die davon tatsächlich 
profitieren würden und der Aufwand dafür recht hoch wäre. \tcr{Diese 
Erweiterungsmöglich stellt - von den bisher genannten - die für fgn 
unattraktivste dar, da der Aufwand dafür recht hoch wäre, während nur wenige 
Windowsrechner von der Umstellung tatsächlich profitieren würden.}


\appendix
\newpage
\addcontentsline{toc}{section}{Anhänge} %fügt Überschrift Anhänge in die TOC ein
\section{Übernommene LDAP Attribute} \label{sec:LDAP-Attribute}
\newpage
\section{Erklärung der OpenLDAP Loglevel} \label{sec:LDAP-Loglevel}
\newpage
\section{Erläuterungen zu den verwendeten LDAP Schemas}\label{sec:LDAP-Schema}
\newpage
\section{Erstellung der Verzeichnisstruktur}\label{sec:Erstelle-DB}
\newpage
\section{Skript zum Auslesen der Anwenderkonten aus der CommuniGate Dateien}\label{sec:SkriptA}
\newpage
\section{Skript zum Hashen der Anwenderpasswörter}\label{sec:SkriptB}
\newpage
\section{Konfiguration des FreeRADIUS Servers}\label{sec:RADIUS-Konfig}
\newpage
\section{Konfiguration der Firewall}\label{sec:Firewall-Konfig}
\newpage
\section{Änderungen an der Apache Konfiguration}\label{sec:Apache-Konfig}
\newpage
\section{Änderungen an der OpenVPN Konfiguration}\label{sec:VPN-Konfig}
\newpage
\section{Ergebnis eines fgn internen NMAP Tests}\label{sec:NMAP-Test-int}
\newpage
\section{Ergebnis eines externen NMAP Tests}\label{sec:NMAP-Test-ext}

%\section{Testkapitel}
%\subsection{Testsektion}
%Dies ist eine Testaufzählung:
%\begin{itemize}
%\item Item 1
%\item Item 2
%\item Item 3
%\item Item 4
%\end{itemize}
%Und weiter im Text.
%
%Hier bricht die Seite nun um.
%
%\Blinddocument
%\newpage


\end{document}
