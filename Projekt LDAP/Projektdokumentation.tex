% !TEX TS-program = pdflatex
% !TEX encoding = UTF-8 Unicode

% This is a simple template for a LaTeX document using the "article" class.
% See "book", "report", "letter" for other types of document.

\documentclass[11pt,a4paper,titlepage]{scrartcl} % use larger type; default would be 10pt

\usepackage[utf8x]{inputenc} % set input encoding (not needed with XeLaTeX)

%%% Examples of Article customizations
% These packages are optional, depending whether you want the features they provide.
% See the LaTeX Companion or other references for full information.

%%% PAGE DIMENSIONS
\usepackage{geometry} % to change the page dimensions
%\geometry{a4paper} % or letterpaper (US) or a5paper or....
\geometry{bottom=3.5cm} % for example, change the margins to 2 inches all round
% \geometry{landscape} % set up the page for landscape
%   read geometry.pdf for detailed page layout information

\usepackage{graphicx} % support the \begin{center}\includegraphics command and options
\usepackage{float}
\usepackage{tocstyle}

% \usepackage[parfill]{parskip} % Activate to begin paragraphs with an empty line rather than an indent

%%% PACKAGES
%\usepackage{booktabs} % for much better looking tables
%\usepackage{array} % for better arrays (eg matrices) in maths
\usepackage{paralist} % very flexible & customisable lists (eg. enumerate/itemize, etc.)
\usepackage{verbatim} % adds environment for commenting out blocks of text & for better verbatim
%\usepackage{subfig} % make it possible to include more than one captioned figure/table in a single float
% These packages are all incorporated in the memoir class to one degree or another...
\usepackage[ngerman]{babel}
\usepackage{blindtext}
%\usepackage{pifont} %for symbols (i.e. arrows)
%\usepackage{showframe} %shows the margins

\usepackage[colorlinks]{hyperref} % package for hyperlinks with \url
%\usepackage[svgnames]{xcolor}
%\usepackage[anythingbreaks]{breakurl}
%\usepackage{listings}

\newcommand{\hilight}[1]{\colorbox{yellow}{#1}} %command for magic marker highlighting
%   (from http://pleasemakeanote.blogspot.de/2009/08/how-to-highlight-text-in-latex.html)

%%redifine of emph, see http://tex.stackexchange.com/questions/6754/what-is-the-canonical-way-to-redefine-the-emph-command
\makeatletter
\DeclareRobustCommand{\em}{%
  \@nomath\em \if b\expandafter\@car\f@series\@nil
  \normalfont \else \bfseries \fi}
\makeatother

\setlength{\parindent}{0mm} %set paragraph begin indentation to 0

% hyperlink color definitions
%\hypersetup{citecolor=DeepPink4}
%\hypersetup{linkcolor=DarkRed}
%\hypersetup{urlcolor=DarkBlue} 

%%% SECTION TITLE APPEARANCE
%\usepackage{sectsty}
%\allsectionsfont{\sffamily\mdseries\upshape} % (See the fntguide.pdf for font help)
% (This matches ConTeXt defaults)

%%% ToC (table of contents) APPEARANCE
%\usepackage[nottoc,notlof,notlot]{tocbibind} % Put the bibliography in the ToC
%\usepackage[titles,subfigure]{tocloft} % Alter the style of the Table of Contents
%\renewcommand{\cftsecfont}{\rmfamily\mdseries\upshape}
%\renewcommand{\cftsecpagefont}{\rmfamily\mdseries\upshape} % No bold!

%\usepackage{uarial}
\usepackage{helvet}
\renewcommand{\familydefault}{\sfdefault}

%%% END Article customizations

%%% The "real" document content comes below...

\title{Projektdokumentation "Aufsetzen eines Authentifizierungsservers als Ersatz eines veralteten propriet\"aren CommuniGate Servers"}
\author{Sebastian Deu\ss{}er}
%\date{4. April 2014} % Activate to display a given date or no date (if empty),
         % otherwise the current date is printed 
%\setcounter{section}{-1} % sets the section counter to start with 0

\begin{document}
\maketitle %title (page)


\tableofcontents
\newpage

\section{Einleitung}
In den meisten Firmennetzwerken wird eine zentrale Stelle benötigt um verschiedene Anwender Metadaten zu verwalten. F\"ur diese Aufgabe wird \"ublicherweise ein Verzeichnisdienst verwendet. Basis eines Verzeichnisdienstes ist eine hierarchische Datenbank, die im Netzwerk verteilt angelegt sein kann. In dieser Datenbank k\"onnen Daten zu verschiedenen Objekten abgelegt werden wie etwa Konfigurationsdaten f\"ur Rechner oder diverse Daten f\"ur Anwender wie z.B. Name, Passwort, E-Mail Konto, Gruppenzugeh\"origkeiten u.s.w.. Die Anforderungen an einen Verzeichnisdienst wurden in den 80er Jahren von der International Telecommunication Union in der X.500 Spezifikation festgeschrieben nachdem weltweit Telekommunikationsunternehmen jahrzehntelang Praxiserfahrungen mit dem Thema bei Erstellung und Verwaltung von Telefon Verzeichnissen sammelten und dann zur Spezifikation beisteuerten.

In der Praxis stellte es sich als unpraktikabel f\"ur so gut wie alle Firmen au\ss{}er große Telekommunikationsunternehmen heraus die gesamte X.500 Spezifikation zu implementieren da dies einen hohen Implementierungsaufwand darstellt und der Betrieb sehr Hardware intensiv ist. Deswegen wurde an der Universität von Michigan 1993 LDAP (Lightweight Directory Access Protocol) entwickelt das ursprünglich als abgespeckte Alternative zu DAP (Directory Access Protocol) das traditionell zum Zugriff auf X.500 Verzeichnisse verwendet wurde. Der Ansatz von LDAP war den Zugriff durch den TCP/IP Protokollstapel zur ermöglichen, was weit weniger Implementierungsaufwand darstellt als die komplette Implementierung alles OSI-Schichten wie es bei DAP der Fall ist. Heutzutage ist LDAP das verbreiteste Protokoll zur Abfrage von Verzeichnisdiensten, u.a. auch da TCP/IP-basierte Netzwerke der verbreitesten Netzwerktyp ist. LDAP wird in vielen Anwendungen eingesetzt, vor allem im Adressbuchteil der meisten E-Mail Clients wie z.B. Apple Adressbuch, Microsoft Outlook, Mozilla Thunderbird und in Verzeichnisdiensten wie z.B. Microsoft Active Directory Services, Apple Open Directory und dem aussterbenden Pionier Novell eDirectory. Wie viele Protokolle ist LDAP Client/Server basierend.

In diesem Projekt wurde OpenLDAP verwendet, die verbreiteste freie Open Source Implementierung des Netzwerkstandards und steht unter einen eigenen freien Lizenz, der OpenLDAP Public License. OpenLDAP ist au\ss{}erdem die Referenzimplementierung des Standards, weswegen es in vielen Belangen (z.B. bei Schemadateien) mehr auf Protokollkonformit\"at als andere Implementierungen. Der OpenLDAP Server ist nicht nur das Abfrageprotokoll enthalten sondern auch ein Verzeichnisdienst. Es ist somit eine kosteng\"unstige L\"osung zum Aufbau eines Verzeichnisdienstes f\"ur viele Anwendungsf\"alle.

RADIUS (Remote Authentication Dial-In User Service) ist ebenfalls ein Client/Server basierendes Protokoll zur Authentifizierung, Autorisierung und Accounting (AAA-System) von Benutzern bei Einwahlverbindungen zu Netzwerken. RADIUS stellt den De-facto Standard zur zentralen Authentifizierung von Einwahlverbindungen wie z.B. über ISDN, DSL und WLAN (\"uber IEEE 802.1X) dar. \"Ublicherweise ist ein RADIUS-Server an einen Verzeichnisdienst angebunden um von ihm die Benutzerdaten f\"ur die Authentifizierung und Autorisierung (und teilweise auch Accounting) abzufragen.

Hier wurde FreeRADIUS verwendet, der laut Aussage des Projektes weltweit verbreiteste RADIUS Server. Das freie Open Source Projekt unter GPLv2 umfasst neben dem RADIUS Server au\ss{}erdem eine PAM Bibliothek, ein Apache Webserver Modul und eine Client Bibliothek (im Gegensatz zum Rest unter BSD Lizenz).

\section{Analyse und Planung}
\subsection{Ist-Analyse}
Im Praktikumsbetrieb fgn GmbH l\"auft der E-Mailverkehr und die Authentifizierung an den internen Webservices und OpenVPN Server \"uber einen alten CommuniGate Server (v5.0.13 von November 2006). Urspr\"unglich wurde dieser auf einem eigenen Rechner aufgesetzt, inzwischen aber wie viele andere Rechner der Firma virtualisiert. Die Webservices, die ihn zur Anwender-Authentifizierung verwenden, laufen auf drei anderen VMs, ebenso der OpenVPN Server. Lediglich die Anwenderkonten auf den Betriebssystemen der diversen Anwender-PCs sind nicht von CommuniGate abh\"angig. Da die fgn GmbH \"uber das Netz der Technischen Universit\"at Kaiserslautern angebunden ist m\"ussen alle E-Mails auch \"uber die Mailserver der Universit\"at laufen. Diese verwenden RADIUS um zu pr\"ufen ob die E-Mailkonten der fg-networking.de Domain tats\"achlich vorhanden sind. Der RADIUS Server dazu wird ebenso von CommuniGate bereitgestellt.


\subsection{Soll-Analyse}
\hilight{TODO} Der CommuniGate Server der fgn GmbH f\"ur E-Mail und Identity Management (\mbox{LDAP} und \mbox{RADIUS}) wurde seit langem nicht aktualisiert und ist deswegen eine sehr in die Jahre gekommene Version. Schon seit l\"angerem gibt es deswegen regelm\"a\ss{}ig Probleme, zB mit der SSL Authentifizierung neuerer E-Mail Clients. Da eine Aktualisierung von CommuniGate mit \"ahnlichem Aufwand verbunden w\"are wie ein komplettes Neuaufsetzen von Ersatzservern (und au\ss{}erdem mit dem Kauf einer neuen Lizenz verbunden w\"are) wurde entschieden die Serverkomponenten E-Mail und Identity Management durch neue Server abzul\"osen.
\hilight{TODO} Da die E-Mail Infrastruktur sehr kritisch f\"ur die Arbeit der Firma ist und bei der Migration der E-Mailkonten mit vielen vertraulichen firmeninternen Informationen hantiert werden muss soll die Installation des neuen E-Mail Servers von einem Mitarbeiter von fgn durchgef\"uhrt werden.


%\section{Testkapitel}
%\subsection{Testsektion}
%Dies ist eine Testaufz\"ahlung:
%\begin{itemize}
%\item Item 1
%\item Item 2
%\item Item 3
%\item Item 4
%\end{itemize}
%Und weiter im Text.
%
%Hier bricht die Seite nun um.
%
%\Blinddocument
%\newpage


\end{document}
