% !TEX TS-program = pdflatex
% !TEX encoding = UTF-8 Unicode

% This is a simple template for a LaTeX document using the "article" class.
% See "book", "report", "letter" for other types of document.

\documentclass[11pt,a4paper,titlepage=firstiscover]{scrartcl} % use larger type; default would be 10pt

\usepackage[utf8x]{inputenc} % set input encoding (not needed with XeLaTeX)

%%% Examples of Article customizations
% These packages are optional, depending whether you want the features they provide.
% See the LaTeX Companion or other references for full information.

%%% PAGE DIMENSIONS
\usepackage{geometry} % to change the page dimensions
%\geometry{a4paper} % or letterpaper (US) or a5paper or....
\geometry{top=3.5cm,bottom=3.5cm} % for example, change the margins to 2 inches all round
% \geometry{landscape} % set up the page for landscape
%   read geometry.pdf for detailed page layout information

\usepackage{graphicx} % support the \begin{center}\includegraphics command and options
\usepackage{float}
\usepackage{tocstyle}

% \usepackage[parfill]{parskip} % Activate to begin paragraphs with an empty line rather than an indent

%%% PACKAGES
%\usepackage{booktabs} % for much better looking tables
%\usepackage{array} % for better arrays (eg matrices) in maths
\usepackage{paralist} % very flexible & customisable lists (eg. enumerate/itemize, etc.)
\usepackage{verbatim} % adds environment for commenting out blocks of text & for better verbatim
%\usepackage{subfig} % make it possible to include more than one captioned figure/table in a single float
% These packages are all incorporated in the memoir class to one degree or another...
\usepackage[ngerman]{babel}
\usepackage{blindtext}
%\usepackage{pifont} %for symbols (i.e. arrows)
%\usepackage{showframe} %shows the margins

\usepackage[colorlinks,linkcolor=blue]{hyperref} % package for hyperlinks with \url
%\usepackage[svgnames]{xcolor}
%\usepackage[anythingbreaks]{breakurl}
%\usepackage{listings}

\newcommand{\hilight}[1]{\colorbox{yellow}{#1}} %command for magic marker highlighting
%   (from http://pleasemakeanote.blogspot.de/2009/08/how-to-highlight-text-in-latex.html)

%%redifine of emph, see http://tex.stackexchange.com/questions/6754/what-is-the-canonical-way-to-redefine-the-emph-command
\makeatletter
\DeclareRobustCommand{\em}{%
  \@nomath\em \if b\expandafter\@car\f@series\@nil
  \normalfont \else \bfseries \fi}
\makeatother

%%% HEADERS & FOOTERS
%set with fancy layout package
\usepackage{fancyhdr} % This should be set AFTER setting up the page geometry
\pagestyle{fancy} % options: empty , plain , fancy
\renewcommand{\headrulewidth}{0pt} % customise the layout...
\lhead{}\chead{}\rhead{}
\lfoot{}\cfoot{\thepage}\rfoot{}

%\setlength{\parindent}{0mm} %set paragraph begin indentation to 0

% hyperlink color definitions
%\hypersetup{citecolor=DeepPink4}
%\hypersetup{linkcolor=DarkRed}
%\hypersetup{urlcolor=DarkBlue} 

%%% SECTION TITLE APPEARANCE
%\usepackage{sectsty}
%\allsectionsfont{\sffamily\mdseries\upshape} % (See the fntguide.pdf for font help)
% (This matches ConTeXt defaults)

%%% ToC (table of contents) APPEARANCE
%\usepackage[nottoc,notlof,notlot]{tocbibind} % Put the bibliography in the ToC
%\usepackage[titles,subfigure]{tocloft} % Alter the style of the Table of Contents
%\renewcommand{\cftsecfont}{\rmfamily\mdseries\upshape}
%\renewcommand{\cftsecpagefont}{\rmfamily\mdseries\upshape} % No bold!

%\usepackage{uarial}
\usepackage{helvet}
\renewcommand{\familydefault}{\sfdefault}

%%% END Article customizations

%%% The "real" document content comes below...

\titlehead{taylorix institut f\"ur berufliche Bildung e.V.}
\title{Projektdokumentation "Aufsetzen eines Authentifizierungsservers als Ersatz eines veralteten propriet\"aren CommuniGate Servers"}
\author{Sebastian Deu\ss{}er}
%\date{4. April 2014} % Activate to display a given date or no date (if empty),
         % otherwise the current date is printed 
%\setcounter{section}{-1} % sets the section counter to start with 0

\begin{document}
\maketitle %title (page)

%header and footer definitions for fancyhdr
\pagestyle{fancy}
\lhead{}
\chead{\leftmark}
\rhead{}
\lfoot{Sebastian Deußer}
\cfoot{}
\rfoot{Seite \thepage}

\thispagestyle{fancy}

\tableofcontents
\newpage

\section{Einleitung}
In den meisten Firmennetzwerken wird eine zentrale Stelle benötigt um verschiedene Anwender Metadaten zu verwalten. F\"ur diese Aufgabe wird \"ublicherweise ein Verzeichnisdienst verwendet. Basis eines Verzeichnisdienstes ist eine hierarchische Datenbank, die im Netzwerk verteilt angelegt sein kann. In dieser Datenbank k\"onnen Daten zu verschiedenen Objekten abgelegt werden wie etwa Konfigurationsdaten f\"ur Rechner oder diverse Daten f\"ur Anwender wie z.B. Name, Passwort, E-Mail Konto, Gruppenzugeh\"origkeiten u.s.w.. Die Anforderungen an einen Verzeichnisdienst wurden in den 80er Jahren von der International Telecommunication Union in der X.500 Spezifikation festgeschrieben nachdem weltweit Telekommunikationsunternehmen jahrzehntelang Praxiserfahrungen mit dem Thema bei Erstellung und Verwaltung von Telefon Verzeichnissen sammelten und dann zur Spezifikation beisteuerten.

In der Praxis stellte es sich als unpraktikabel f\"ur so gut wie alle Firmen au\ss{}er große Telekommunikationsunternehmen heraus die gesamte X.500 Spezifikation zu implementieren da dies einen hohen Implementierungsaufwand darstellt und der Betrieb sehr Hardware intensiv ist. Deswegen wurde an der Universität von Michigan 1993 LDAP (Lightweight Directory Access Protocol) entwickelt das ursprünglich als abgespeckte Alternative zu DAP (Directory Access Protocol) das traditionell zum Zugriff auf X.500 Verzeichnisse verwendet wurde. Der Ansatz von LDAP war den Zugriff durch den TCP/IP Protokollstapel zur ermöglichen, was weit weniger Implementierungsaufwand darstellt als die komplette Implementierung alles OSI-Schichten wie es bei DAP der Fall ist. Heutzutage ist LDAP das verbreiteste Protokoll zur Abfrage von Verzeichnisdiensten, u.a. auch da TCP/IP-basierte Netzwerke der verbreitesten Netzwerktyp ist. LDAP wird in vielen Anwendungen eingesetzt, vor allem im Adressbuchteil der meisten E-Mail Clients wie z.B. Apple Adressbuch, Microsoft Outlook, Mozilla Thunderbird und in Verzeichnisdiensten wie z.B. Microsoft Active Directory Services, Apple Open Directory und dem aussterbenden Pionier Novell eDirectory. Wie viele Protokolle ist LDAP Client/Server basiert.

In diesem Projekt wurde OpenLDAP verwendet, die verbreiteste freie Open Source Implementierung des Netzwerkstandards und steht unter einen eigenen freien Lizenz, der OpenLDAP Public License. OpenLDAP ist au\ss{}erdem die Referenzimplementierung des Standards, weswegen es in vielen Belangen (z.B. bei Schemadateien) mehr auf Protokollkonformit\"at als andere Implementierungen. Der OpenLDAP Server ist nicht nur das Abfrageprotokoll enthalten sondern auch ein Verzeichnisdienst. Es ist somit eine kosteng\"unstige L\"osung zum Aufbau eines Verzeichnisdienstes f\"ur viele Anwendungsf\"alle.

RADIUS (Remote Authentication Dial-In User Service) ist ebenfalls ein Client/Server basierendes Protokoll zur Authentifizierung, Autorisierung und Accounting (AAA-System) von Benutzern bei Einwahlverbindungen zu Netzwerken. RADIUS stellt den De-facto Standard zur zentralen Authentifizierung von Einwahlverbindungen wie z.B. über ISDN, DSL und WLAN (\"uber IEEE 802.1X) dar. \"Ublicherweise ist ein RADIUS-Server an einen Verzeichnisdienst angebunden um von ihm die Benutzerdaten f\"ur die Authentifizierung und Autorisierung (und teilweise auch Accounting) abzufragen.

Hier wurde FreeRADIUS verwendet, der laut Aussage des Projektes weltweit verbreiteste RADIUS Server. Das freie Open Source Projekt unter GPLv2 umfasst neben dem RADIUS Server au\ss{}erdem eine PAM Bibliothek, ein Apache Webserver Modul und eine Client Bibliothek (im Gegensatz zum Rest unter BSD Lizenz).

\section{Projektbeschreibung}
\subsection{Projektumfeld}
Die fgn GmbH wurde im August 2000 als SpinOff der Technische Universität (TU) Kaiserslautern gegründet. Die Gründer waren zuvor mehrere Jahre (seit 1996 bzw. 1989) als freischaffende Consultants und Trainer tätig. Die Firma pflegt enge Kontakte zum Regionalen Hochschulrechenzentrum Kaiserslautern (RHRK), da die meisten Mitarbeiter das Netz der TU Kaiserslautern mit ca. 10.000 Ports, Diensten wie Mail, DNS und DHCP in der Vergangenheit betreut haben oder es heute noch betreuen.

Die Kernkompetenz der fgn GmbH ist anspruchsvolles Netzwerk-Knowhow, welches als Dienstleistung in drei eng verknüpften Tätigkeitsfeldern angeboten wird: Schulungen, Workshops und Netzwerk-Consulting (Beratung und vor Ort Support von Firmen bei Problemen, Umstrukturierungen, Erweiterungen und Neuaufbau von Produktivnetzwerken).

\subsection{Ist-Analyse}
Im Praktikumsbetrieb fgn GmbH l\"auft der E-Mailverkehr und die Authentifizierung an den internen Webservices und OpenVPN Server \"uber einen alten CommuniGate Server (v5.0.13 von November 2006). Urspr\"unglich wurde dieser auf einem eigenen Rechner aufgesetzt, inzwischen aber wie viele andere Rechner der Firma virtualisiert. 

Die Webservices, die ihn zur Anwender-Authentifizierung verwenden, laufen auf drei anderen VMs, ebenso der OpenVPN Server. Sie kommunizieren mit dem LDAP-Teil von CommuniGate mittels den Apache Modul \texttt{mod\_ldap} und verwenden zur Autorisierung entsprechend \texttt{mod\_authnz\_ldap} (bzw. auf einem Rechner wegen eines veralteten Apache Webserver noch \texttt{mod\_auth\_ldap}). Apache pr\"uft dabei momentan nur auf Existenz des angegebene Benutzeraccounts und ob das richtige Passwort angegeben wurde, weitere Berechtigungen sind momentan nicht implementiert. Ausnahme dabei ist Nagios, aber dort werden die Berechtigungsgruppen intern verwaltet und nicht in LDAP abgelegt.

Lediglich die Anwenderkonten auf den Betriebssystemen der diversen Anwender-PCs sind nicht von CommuniGate abh\"angig. Da die fgn GmbH \"uber das Netz der Technischen Universit\"at Kaiserslautern angebunden ist m\"ussen alle E-Mails auch \"uber die Mailserver der Universit\"at laufen. Diese verwenden RADIUS um zu pr\"ufen ob die E-Mailkonten der fg-networking.de Domain tats\"achlich vorhanden sind. Der RADIUS Server dazu wird ebenso von CommuniGate bereitgestellt. 

Wegen der recht alten Softwareversion gibt es schon seit l\"angerem regelm\"a\ss{}ig Probleme, z.B. mit der SSL Authentifizierung neuerer E-Mail Clients (die unterst\"utzen Versionen von SSL/TLS benutzen aktuelle E-Mail Clients aus Sicherheitsgr\"unden nur noch ungern).

\subsection{Soll-Analyse}
Eine Aktualisierung von CommuniGate w\"are mit \"ahnlichem Aufwand verbunden wie ein komplettes Neuaufsetzen von Ersatzservern (und au\ss{}erdem mit dem Kauf einer neuen Lizenz verbunden w\"are). Deswegen wurde entschieden die Serverkomponenten E-Mail und Identity Management durch neue Server abzul\"osen. In der Firma momentan sehr viel freie Software verwendet wird sollen die Ersatzserver auch auf auf Basis von freie Software aufgesetzt werden. Da die E-Mail Infrastruktur sehr kritisch f\"ur die Arbeit der Firma ist und bei der Migration der E-Mailkonten mit vielen vertraulichen firmeninternen Informationen hantiert werden muss soll die Installation des neuen E-Mail Servers von einem Mitarbeiter von fgn durchgef\"uhrt werden.

F\"ur das Identity Management sollen OpenLDAP und FreeRADIUS zum Einsatz kommen da dies die verbreitesten freien Implementierungen von LDAP und RADIUS sind und somit das meiste Know How in B\"uchern und dem Internet verf\"ugbar ist. Au\ss{}derdem erleichert es die sp\"atere Pflege des Systems da sich f\"ur solche Verbreiteten Systeme Personal mit Fachkenntnis finden lassen als f\"ur die meisten anderen propriet\"aren Implementierungen.

F\"ur den E-Mail Teil soll ein postfix Server verwendet werden, allerdings war dieser zur Fertigstellung dieses Projektes noch nicht einsatzf\"ahig. Die Intergration des neuen E-Mailservers ist damit nicht Teil dieses Projektes und wird zu einem sp\"ateren Zeitpunkt durchgef\"uhrt. Es werden lediglich die Vorarbeiten auf der Seite vom Identity Management get\"atigt die dazu unabh\"angig von der endg\"ultigen Konfiguration des neuen E-Mail Servers sind.\newline

\noindent 		%verhindert einmalig Einrueckung
Somit m\"ussen im Rahmen der Projektarbeit folgende Arbeiten durchgef\"uhrt werden: 
\begin{itemize}
\item Linux Grundsystem installieren
\item OpenLDAP Server auf dem Linux System installieren
\item OpenLDAP Server konfigurieren
\item Verzeichnisstruktur erstellen
\item grundlegende Benutzerdaten (Name, Passwort) aus bestehendem System \"ubernehmen
\item FreeRADIUS Server installieren
\item FreeRADIUS Server konfigurieren und an den OpenLDAP Server anbinden
\item Absicherung des Systems
\item Umkonfiguration der entsprechenden Systeme auf den neuen LDAP-Server
\item Funktions- und Sicherheitstests
\end{itemize}


\subsection{Vorgaben}
\subsubsection{Wirtschaftliche Vorgaben}
Wie die meisten anderen Server der fgn GmbH soll der Server in einer neuen VM auf einem bereits bestehenden VMware ESXi-Servers laufen, somit fallen keine zus\"atzlichen Hardwarekosten an. Da in dem Projekt  ausschlie\ss{}lich freie Software zum Einsatz kommen soll fallen auch keine zus\"atzlichen Softwarelizenzkosten an.

\subsubsection{Organisatorische Vorgaben}
Das Projekt wird im Praktikumsbetrieb mit Unterst\"utzung des Mitarbeiters Erik Auerswald durchgef\"uhrt.
Zur Projektdokumentation werden zus\"atzlich jeweils eine Kundendokumentation für Administratoren im firmeninternen Wiki erstellt. Hinzu kommen Funktions- und Sicherheitstests zur Qualit\"atssicherung. Eine Anwenderdokumentation ist nicht notwendig da sich aus Sicht des Anwenders nichts gegen\"uber des Ausgangszustandes \"andern soll.

\subsubsection{Zeitliche Vorgaben}
Das Projekt wird im Zeitraum vom 04.05.2015 – 18.05.2015 durchgef\"uhrt, wobei die Bearbeitungszeit von 35 Stunden nicht \"uberschritten werden darf.

\section{Projektplanung}
\subsection{Planung des Ersatzservers}
Als Ersatzserver wird eine neue Virtual Machine auf einen der firmeneigenen ESXi Servern verwendet. Als Betriebssystem wird Debian Stable (zur Durchf\"uhrungszeit des Projektes Version 8.0 Jessie) mit der Standardpaketauswahl ohne zus\"atzliche Vorauswahlen.

\subsection{Planung des Kommunikationskonzepts}
S\"amtliche firmeninterne Webservices, der VPN, der kommende neue E-Mailserver und der FreeRADIUS Daemon greifen \"uber das LDAP-Protokoll auf den OpenLDAP Daemon zu um Benutzer zu authentifizieren und Einstellungen zu diesen zu holen. Der E-Mailserver der Universit\"at greift auf den FreeRADIUS Daemon zu um Benutzerkonten zu pr\"ufen, welcher wiederrum die Daten beim OpenLDAP-Daemon erfragt.

\subsection{Planung des Sicherheitskonzepts}
Das Firmennetzwerk der fgn GmbH ist \"uber die Netzwerk der TU Kaiserslautern ans Internet angebunden. Entsprechend wird jeglicher Netzwerkverkehr von der TU Firewall vorgefiltert. Das Firmennetz von fgn ist zus\"atzlich noch durch eine eigene Firewall gesichert. In diese muss eine Ausnahme f\"ur den neuen RADIUS Server eingetragen werden (analog zur alten Regel). Zus\"atzlich wird auf dem neuen Server eine Software-Firewall installiert die nur die f\"ur LDAP, RADIUS und zur Wartung ben\"otigten Ports zulassen soll.

\subsection{Projektablaufplan}
Analyse und Planung (insgesamt 6 h)
	\begin{itemize}
	\item Ist-Analyse (3 h)
		\begin{itemize}
		\item Analyse des bestehenden CommuniGate Servers und der damit verbundenen Webservices (2 h)
		\item Aufnahme der Anforderungen an einen Ersatzserver (1 h)
		\end{itemize}
	\item Planung (3 h)
		\begin{itemize}
		\item Ausarbeitung eines Konzepts für den Ersatzserver (1 h)
		\item Ausarbeitung des Sicherheitskonzepts (unter Berücksichtigung des Firmenkonzepts) (1 h)
		\item Ausarbeitung des Kommunikationskonzepts (Serverdienste untereinander und extern) (1 h)
		\end{itemize}
	\end{itemize}
Umsetzung (insgesamt 20 h)
	\begin{itemize}
	\item Vorbereitungen (6 h)
		\begin{itemize}
		\item Dokumentation der Konfiguration des zu ersetzenden Servers (2 h)
		\item Erstellen einer Liste aller Dienste, die das bestehende Identity Management nutzen (2 h)
		\item Pr\"ufung von M\"oglichkeiten zum Importieren der bestehenden Anwender-Konten in die neue L\"osung (2 h)
		\end{itemize}
	\item Installation und Einrichtung des neuen Servers (8 h)
		\begin{itemize}
		\item Grundinstallation des Linux Systems des neuen Servers (1 h)
		\item Installation und Konfiguration des LDAP Servers (3 h)
		\item Installation und Konfiguration des RADIUS Servers (2 h)
		\item Absicherung des Rechners (Firewall etc.) (2 h)
		\end{itemize}
	\item Abschlie\ss{}ende Arbeiten (6 h)
		\begin{itemize}
		\item Umkonfiguration der Webservices (2 h)
		\item Import/Anlegen der Anwender Konten (2 h)
		\item Funkions- und Sicherheitstests (2 h)
		\end{itemize}
	\end{itemize}
Dokumentation (insgesamt 9 h)
	\begin{itemize}
	\item Erstellen der Projektdokumentation (8 h)
	\item Erstellen der Dokumentation für das firmeninterne Wiki (1 h)
	\end{itemize}

\section{Realisierung}
\subsection{Vorbereitungen}
\subsubsection{Dokumentation der Konfiguration des zu ersetzenden Servers}
\hilight{TODO}
\subsubsection{Erstellen einer Liste aller Dienste die das bestehende Identity Management nutzen}
\hilight{TODO}
\subsubsection{Pr\"ufung von M\"oglichkeiten zum Importieren der bestehenden Anwender-Konten in die neue L\"osung}
\hilight{TODO}

\subsection{Installation und Einrichtung des neuen Servers}
\subsubsection{Grundinstallation des Linux Systems des neuen Servers}
\hilight{TODO}
\subsubsection{Installation und Konfiguration des LDAP Servers}
\hilight{TODO}
\subsubsection{Installation und Konfiguration des RADIUS Servers}
\hilight{TODO}
\subsubsection{Absicherung des Rechners (Firewall etc.)}
\hilight{TODO}

\subsection{Abschlie\ss{}ende Arbeiten}
\subsubsection{Umkonfiguration der Webservices}
\hilight{TODO}
\subsubsection{Import/Anlegen der Anwender Konten}
\hilight{TODO}
\subsubsection{Funkions- und Sicherheitstests}
\hilight{TODO}


\newpage
\section{Qualit\"atssicherung}
\section{Projektkosten}
\section{Projektabschluss}
\subsection{Fazit}
\subsection{Ausblick}


%\section{Testkapitel}
%\subsection{Testsektion}
%Dies ist eine Testaufz\"ahlung:
%\begin{itemize}
%\item Item 1
%\item Item 2
%\item Item 3
%\item Item 4
%\end{itemize}
%Und weiter im Text.
%
%Hier bricht die Seite nun um.
%
%\Blinddocument
%\newpage


\end{document}
