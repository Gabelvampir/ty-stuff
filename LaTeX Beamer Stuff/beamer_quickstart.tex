\documentclass[a4paper,presentation]{beamer}

\usepackage{latexsym}
\usepackage{exscale}

\usepackage{amssymb}
\usepackage{amsmath}

\usepackage{amsbsy}

%\usepackage{amsopn}
%\DeclareMathOperator{\supp}{supp}

%\usepackage{pdfpages}

%\usetheme{CambridgeUS} % 2.5d red/grey
%\usetheme{Frankfurt} % 3d blue/black
%\usetheme{Luebeck} % 2d blue/black
%\usetheme{Madrid} % 2.5d blue
%\usetheme{Rochester} % 2d blue
%\usetheme{Singapore} % blue shade at top, plain
\usetheme{Warsaw} % 3d blue/black

% \AtBeginSubsection[]
% {
%   \begin{frame}<beamer>
%     \frametitle{Structure}
%     \tableofcontents[currentsection,currentsubsection]
%   \end{frame}
% }

\AtBeginSection[]
{
  \begin{frame}<beamer>
    \frametitle{Structure}
    \tableofcontents[currentsection]
  \end{frame}
}

\title[Beamer quickstart]{An hands-on introduction to the beamer package}
\author{Andreas N\'emeth}
\institute[Name of the Talk]{University of Vienna\\Faculty of Mathematics}
\date{January 2006}

\begin{document}

\begin{frame}
  \titlepage
\end{frame}

\begin{frame}
  \frametitle{Structure}
  \tableofcontents
  % Die Option [pausesections] k�nnte n�tzlich sein.
\end{frame}

\section{Introduction}

\subsection[Structure of Talk]{The structure of the talk mapped to slides}

\begin{frame}
  \frametitle{How to give a talk using a beamer}
  \large
  \begin{itemize}
    \item<2-> Use the \LaTeX {\tt beamer} package
    \item<3-> Set the options {\tt handout} and {\tt presentation} accordingly
    \item<4-> Don't squeeze too much into the slides
    \item<5-> Stand right next to the projection but don't cover the content
    \item<6-> The audience is distracted more easily by your slides -- SPEAK UP!
  \end{itemize}
\end{frame}

\subsection[Structure of Slides]{The possible structure of slides using the beamer package}

\begin{frame}
  \frametitle{Layout of your talk}
  \large
  The following types of slides will be introduced:
  \begin{itemize}
    \item[a)]<2-> Slide with several blocks
    \item[b)]<3-> Slide with block \& note
    \item[c)]<4-> Slide without anything special on it
    \item[d)]<5-> Slide with highlighted formula
    \item[e)]<6-> Slide for bibliography
  \end{itemize}
\end{frame}

\section{Slides}

\subsection[Boxes]{Slides with boxes}

\begin{frame}
  \frametitle{Slide with several blocks}
  \uncover<2->{
    \begin{block}
      {Title of first block}
      This is a generic block with just text in it.
    \end{block}
  }
  \uncover<3->{
    \begin{block}
      {A block with a list of items}
      \begin{itemize}
        \item<4-> Item 1
        \item<5-> Item 2
        \item<6-> Item 3
        \item<7-> Item 4
        \item<8-> Item 5
      \end{itemize}
    \end{block}
  }
  \uncover<9->{
    \begin{block}
      {Title of third block}
      There is nothing special about this block, either.
    \end{block}
  }
\end{frame}

\begin{frame}
  \frametitle{Slide with block and annotation}
  \begin{block}{The first block with an item list}
    \begin{itemize}
      \item[(i)]<2-> Item 1
      \item[(ii)]<3-> Item 2
      \item[(iii)]<4-> Item 3
      \item[(iv)]<5-> Item 4
      \item[(v)]<6-> Item 5
    \end{itemize}
  \end{block}
  \uncover<7->{
    {\bf Note:}
    Such a note can be easily added and makes a easy to see divide between the important block and the annotation beneath. 
  }
\end{frame}

\subsection{Empty slides}

\begin{frame}
  \frametitle{Empty slide}
  This slide has nothing special on it, so the audience can grab the content more efficiently.
  
  Additionally, the author has more space to fill with his important content.
\end{frame}

\subsection[Formulas]{Important formulas}

\begin{frame}
  \frametitle{Slide with important formula}
  In order to assure that your audience grabs the most important element of this slide, you can add a box around your centered key element:
  \begin{center}
    {\textcolor{blue}{\fbox{$E = mc^2$\vphantom{\Big)}}}}
  \end{center}
\end{frame}

\subsection{Bibliography}

\begin{frame}
  \frametitle{Bibliography}
  \begin{itemize}
    \item<2-> Title of first cited book/paper/work
      \newline \hspace*{\fill} {\tiny [Author, Publisher, Year]}
    \item<3-> Title of second cited book/paper/work
      \newline \hspace*{\fill} {\tiny [Author, Publisher, Year]}
    \item<4-> Title of third cited book/paper/work
      \newline \hspace*{\fill} {\tiny [Author, Publisher, Year]}
    \item<5-> Title of fourth cited book/paper/work
      \newline \hspace*{\fill} {\tiny [Author, Publisher, Year]}
    \item<6-> Title of fifth cited book/paper/work
      \newline \hspace*{\fill} {\tiny [Author, Publisher, Year]}
  \end{itemize}
\end{frame}

\end{document}
